\pdfprotrudechars2
\pdfadjustspacing2
\pdfminorversion=7
\pdfcompresslevel=9
\pdfobjcompresslevel=3
\documentclass[initial=ZallmanCaps,staff=19,font=greciliae,11pt,a4paper,openany,twoside,choralsign=PfefferMediaeval]{gregorian}
\usepackage[a4paper,twoside,inner=3cm,outer=1.5cm,top=2cm,bottom=2cm]{geometry}
\usepackage[ngerman,pdfstartview={FitH},bookmarksopen,colorlinks=true,menucolor=black,urlcolor=black,linkcolor=black]{hyperref}
\newfontfeature{Microtype}{protrusion=default;expansion=default;}
\setmainfont[Microtype,Ligatures={Common,TeX},Numbers=OldStyle]{Linux Libertine O}
\setsansfont[Microtype,Ligatures={Common,TeX},Numbers=OldStyle,SmallCapsFont={Alcuin URW SC}]{Alcuin URW}
\setmonofont[Microtype,Ligatures={TeX}]{Inconsolata}
\sloppy
\pagestyle{myheadings}
\makefootrule{myheadings}{\textwidth}{0.4pt}{\footruleskip}
\makeheadrule{myheadings}{\textwidth}{0pt}
\makeevenfoot{myheadings}{\thepage}{\gebet}{}
\makeoddfoot{myheadings}{}{\gebet}{\thepage}
\makeevenhead{myheadings}{\rightmark}{}{}
\makeoddhead{myheadings}{\hphantom{\thechapter}}{}{}
\hbadness 99999
\renewcommand{\greabovelinestextstyle}{\scriptsize}
\begin{document}
\frontmatter
\thispagestyle{empty}
\ThisTileWallPaper{21cm}{\paperheight}{lskj.pdf}
\phantom{l}
\vskip12em
\begin{center}
{\Huge\bfseries\sffamily GREGORIANISCHES BREVIER}\par\vskip0.5em
{\Huge\bfseries\sffamily STUNDENGEBETBUCH}\par
{\sffamily\scshape Laudes -- Sext -- Vesper -- Complet}\par
{\sffamily\scshape Reisesegen -- lateinische Vesper und Complet}\par
\vskip3em
{\large\bfseries\sffamily\scshape Lutherhaus Jena}
 \vfill
\end{center}
\newpage
\thispagestyle{empty}
\textit{gewidmet Dr. Bernhard Gröbler}\par
\vfill
\begin{center}
\textsc{Ich danke allen,}\par
\textsc{die an der Erstellung dieses Buches direkt und indirekt beteiligt waren,
insbesondere:}\par\vskip0.5em
\textsc{Dem Liturgischen Singkreis Jena und seinen Mitgliedern,}\par
\textsc{Élie Roux und dem Gregorio-Team für Gregorio, mit dem dieses Buch gesetzt wurde,}\par
\textsc{Meiner Familie und meinen Freunden für Unterstützung bei der Erstellung,}\par
\textsc{Der Erzabtei St. Ottilien, an der ich mit dem Gregorianischen Choral in Berührung kam.}\par
\end{center}
\vfill
\begin{framed}
3. Aufl. \textcopyright~2010. \textit{Auf der Grundlage des Buches Liturgie III und des Breviarium Lipsiens\ae{} der Evangelisch Lutherischen Gebetsbruderschaft erstellt und bearbeitet von:}\par
David Gippner M.A.\par
Hans-Berger-Straße 20\par
07747 Jena.\par
\textit{\href{http://creativecommons.org/licenses/by-nc-sa/3.0/de/}{\includegraphics{88x31.png}} Dieses Werk steht unter einer \href{http://creativecommons.org/licenses/by-nc-sa/3.0/de/}{Creative Commons Namensnennung-Keine kommerzielle Nutzung-Weitergabe unter gleichen Bedingungen 3.0 Deutschland Lizenz}.}\par\vskip0.5em
\end{framed}
\newpage
\tableofcontents\vfill
\begin{center}
\scalebox{1.6}{\Huge\initfamily SOLI DEO GLORIA}
\end{center}
\vfill
\mainmatter
\chapter[Einführung]{Einführung in das Gregorianische Brevier}
\def\gebet{\textsc{Einführung}}
Im vorliegenden Buch findet man eine Auswahl aus dem \textit{Breviarium
Lipsiensæ}, dem Leipziger Brevier der Evangelisch-Lutherischen
Gebetsbruderschaft. Diese Auswahl ist so getroffen, dass man ganzjährig die
folgenden vier Gebete in einer Grundform beten kann:\par
\begin{asparaenum}
\item Laudes -- das Morgenlob
\item Sext -- das Mittagsgebet
\item Vesper -- das Abendgebet und
\item Complet -- das Nachtgebet.
\end{asparaenum}
Zusätzlich sind im Anhang reichere Responsorien für die Vesper sowie ein
Reisesegen, das Vaterunser, ein lateinisches Credo und ein Psalmvers und Hymnus
für die Complet in der Adventszeit abgedruckt.\par
Laudes, Sext und Vesper sind teilweise an die Kirchenjahreszeit angepasst; der
Charakter eines Auszugs verbietet aber eine vollständige Anpassung, andernfalls
läge mit diesem Band eine Kopie des Breviers vor.
\section*{Bestandteile}
\begin{asparaenum}
\item Psalmen mit Antiphonen und alttestamentliche Cantica
\item Neutestamentliche Cantica:
\begin{asparaitem}[$\triangleright$]
 \item Das \enquote{Benedictus} (Lobgesang des Zacharias, Lk.1,68-79) in den
Laudes
\item Das \enquote{Magnificat} (Lobgesang der Maria, Lk. 1,46-55) in der Vesper
\item Das \enquote{Nunc dimittis} (Lobgesang des Simeon, Lk. 2,29-32) in der
Complet
\end{asparaitem}
\item Lesungen aus der Bibel nach dem Kirchenjahr (Texte sind in dieser Ausgabe
nicht enthalten)
\item Fürbitten
\end{asparaenum}
Die Ordnung dieses Buches reiht die Tagzeitengebete in ihrer Reihenfolge auf, am
Morgen beginnend. Im Anhang des Buches findet sich der Reisesegen, der vor
Antritt einer Reise gebetet werden kann sowie zwei Responsoria Prolixa für die
Vesper am Sonntag und ein gemeinschaftlich gesungenes Vaterunser und
lateinisches Credo für den Gebrauch in der Complet.
\section*{Wie betet man die einzelnen Teile?}
Das Grundprinzip der Tagzeitengebete ist responsorial, d.h. in den Gebeten gibt
es ständigen Wechsel zwischen einem Vorbeter oder Amtsträger (gekennzeichnet mit
\Vbar) und der versammelten Gebetsgemeinde (gekennzeichnet mit \Rbar). An
Stellen, bei denen alle gemeinsam beten, ist dies mit \Abar gekennzeichnet.\par
Daneben gibt es für die Psalmen und Hymnen eine Einteilung in zwei Gruppen, die
mit römischen Zahlen gekennzeichnet sind. Gruppe I ist auf der Seite des
Kantors, Gruppe II die gegenüberliegende.\par
Am Anfang der Gebete steht der Ingressus, der den Beginn des Gebets markiert.
Dabei steht die ganze Gebetsgemeinde.\par
In der Complet geht ihm ein Segen, eine Lesung und ein gemeinsames
Schuldbekenntnis voran.\par
In der Laudes, Vesper und Complet folgt darauf die Psalmodie. Die Antiphon und
der erste Psalmvers werden dabei vom Cantor (markiert mit \Vbar) bis zum Stern
angestimmt, in der Antiphon fallen alle ein, im ersten Psalmvers die Gruppe I.
Am Ende des Psalms wird die Antiphon wiederholt.\par
Responsorien stimmt der Kantor an und wird ab dem Stern dabei von der Schola
(aus anwesenden Amtsträgern) unterstützt. Die gesamte Gebetsgemeinde wiederholt
den Gesang, nach dem von der Schola gesungenen Vers nur die zweite Hälfte, nach
dem Gloria Patri wieder den gesamten Responsorialvers.\par
Beim Hymnus wird die erste Strophe vom Kantor angestimmt und er dann ab dem
Stern von Gruppe I unterstützt, danach die Strophen im Wechsel gesungen. Vor der
letzten Strophe steht die Gebetsgemeinde auf und bleibt bis zum Ende des Gebetes
stehen.\par
Folgende Ämter gibt es:\par
\begin{asparaenum}
\item Lektor -- er/sie liest die Lesungen und das Capitel in der Complet.
\item Kantor -- er/sie stimmt Gesänge an (Psalmodie, Cantica, Responsorium).
\item Hebdomadarius -- \enquote{Wochendiensthaber}; er/sie betet die Gebete vor.
\item Præses Chori -- geistlicher Leiter des Gebetes; er/sie erteilt den
Lesesegen und den Abschlusssegen.
\end{asparaenum}
\section*{Besonderheiten}
Von Aschermittwoch bis Ostersonntag entfällt das Halleluja in allen Gebeten und
wird durch \enquote{Lob sei Dir, \textsc{Herr}, du König der ewigen
Herrlichkeit} ersetzt. Vom Sonntag Judica bis Karsamstag entfällt das Gloria
Patri in den Responsorien. Für Himmelfahrt und St. Michael finden sich für
Laudes, Sext und Vesper gesonderte Versikel.
\section*{Quadratnotation und wie sie gesungen wird\protect\footnote{\textbf{Quelle:} Bernhard K. Gröbler, Einführung in den Gregorianischen Choral, 2. Aufl. Jena 2005, S. 144.}}
\subsection*{Schlüssel}
\includegabcscore{schluessel.gabc}
\includegabcscore{schluesself.gabc}
Die vom Schlüssel umschlossene Linie markiert das c bzw. f.
\subsection*{Einzeltöne}
\includegabcscore{einzeltoene.gabc}\par
\subsection*{Schnelle Tonverbindungen (Ligaturen)}
\includegabcscore{ligaturen.gabc}
Umschrift:\par
{%
\parindent 0pt
\ifx\preLilyPondExample \undefined
\else
  \expandafter\preLilyPondExample
\fi
\def\lilypondbook{}%
\input 4e/lily-23d0a746-systems.tex
\ifx\postLilyPondExample \undefined
\else
  \expandafter\postLilyPondExample
\fi
}\par
Der letzte Ton von Ligaturen ist stets leicht gedehnt bzw. artikuliert.
\subsection*{Gedehnte Noten}
\includegabcscore{episem.gabc}
Bei Clivis und Pes gilt das Episem für beide Töne, wenn es bei der ersten Note
steht, sonst gilt es nur für den zweiten Ton.
\subsection*{\enquote{Liqueszierende} Noten (Liqueszenzneumen)}
\includegabcscore{liqueszenz.gabc}
Die kleine Note wird auf einen klingenden Konsonant (Semivokal) am Silbenende
gesungen. Auch am Ende einer Ligatur sind entsprechende Liqueszenznoten möglich.
\subsection*{Neographien}
\includegabcscore{quilisma.gabc}
Die gezackte Note ist ein kurzer, schwacher Ton, der zum Zielton leitet und kann
ähnlich einem Glissando gesungen werden.\par
\includegabcscore{neographien.gabc}
\subsection*{Reperkussion}
\includegabcscore{reperkussion.gabc}
Der doppelte Ton wird auf gleicher Höhe neu angesetzt.
\subsection*{Custos}
Der Custos ist eine halbe Note am Ende einer Notenzeile, der den nachfolgenden
Ton auf der nächsten Zeile anzeigt.\par
\section*{Lektionstöne\protect\footnote{Quelle: Breviarium Lipsiensae 1988, S. 78f.}}
Für die nicht vorgegebenen Lesungen in Laudes, Sext und Vesper gelten die untenstehenden Melodiemodelle. Der jeweilige Abschluss (Conclusio) findet sich in den Gebeten jeweils an der Stelle der Lesung.\par
\subsection*{Laudes und Vesper sonntags}
\includegabcscore{lt-lave-so.gabc}
\subsection*{Laudes und Vesper werktags, Sext sonntags}
\includegabcscore{lt-lave-we.gabc}
\subsection*{Sext werktags}
\includegabcscore{lt-se-we.gabc}
\chapter[Laudes]{\textsc{Laudes}}
\def\gebet{\textsc{Laudes}}
\section*{Ingressus}
\ueberinitiale{sonn-}{tags}
\includegabcscore{ingressus-so-lave.gabc}\par
\ueberinitiale{werk-}{tags}
\includegabcscore{ingressus-wo-lave.gabc}\newpage
\section*{Alttestamentliches Canticum (fakultativ)}
\commentary{\textbf{Lied der Hannah}}
\ueberinitiale{\Abar}{VII a}
\includegabcscore{laudes-lh.gabc}
\begin{multicols}{2}\setlength{\columnseprule}{0.2pt}
 Mein Mund hat sich aufgetan wider 'meine Feinde, \grestar{} denn ich freue mich
'deines Heiles.\par
 \einzug{Es ist niemand heilig wie der \textsc{Herr}, \gredagger{} außer
'Dir ist keiner \grestar{} und ist kein Fels wie 'unser Gott ist.}
 Lasst euer großes 'Rühmen und Trotzen; \grestar{} freches Reden gehe nicht aus
'eurem Munde.\par
 \einzug{Denn der \textsc{Herr} ist ein Gott, 'der es merket \grestar{} und von
ihm werden 'Taten gewogen.}
 Der Bogen der Starken 'ist zerbrochen \grestar{} und die Schwachen sind
um'gürtet mit Stärke.\par
 \einzug{Die da satt waren, müssen 'um Brot dienen \grestar{} und die Hunger
litten, 'hungert nicht mehr.}
 Die Unfruchtbare hat 'sieben geboren \grestar{} und die viele Kinder hatte,
'welket dahin.\par
 \einzug{Der \textsc{Herr} tötet und 'macht lebendig \grestar{} führt hinab zu
den Toten und 'wieder herauf.}
 Der \textsc{Herr} macht 'arm und macht reich \grestar{} er er'niedrigt und
erhöht.\par
 \einzug{Er hebt auf den Dürftigen 'aus dem Staube \grestar{} und erhöht den
Armen 'aus der Asche.}
 Dass er ihn setze 'unter die Fürsten \grestar{} und den Thron der Ehre 'erben
lasse.\par
 \einzug{Denn der Welt Grundfesten 'sind des Herren \grestar{} und er hat die
'Erde daraufgesetzt.}
 Er wird behüten die Füße seiner Heiligen \gredagger{} aber die Gottlosen
sollen zunichte 'werden in Finsternis, \grestar{} denn viel Macht 'hilft doch
niemand.\par
 \einzug{Die mit dem Herrn hadern, sollen zugrunde gehen. \gredagger{} Der
Höchste im Himmel wird 'sie zerschmettern; \grestar{} der \textsc{Herr} wird
richten 'der Welt Enden.}
Ehre sei dem \textsc{Vater} 'und dem \textsc{Sohne} \grestar{} und dem
'\textsc{Heiligen Geiste}.\par
\einzug{Wie im Anfang, so auch 'jetzt und allezeit \grestar{} und in 'Ewigkeit.
Amen.}
 \textit{$\rightarrow$ Antiphon}
\end{multicols} \vskip0.4em
\newpage
\section*{Psalmodie}
\ueberinitiale{1. \Abar}{IIIa}
\commentary{\textbf{Psalm 51}}
\includegabcscore{laudes-1ant.gabc}
\begin{multicols}{2}\setlength{\columnseprule}{0.2pt}
 Wasche mich rein von \underline{mei}ner 'Missetat: \grestar{} und reinige mich von \underline{mei}ner 'Sünde.\par
 \einzug{Denn ich erkenne \underline{mei}ne 'Missetat: \grestar{} und meine \underline{Sün}de ist 'stets vor mir.}
 An Dir allein habe \ulin{ich} ge'sündigt: \grestar{} und \ulin{ü}bel vor 'Dir getan.\par
 \einzug{Damit Du Recht behaltest in \underline{Dei}nen 'Worten: \grestar{} und rein dastehst, \underline{wenn} Du 'richtest.}
 Siehe, ich bin als \underline{Sün}der ge'boren: \grestar{} und meine Mutter hat mich in \underline{Sün}den emp'fangen.\par
 \einzug{Siehe, Dir gefällt Wahrheit, die \underline{im} Ver'borgnen liegt: \grestar{} und im Geheimen tust \underline{Du} mir 'Weisheit kund.}
 Entsündige mich mit Ysop, dass \underline{ich} rein 'werde: \grestar{} wasche mich, dass ich \underline{schnee}weiß 'werde.\par
 \einzug{Lass mich hören \underline{Freu}de und 'Wonne: \grestar{} dass die Gebeine fröhlich werden, die \underline{Du} zer'schlagen hast.}
 Verbirg dein Antlitz vor \underline{mei}nen 'Sünden: \grestar{} und tilge alle \underline{mei}ne 'Missetat.\par
 \einzug{Schaffe in mir, \underline{\textsc{Gott}}, ein 'reines Herz: \grestar{} und gib mir einen be\underline{stän}digen, 'neuen Geist.}
 Verwirf mich nicht von \underline{Dei}nem 'Angesicht: \grestar{} und nimm Deinen \underline{heil}gen Geist 'nicht von mir.\par
 \einzug{Erfreue mich wieder mit \underline{Dei}ner 'Hilfe: \grestar{} und mit einem willigen \underline{Gei}ste 'stärke mich.}
 Ich will die Übertreter Deine \underline{We}ge 'lehren: \grestar{} dass sich die Sünder zu \underline{Dir} be'kehren.\par
 \einzug{Er\-ret\-te mich von Blut\-schuld, \textsc{Gott}, der Du mein \underline{\textsc{Gott}} und 'Hei\-land bist: \grestar{} \hfill dass meine Zunge deine Ge\underline{rech}\-tigkeit 'rühme.}
 \textsc{Herr}, tue \underline{mei}ne 'Lippen auf: \grestar{} dass mein Mund deinen \underline{Ruhm} ver'kündige.\par
 \einzug{Denn du willst keine Schlachtop\-\ulin{fer} \gredagger{} ich wollte sie \underline{Dir} sonst 'ge\-ben: \grestar{} und dir gefallen \underline{kei}ne Brand'op\-fer.}
 Die Opfer, die \textsc{Gott} gefallen, sind eine ge\underline{beug}te 'Seele: \grestar{} ein geängstetes, zerschlagenes Herz wirst Du, \textsc{Gott}, \underline{nicht} ver'achten.\par
 \einzug{Tue wohl an Zion nach \underline{Dei}ner 'Gnade: \grestar{} baue die Mauern \underline{zu} Je'rusalem.}
 Dann werden Dir rechte Op\-fer gefal\ulin{len} \gredagger{} Brandop\-fer \underline{und} Ganz'op\-fer: \grestar{} dann wird man Stiere auf Deinem \underline{Al}\-tar 'op\-fern.\par
 \einzug{Ehre sei dem \textsc{Vater} \underline{und} dem '\textsc{Sohne}: \grestar{} und dem \underline{\textsc{Hei}}\textsc{ligen 'Geiste}.}
 Wie im Anfang, so auch \ulin{jetzt} und alle'zeit: \grestar{} und in \ulin{E}wigkeit. 'Amen.\par
  \textit{$\rightarrow$ Antiphon}
\end{multicols} \vskip0.4em
\newpage
\commentary{\textbf{Psalm 108}}
\ueberinitiale{2. \Abar}{IVb}
\includegabcscore{laudes-2ant.gabc}
\begin{multicols}{2}\setlength{\columnseprule}{0.2pt}
Wach auf, Psal\underline{ter} und 'Harfen: \grestar{} ich will das
\underline{Mor}genrot 'wecken.\par
 \einzug{Ich will Dir danken, \textsc{Herr}, un\underline{ter} den 'Völkern:
\grestar{} ich will dir lobsingen \underline{un}ter den 'Leuten.}
 Denn Deine Gnade reicht, so \underline{weit} der 'Himmel ist: \grestar{} und
deine Treue, so weit \underline{die} Wolken 'gehen.\par
 \einzug{Erhebe Dich, \textsc{Gott}, ü\underline{ber} den 'Himmel: \grestar{}
und Deine Herrlichkeit ü\underline{ber} alle 'Lande.}
 Lass Deine Freunde er\underline{ret}tet 'werden: \grestar{} dazu hilf mit
Deiner Rech\underline{ten} und er'höre uns.\par
 \einzug{\textsc{Gott} hat in Seinem Heiligtum geredet - ICH will
frohloc\underline{ken} \gredagger{} ICH will Si\underline{chem} ver'teilen:
\grestar{} und das Tal \underline{Suk}kot aus'messen.}
 Gilead ist mein, Manasse ist auch \underline{mein} \gredagger{} und Ephraim ist
der Schutz \underline{mei}nes 'Hauptes: \grestar{} Ju\underline{da} ist mein
'Zepter.\par
 \einzug{Moab ist mein Waschbec\underline{ken} \gredagger{} ich will meinen
Schuh auf \underline{E}dom 'werfen: \grestar{} über die Philis\underline{ter}
will ich 'jauchzen.}
 Wer wird mich führen \underline{in} die 'feste Stadt: \grestar{} wer wird mich
\underline{nach} Edom 'leiten?\par
 \einzug{Wirst Du es nicht tun, \textsc{Gott}, der Du \underline{uns}
ver'stoßen hast: \grestar{} und ziehst nicht aus, \textsc{Gott},
\underline{mit} unserm 'Heere?}
 Schaffe uns \underline{Bei}stand 'vor dem Feind: \grestar{} denn
Menschenhil\underline{fe} ist nichts 'nütze.\par
 \einzug{Mit \textsc{Gott} wol\underline{len} wir 'Taten tun: \grestar{} Er wird
unsere Fein\underline{de} nieder'treten.}
 Ehre sei dem \textsc{Vater} \underline{und} dem '\textsc{Sohne}: \grestar{} und
dem \textsc{\underline{Hei}ligen 'Geiste}.\par
 \einzug{Wie im Anfang, so auch \underline{jetzt} und 'allezeit: \grestar{} und
in \underline{E}wig'keit. Amen.}
  \textit{$\rightarrow$ Antiphon}
\end{multicols} \vskip0.4em
\commentary{\textbf{Psalm 147j}}
\ueberinitiale{3. \Abar}{Ia}
\includegabcscore{laudes-3ant.gabc}
\begin{multicols}{2}\setlength{\columnseprule}{0.2pt}
Der \textsc{Herr} baut \underline{auf} Je'rusalem: \grestar{} und bringt
zusammen die Ver\underline{streu}ten 'Israels.\par
 \einzug{Er heilet, die zer\underline{broch}enen 'Herzens sind: \grestar{} und
verbindet \underline{ih}re 'Wunden.}
 Er \underline{zäh}let die 'Sterne: \grestar{} und nennt sie al\underline{le}
mit 'Namen.\par
 \einzug{Unser \textsc{Herr} ist \underline{groß} und von 'großer Kraft:
\grestar{} und unbegreiflich ist, wie \underline{Er} re'gieret.}
 Der \textsc{Herr} richtet \underline{auf} die 'Elenden: \grestar{} und stößt
die Gottlo\underline{sen} zu 'Boden.\par
 \einzug{Singet umeinander dem \underline{Herrn} ein 'Danklied: \grestar{} und
lobet unsern \underline{\textsc{Gott}} mit 'Harfen.}
 Der den Himmel mit Wolken bedec\underline{ket} \gredagger{} und gibt
\underline{Re}gen auf 'Erden: \grestar{} der Gras auf den \underline{Ber}gen
'wachsen lässt.\par
 \einzug{Der dem \underline{Vieh} sein 'Futter gibt: \grestar{} den jungen
Raben, die \underline{zu} ihm 'rufen.}
 Er hat keine Freude an der \underline{Stär}ke des 'Rosses: \grestar{} und kein
Gefallen an den Schen\underline{keln} des 'Mannes.\par
 \einzug{Der \textsc{Herr} hat Gefallen an denen, \underline{die} IHN 'fürchten:
\grestar{} die auf Seine \underline{Gü}te hoffen.}
 Ehre sei dem \textsc{Vater} \ulin{und} dem '\textsc{Sohne}: \grestar{} und dem
\textsc{Hei\underline{li}gen 'Geiste.}\par
 \einzug{Wie im Anfang, so auch \underline{jetzt} und 'allezeit: \grestar{} und
in E\underline{wig}keit. 'Amen.}
  \textit{$\rightarrow$ Antiphon}
\end{multicols} \par
\ueberinitiale{4. \Abar}{Irr.}
\includegabcscore{laudes-4ant-ant.gabc}
\textit{Die Antiphon wird im Vers nicht wiederholt.}\par
\commentary{\textbf{Psalm 147ij}}
\includegabcscore{laudes-4ant-ps.gabc}
\begin{multicols}{2}\setlength{\columnseprule}{0.2pt}
Denn er macht die Riegel Deiner 'Tore fest: \grestar{} und segnet deine Kinder
\underline{in} der 'Mitte.\par
 \einzug{Er schafft deinen Grenzen 'Frieden: \grestar{} und sättiget dich mit
dem \underline{bes}ten 'Weizen.}
 Er sendet Sein Gebot auf die 'Erde: \grestar{} \underline{Sein} Wort 'läuft
schnell.\par
 \einzug{Er gibt Schnee wie 'Wolle: \grestar{} Er streuet \underline{Reif} wie
'Asche.}
 Er wirft seine Schloßen herab wie 'Brocken: \grestar{} wer kann bleiben
\underline{vor} 'Seinem Frost?\par
 \einzug{Er sendet Sein Wort, da 'schmilzt der Schnee: \grestar{} Er lässt
Seinen Wind wehen, \underline{da} 'tauet es.}\vfill\columnbreak
 Sein Wort hat Er Jakob ver'kündet: \grestar{} Israel Seine Gebo\underline{te}
'und Sein Recht.\par
 \einzug{So hat Er an keinem Volk 'getan: \grestar{} sie kennen
Sei\underline{ne} 'Rechte nicht.}
 Ehre sei dem \textsc{Vater} und dem '\textsc{Sohne}: \grestar{} und dem
\textsc{Hei\underline{li}gen 'Geiste}.\par
 \einzug{Wie im Anfang, so auch jetzt und 'allezeit: \grestar{} und in
E\ulin{wig}keit. 'Amen.}
  \textit{$\rightarrow$ Antiphon}
\end{multicols} \vskip0.4em
\section*{Lektion} \subsection*{Conclusio}
\ueberinitiale{sonn-}{tags}
\includegabcscore{conclusio-la-so.gabc}
\ueberinitiale{werk-}{tags}
\includegabcscore{conclusio-la-we.gabc}
\newpage
\section*{Responsorium}
\ueberinitiale{So}{IV}
\includegabcscore{laudes-rs-so.gabc}\par
\ueberinitiale{Mo-Sa}{IV}
\includegabcscore{laudes-rs-we.gabc}\newpage
\section*{Hymnus}
\ueberinitiale{So}{VIII}
\includegabcscore{laudes-h-so.gabc}\par
\begin{multicols}{2}\setlength{\columnseprule}{0.2pt}
Du bist des Wandrers nächtlich Licht, / führst auch zum Ende alle Nacht; //
schon tönt des Hahnes Morgenruf / und lockt hervor der Sonne Strahl.\par
 \einzug{Sein Ruf erweckt den Morgenstern, / macht frei die Welt von Finsternis.
// Der bösen Geister Heer entflieht, / verlässt den Weg der Trug und List.}
 Der Seemann schöpfet neuen Mut, / die Meereswogen glätten sich. // Der Kirche
Fels vernahm den Schrei, / bereut nun alle seine Schuld.\par
 \einzug{Erheben wir uns deshalb schnell! / Die Schlummernden erweckt der Hahn,
// er klagt die trägen Schläfer an, / und, die vergessen ihre Pflicht.}
 Beim Hahnenschrei zieht Hoffnung ein, / Genesung strömt dem Kranken zu. // Der
Räuber nimmt die Waffe weg / und neu erglänzt des Glaubens Licht.\par
\einzug{\textsc{Jesu}, uns Wankende sieh an, / durch Deinen Blick verleih uns
Kraft; // so sinke unsre Sündenlast, / in Tränen schwinde alle Schuld.}
 Du Licht, erleuchte unser Herz, / vertreibe allen Geistesschlaf: // Dir sei das
erste Lied geweiht / als Dank, den wir dir schuldig sind.\par
 \einzug{\emph{\Abar} Lob sei dem \textsc{Vater} auf dem Thron / und Seinem
eingebornen \textsc{Sohn}, // dem \textsc{Heilgen Geist} auch allezeit / von nun
an bis in Ewigkeit. Amen.}
\end{multicols}
\ueberinitiale{Mo-Sa}{I}
\includegabcscore{laudes-h-we.gabc}\par
\begin{multicols}{2}\setlength{\columnseprule}{0.2pt}
Beim Aufstehn reich uns Deine Hand, / es stehe nüchtern auf der Geist, // er
bring, entflammt zum \textsc{Gott}eslob, / den Dank, den wir Dir schuldig
sind.\par
 \einzug{Schon leuchtet auf der Morgenstern / und schreitet vor der Sonn einher;
// der Nächte Nebel fallen tief, / in uns erstrahle heilges Licht.}
 Es bleib in unsern Herzen stets / und treib hinweg die Nacht der Welt, // und
bis zum Ende aller Zeit / bewahre es die Seele rein.\par\vfill\columnbreak
 \einzug{In unsern Herzen wurzle ein / der Glaube, der zuerst gepflanzt, // dann
soll die Hoffnung uns erfreun / und größer noch die Liebe sein.}
 \emph{\Abar} Lob sei dem \textsc{Vater} auf dem Thron / und Seinem eingebornen
\textsc{Sohn}, // dem \textsc{Heilgen Geist} auch allezeit / von nun an bis in
Ewigkeit. Amen.\par
\end{multicols} \vskip0.4em
\section*{Versikel}
\ueberinitiale{sonn-}{tags}
\includegabcscore{laudes-ver-so.gabc}
\ueberinitiale{werk-}{tags}
\includegabcscore{laudes-ver-we.gabc}
\ueberinitiale{Micha-}{elis}
\includegabcscore{laudes-ver-mi.gabc}
\ueberinitiale{Himmel-}{fahrt}
\includegabcscore{laudes-ver-hi.gabc}\newpage
\section*{Canticum Zachariae}
\ueberinitiale{\Abar}{VII a}
\commentary{\small\textit{Lk 1, 68-79}}
\includegabcscore{laudes-cz.gabc}
\begin{multicols}{2}\setlength{\columnseprule}{0.2pt}
Und hat uns aufgerichtet ein \underline{Horn} des 'Heiles: \grestar{} in dem
Hause seines \underline{Die}ners 'David.\par
\einzug{Wie er vor\underline{zei}ten ge'redet hat: \grestar{} durch den Mund
seiner \underline{heil}gen Pro'pheten.}
Dass er uns errettete von \underline{un}sern 'Feinden: \grestar{} und von der
Hand aller, \underline{die} uns hassen.\par
\einzug{Und Barmherzigkeit erzeigete \underline{un}sern 'Vätern: \grestar{} und
gedächte Seines \underline{hei}ligen 'Bundes.}
Des Eides, den Er geschworen hat unserm \underline{Va}ter 'Abraham: \grestar{}
\underline{uns} {\tiny$\downarrow$}zu 'geben.\par
\einzug{Dass wir, erlöset aus der Hand \underline{uns}rer 'Feinde: \grestar{}
IHM dienten ohne Furcht \underline{un}ser 'Leben lang.}
In Heiligkeit \underline{und} Ge'rechtigkeit: \grestar{} die \underline{IHM}
ge'fällig ist.\par
\einzug{Und du, Kindlein, wirst ein Prophet des \underline{Höchs}ten 'heißen:
\grestar{} du wirst vor dem Herrn hergehen, dass du Seinen \underline{Weg}
be'reitest.}
Und Erkenntnis des Heiles \underline{ge}best 'Seinem Volk: \grestar{} in
Vergebung \underline{ih}rer 'Sünden.\par
\einzug{Durch die herzliche Barmherzigkeit \underline{un}sers '\textsc{Gott}es:
\grestar{} durch welche uns besucht hat der Aufgang \underline{aus} der 'Höhe.}
Auf dass Er erscheine denen, die da sitzen in Finsternis und
\underline{Schat}ten des 'Todes: \grestar{} und richte unsre Füße auf den
\underline{Weg} des 'Friedens.\par
\einzug{Ehre sei dem \textsc{Vater} \underline{und} dem '\textsc{Sohne}:
\grestar{} und dem \textsc{\underline{Hei}ligen 'Geiste.}%
}
Wie im Anfang so auch \underline{jetzt} und 'allezeit: \grestar{} und in
\underline{E}wigkeit. 'Amen.\par
 \textit{$\rightarrow$ Antiphon}
\end{multicols}
\section*{Orationes}
\ueberinitiale{sonn-}{tags}
\includegabcscore{laudes-o1.gabc}
\ueberinitiale{werk-}{tags}
\includegabcscore{laudes-o2.gabc}
\subsection*{Preces (werktags)}
\includegabcscore{preces.gabc}
{\itshape\small Im Wechsel weiter:}
\begin{multicols}{2}\setlength{\columnseprule}{0.2pt}
\Vbar \textsc{Herr}, erweise uns Deine Gnade.\par
 \Rbar Und schenke uns Dein Heil.\par
 \Vbar \textsc{Herr}, kehre dich doch wieder zu uns,\par
 \Rbar Und sei deinen Knechten gnädig.\par
 \Vbar Deine Güte, \textsc{Herr}, sei über uns.\par
 \Rbar Wie wir auf Dich hoffen.\par
 \vskip0.4em\hrule\vskip0.4em
\Vbar Lasset uns beten für die heilige Kirche Gottes.\par
 \Rbar \textsc{Herr}, tue wohl an Zion nach \underline{Dei}ner Gnade / baue die
Mauern zu Jerusalem.\par
 \Vbar Es möge Friede sein in deinen Mauern,\par
 \Rbar Und Glück in deinen Palästen.\par
 \Vbar Deine Priester lass sich kleiden mit Gerechtigkeit.\par
 \Rbar Und deine Heiligen sich freuen.\par
 \Vbar Lasset uns beten für unsere Hirten und Lehrer.\par
 \Rbar \textsc{Herr}, nimm nicht von ihrem Munde das Wort der Wahrheit.\par
 \Vbar Lass sie auftreten und weiden in deiner Kraft.\par
 \Rbar Und in der Macht Deines Namens, \textsc{Herr}, unser \textsc{Gott}.\par
 \Vbar Lasset uns beten für alle, die im Glauben unterwiesen werden.\par
 \Rbar \textsc{Herr}, lass sie wachsen in der Gnade und Erkenntnis des Herrn
\textsc{Jesus Christus}.\par
 \Vbar Für die Heimführung des Volkes Israel.\par
 \Rbar Nimm weg, \textsc{Herr}, die Decke von \underline{sei}nem Herzen~/ dass
es sich zu Deinem Sohne bekehre.\par
 \Vbar Für die Ausbreitung des Evangeliums unter den Heiden.\par
 \Rbar Sende Arbeiter in \underline{Dei}ne Ernte / dass alle Menschen zur
Erkenntnis der Wahrheit kommen.\par
 \Vbar Lasset uns beten für unser Volk.\par
 \Rbar Hilf Du uns, \textsc{Gott}, unser Helfer, um Deines Namens Ehre
willen.\par
 \Vbar Für alle Regierenden.\par
 \Rbar \textsc{Herr}, gib ihnen Weisheit und Einsicht gerecht \underline{zu}
regieren / dass Dein Wort geehret werde.\par
 \Vbar Für die Fruchtbarkeit der Erde.\par
 \Rbar Suche das Land heim und wässere es und segne sein Gewächs.\par
 \Vbar Für den Frieden der ganzen Welt.\par
 \Rbar \textsc{Herr}, lass Deine Hilfe nahe sein denen, die Dich fürchten.\par
 \Vbar Dass Güte und Treue einander begegnen.\par
 \Rbar Gerechtigkeit und Friede sich küssen.\par
 \Vbar Gedenke, \textsc{Herr}, Deiner Gemeinde.\par
 \Rbar Die Du vor Zeiten erworben hast.\par
  \vskip0.4em\hrule\vskip0.4em
 \Vbar Breite deine Güte über die, die dich kennen.\par
 \Rbar Und Deine Gerechtigkeit über die Frommen.\par
 \Vbar Lasset uns beten für die Elenden und Betrübten.\par
 \Rbar \textsc{Herr}, stehe ihnen bei und tröste sie.\par
 \Vbar Für die Witwen und Waisen.\par
 \Rbar \textsc{Herr}, lass Deine Güte und Treue allewege sie behüten.\par
 \Vbar Für die Kranken.\par
 \Rbar \textsc{Herr}, erquicke sie nach Deiner Gnade.\par
 \Vbar Lasset uns beten für unsere Widersacher und Verfolger.\par
 \Rbar \textsc{Herr}, behalte ihnen \underline{die}se Sünde nicht, / denn sie
wissen nicht, was sie tun.\par
 \Vbar Für die Abtrünnigen und Verirrten.\par
 \Rbar \textsc{Herr}, weise ihnen den Weg und leite sie auf
richti{\tiny$\downarrow$}ger {\tiny$\uparrow$}Bahn.\par
 \Vbar Für die Gefangenen und Angefochtenen.\par
 \Rbar Erlöse sie, \textsc{Gott} Israel, aus aller ihrer Not.\par
 \Vbar Sende ihnen Hilfe vom Heiligtum.\par
 \Rbar Und stärke sie aus Zion.\par
 \Vbar Lasset uns beten für alle unsre Wohltäter.\par
 \Rbar Gewähre, \textsc{Herr}, allen \underline{die} uns Gutes tun / um Deines
Namens willen das ewige Leben.\par
 \Vbar Für alle Reisenden.\par
 \Rbar Erhöre \underline{uns,} \textsc{Gott}, unser Heil / der Du bist
Zuversicht aller auf Erden und fern am Meere.\par
 \Vbar Für die abwesenden Brüder \textit{(und Schwestern)}.\par
 \Rbar Hilf Du, mein \textsc{Gott}, Deinen Knechten, die sich auf dich
verlassen.\par
 \Vbar Für die Sterbenden.\par
 \Rbar In Deine Hände, \textsc{Herr}, befehlen wir ihren Geist.\par
 \Vbar Lehre uns bedenken, dass wir sterben müssen.\par
 \Rbar Damit wir klug werden.\par
 { \itshape \Vbar Für den / die im Glauben entschlafene/n N. N.\par
 \Rbar \textsc{Herr}, gib ihm / ihr die ewige Ruhe / und das ewige Licht leuchte
ihm / ihr.}\par
 \vskip0.4em\hrule\vskip0.4em
 \Vbar Hilf, \textsc{Herr}, deinem Volke und segne Dein Erbe.\par
 \Rbar Weide die Deinen und trage sie ewiglich.\par
 \Vbar \textsc{Herr} \textsc{Gott} Zebaoth, tröste uns.\par
 \Rbar Lass leuchten dein Antlitz, so genesen wir.\par
 \Vbar Mache Dich auf, CHRISTUS, und hilf uns.\par
 \Rbar Erlöse uns um Deiner Güte willen.\par
 \Vbar \textsc{Herr}, höre mein Gebet.\par
 \Rbar Und lass mein Schreien zu Dir kommen.\par
\end{multicols} 
\subsection*{Collecte und Salutatio\label{Salutatio}}
\includegabcscore{laudes-coll.gabc}
\includegabcscore{laudes-colsa.gabc}
\begin{multicols}{2}\setlength{\columnseprule}{0.2pt} \textbf{Sonntag}\par
 \textit{Gebet des Sonntags nach dem Kirchenjahr.} \ding{118} \ding{166} \ding{167}\par
 \textbf{Montag}\par
 \textsc{Herr}, auf Deiner himmlischen Gnade steht allein unsre Hoffnung / darum bitten wir Dich, erhöre freundlich das Flehen Deines Volkes und bewahre uns mit himmlischen Schutze. \ding{118}\par
 \textbf{Dienstag}\par
 \textsc{Herr}, \textsc{Gott}, wir bitten Dich, bewahre die Herzen Deiner Gläubigen und stärke sie mit der Kraft Deiner Gnade / damit sie beständig vor Dir beten und einander wahrhaftig lieben. \ding{118}\par
 \textbf{Mittwoch}\par
 \textsc{Herr}, höre gnädig unser Flehen und hilf Du selbst unsrer Schwachheit auf / vergib uns unsre Schuld, damit wir uns Deiner Barmherzigkeit unser Leben lang freuen. \ding{118}\par
 \textbf{Donnerstag}\par
 Wir bitten Dich, \textsc{Herr}, erhöre das Flehen Deiner Kirche und schenke ihr Vergebung der Sünden / damit sie fromm werde durch Dein Wirken und unter Deinem Schutze sicher sei. \ding{118}\par
 \textbf{Freitag}\par
 \textsc{Herr}, stehe denen bei, die zu Dir beten und schütze gnädig, die allein auf Deine Barmherzigkeit hoffen / damit sie, von Sünden gereinigt, ein heiliges Leben führen. \ding{118}\par
\textbf{Samstag}\par
 \textsc{Herr} \textsc{Gott}, wir bitten Dich, Deine Rechte schütze das Volk, das zu Dir betet / damit es dieses Leben im Gehorsam führe und so das ewige Leben erlange. \ding{118}\par
\end{multicols} \subsection*{Conclusio}
\includegabcscore{laudes-o3.gabc}
\vskip0.4em
\ding{118} \Vbar \textit{an \textsmcpit{Gott Vater}:}\\
 Durch unsern Herrn \textsc{Jesus Christus}, Deinen Sohn: der mit Dir in der Einheit des \textsc{Heiligen Geistes} ein wahrer \textsc{Gott} / lebet und regieret von Ewigkeit zu Ewigkeit. \Rbar Amen.\par
\ding{166} \Vbar \textit{an \textsmcpit{Gott Sohn}:}\\ der Du mit dem \textsc{Vater} in der Einheit des \textsc{Heiligen Geistes} ein wahrer \textsc{Gott} / lebest und regierest von Ewigkeit zu Ewigkeit. \Rbar Amen.\par
\ding{167} \Vbar \textit{\textsmcpit{Jesus} wird genannt:}\\ Durch IHN, unsern Herrn \textsc{Jesus Christus}, Deinen Sohn: der mit Dir in der Einheit des \textsc{Heiligen Geistes} ein wahrer \textsc{Gott} / lebet und regieret von Ewigkeit zu Ewigkeit. \Rbar Amen.\par\newpage
\section*{Benedicamus}
\includegabcscore{laudes-coll.gabc}
\ueberinitiale{sonn-}{tags}
\includegabcscore{laudes-b1.gabc}
\ueberinitiale{werk-}{tags}
\includegabcscore{laudes-b2.gabc}
\vskip1em \Vbar Der \textsc{Herr} gebe uns Seinen Frieden. \Rbar Und das ewige Leben. Amen.\par
\begin{center}
{\centering{\scalebox{3}{\grecross}}}
\end{center}

\chapter[Sext]{\textsc{Sext}}
\def\gebet{\textsc{Sext}}
\section*{Ingressus}
\ueberinitiale{sonn-}{tags}
\includegabcscore{ingressus-so-seco.gabc}\par
\ueberinitiale{werk-}{tags}
\includegabcscore{ingressus-wo-seco.gabc}
\newpage
\section*{Hymnus}
\ueberinitiale{sonn-}{tags}
\includegabcscore{sext-h-so.gabc}
\begin{multicols}{2}\setlength{\columnseprule}{0.2pt}
Mund, Zunge, Sinn, Gefühl und Kraft / sei Zeuge Deiner Eigenschaft: // die Lieb entzünde jed Geblüt / und teil sich unsern Herzen mit.\par
 \columnbreak\emph{\Abar} Das schenk uns, \textsc{Vater} freudenreich / und ewger \textsc{Sohn}, dem \textsc{Vater} gleich, // zusamt dem \textsc{Geist}, dem Paraklet, / ob aller Zeit und Welt erhöht. Amen.\par
\end{multicols}
\ueberinitiale{werk-}{tags}
\includegabcscore{sext-h-wo.gabc}
\begin{multicols}{2}\setlength{\columnseprule}{0.2pt} \emph{II} Gib Deinen Glanz zur Abendzeit / aus dem das Leben niemals scheid; // ein heilig Sterben hab die Kron / der ewgen Herrlichkeit zum Lohn.\par
 \columnbreak\emph{\Abar} Das schenk uns, \textsc{Vater} freudenreich / und ewger \textsc{Sohn}, dem \textsc{Vater} gleich, // zusamt dem \textsc{Geist}, dem Paraklet, / ob aller Zeit und Welt erhöht. Amen.\par
\end{multicols}\newpage
\ueberinitiale{\Abar}{II D}
\commentary{\textbf{Psalm 86}}
\includegabcscore{sext-1ant-ant.gabc}
\textit{Die Antiphon wird im Vers nicht wiederholt.}\par
\includegabcscore{sext-1ant-ps.gabc}
\begin{multicols}{2}\setlength{\columnseprule}{0.2pt}
Bewahre meine Seele, denn \emph{'ich} bin Dein: \grestar{} hilf Du, mein \textsc{Gott}, Deinem Knechte, der sich \underline{auf} \emph{'Dich ver}lässt.\par
 \einzug{\textsc{Herr}, sei mir \emph{'gnä}dig: \grestar{} denn zu Dir rufe \underline{ich} \emph{'täg}lich.}
 Erfreue die Seele Deines \emph{'Knech}tes: \grestar{} denn nach Dir, \textsc{Herr}, \underline{ver}lan\emph{'get} mich.\par
 \einzug{Denn Du, \textsc{Herr}, bist gut und \emph{'gnä}dig: \grestar{} von großer Güte allen, die Dich \underline{an}\emph{'ru}fen.}
 Vernimm, \textsc{Herr}, \emph{'mein} Gebet: \grestar{} und merke auf die Stimme mei\underline{nes} \emph{'Fle}hens.\par
 \einzug{In der Not rufe \emph{'ich} Dich an: \grestar{} Du wollest mich \underline{er}\emph{'hö}ren.}
 \textsc{Herr}, es ist Dir keiner gleich unter den \emph{'Göt}tern: \grestar{} und was Du tust, \underline{kann} \emph{'nie}mand tun.\par
  \einzug{Alle Völker, die Du gemacht hast, werden kommen und vor Dir an\emph{'be}ten, \textsc{Herr}: \grestar{} und Deinen Na\underline{men} \emph{'eh}ren.}\par
Dass Du so groß bist und \emph{'Wun}der tust: \grestar{} und Du allei\underline{ne} \emph{'\textsc{Gott}} bist.\par 
 \einzug{Weise mir, \textsc{Herr}, \emph{'Dei}nen Weg: \grestar{} dass ich wandle in Dei\underline{ner} \emph{'Wahr}heit.}
 Erhalte mein Herz bei dem \emph{'Ei}nen: \grestar{} dass ich Deinen Na\underline{men} \emph{'fürch}te.\par
  \einzug{Ich danke Dir, \textsc{Herr}, mein \textsc{Gott}, von ganzem \emph{'Her}zen: \grestar{} und ehre Deinen Na\underline{men} 'e\emph{wig}lich.}
Denn groß gegen mich ist Deine \emph{'Gü}te: \grestar{} Du hast mich errettet aus der Tiefe \underline{des} \emph{'To}des.\par
 \einzug{\textsc{Gott}, es erheben sich die Stolzen ge\underline{gen mich} \gredagger{} und eine Rotte von Gewalttätern trachtet mir nach dem \emph{'Le}ben: \grestar{} und haben Dich nicht \underline{vor} \emph{'Au}gen.}
 Du aber, \textsc{Herr} \textsc{Gott}, bist barmherzig und \emph{'gnä}dig: \grestar{} geduldig und von großer Güte \underline{und} \emph{'Treu}e.\par
 \einzug{Wende Dich zu mir und sei mir gnä\underline{dig} \gredagger{} stärke Deinen Knecht mit \emph{'Dei}ner Kraft: \grestar{} und hilf dem Soh\underline{ne} 'Dei\emph{ner} Magd.}\par
 Tu ein Zeichen an mir, dass Du es gut mit mir mei\underline{nest} \gredagger{} dass es sehen, die mich \emph{'has}sen: \grestar{} und sich schämen, weil Du mir beistehest, \textsc{Herr}, und \underline{mich} \emph{'trös}test.\par
 \einzug{Ehre sei dem \textsc{Vater} und dem \textsmcpit{'Soh}\textsc{ne:} \grestar{} und dem \textsc{Heili\underline{gen}} \textsmcpit{'Geis}\textsc{te}.}\par
 Wie im Anfang, so auch jetzt und \emph{'al}lezeit: \grestar{} und in Ewig\underline{keit} \emph{'A}men.\par
  \textit{$\rightarrow$ Antiphon}
\end{multicols}
\section*{Lektion} \subsection*{Conclusio}
\ueberinitiale{sonn-}{tags}
\includegabcscore{conclusio-la-we.gabc}
\ueberinitiale{werk-}{tags}
\includegabcscore{conclusio-co-we.gabc}
\section*{Responsorium breve}
\ueberinitiale{sonn-}{tags}
\includegabcscore{sext-rs-so.gabc}
\ueberinitiale{werk-}{tags}
\includegabcscore{sext-rs-we.gabc}
\section*{Versikel}
\ueberinitiale{sonn-}{tags}
\includegabcscore{sext-ver-so.gabc}
\ueberinitiale{werk-}{tags}
\includegabcscore{sext-ver-we.gabc}
\ueberinitiale{Micha-}{elis}
\includegabcscore{laudes-ver-mi.gabc}
\ueberinitiale{Himmel-}{fahrt}
\includegabcscore{laudes-ver-hi.gabc}

\subsection*{Collecte}
\includegabcscore{collecte-seco.gabc}

\begin{multicols}{2}\setlength{\columnseprule}{0.2pt}
\textbf{Sonntag}\par
 \textsc{Herr} \textsc{Gott}, himmlischer Vater: wir sagen Dir Lob und Dank für alle Deine Gaben / und bitten Dich, erhalte uns allezeit gnädig im wahren Glauben. \ding{118}\par
 \textbf{Montag}\par
 Allmächtiger \textsc{Herr} \textsc{Gott}, stärke unsern Glauben: gib uns auch Liebe und Hoffnung / damit wir Dir und unserm Nächsten nach Deinem Willen dienen. \ding{118}\par
 \textbf{Dienstag}\par
 Allmächtiger, barmherziger \textsc{Gott}, wir bitten Dich: lass uns auf Dein Wort im rechten Glauben hören / damit wir mit Leib und Seele Dein Eigentum werden und ewig bei Dir bleiben. \ding{118}\par
 \textbf{Mittwoch}\par
 Allmächtiger, barmherziger \textsc{Gott}, Du erleuchtest die Herzen durch rechten Glauben: sei bei uns und öffne unsre Augen / damit wir in unserm ganzen Leben Deine Gegenwart erfahren. \ding{118}\par
 \textbf{Donnerstag}\par
 Allmächtiger \textsc{Herr} \textsc{Gott}, Du speisest die Hungrigen: wir bitten Dich, mach uns hungrig nach Deinem Heil / und speise uns mit dem Brot des ewigen Lebens. \ding{118}\par
 \textbf{Freitag}\par
 Allmächtiger \textsc{Gott}, Du hast dich uns in Deinem eingeborenen Sohn selbst gegeben: wir bitten Dich von Herzen, erleuchte uns durch Ihn / damit wir Dich erkennen und im rechten Glauben ewig loben. \ding{167}\par
 \textbf{Samstag}\par
 Allmächtiger \textsc{Gott}, wir bitten Dich herzlich, stärke unsern Glauben: damit wir im Gehorsam wandeln / und das Ziel des Glaubens erreichen -- unsere Seligkeit. \ding{118}\par
\end{multicols}
\subsection*{Conclusio}\par
\includegabcscore{ocon-seco.gabc}
\vskip0.4em
\ding{118} \Vbar \textit{an \textsmcpit{Gott} \textsmcpit{Vater}:}\\ Durch unsern Herrn \textsc{Jesus Christus}, Deinen Sohn: der mit Dir in der Einheit des \textsc{Heiligen Geistes} ein wahrer \textsc{Gott} / lebet und regieret von Ewigkeit zu Ewigkeit. \Rbar Amen.\par
\ding{166} \Vbar \textit{an \textsmcpit{Gott} \textsmcpit{Sohn}:}\\ der Du mit dem \textsc{Vater} in der Einheit des \textsc{Heiligen Geistes} ein wahrer \textsc{Gott} / lebest und regierest von Ewigkeit zu Ewigkeit. \Rbar Amen.\par
\ding{167} \Vbar \textit{\textsmcpit{Jesus} wird genannt:}\\ Durch IHN, unsern Herrn \textsc{Jesus Christus}, Deinen Sohn: der mit Dir in der Einheit des \textsc{Heiligen Geistes} ein wahrer \textsc{Gott} / lebet und regieret von Ewigkeit zu Ewigkeit. \Rbar Amen.\par
\section*{Suffragien (werktags)}
\ueberinitiale{täg-}{lich}
\commentary{Für den Frieden}
\includegabcscore{sext-suffr-taegl.gabc}
\vskip0.4em \Vbar Es möge Frieden sein in Deinen Mauern. \Rbar Und Glück in Deinen Palästen.\par
 \Vbar Lasset uns beten.\par
 \textsc{Herr} \textsc{Gott}, Du schaffest heiligen Mut, guten Rat und rechte Werke: gib Deinen Dienern den Frieden, den die Welt nicht geben kann / damit unsre Herzen an Deinen Geboten hängen und wir unsere Zeit unter Deinem Schutz still und sicher vor Feinden leben. Durch \textsc{Christum}, unsern Herrn. \Rbar Amen.\par\newpage
\ueberinitiale{mon-}{tags}
\commentary{Für die Kirche}
\includegabcscore{sext-suffr-mo.gabc}
\vskip0.4em \Vbar \textsc{Herr}, tue wohl an Zion nach Deiner Gnade. \Rbar Und baue die Mauern zu Jerusalem.\par
 \Vbar Lasset uns beten.\par
 Allmächtiger, ewiger \textsc{Gott}, Du heiligest und regierest mit Deinem Geiste den Leib der Kirche: erhöre unsere
 Bitte für alle Glieder Deiner Christenheit / und schenke, dass sie durch den Beistand Deiner Gnade Dir mit
 wahrem Glauben in Treue dienen. Durch \textsc{Christum}, unsern Herrn. \Rbar Amen.\par
\ueberinitiale{diens-}{tags}
\commentary{Für die Hirten und Lehrer}
\includegabcscore{sext-suffr-di.gabc}
\vskip0.4em \Vbar Erfreue, \textsc{Herr}, die Seele Deiner Knechte. \Rbar Und stärke sie mit Deiner Kraft.\par
 \Vbar Lasset uns beten.\par
 \textsc{Herr} \textsc{Gott}, sieh gnädig auf Deine Diener, die Du in der Kirche zu Hirten berufen hast: lass sie die ihnen anvertraute Herde gewissenhaft führen / damit sie mit ihr zum ewigen Leben gelangen. Durch \textsc{Christum}, unsern Herrn. \Rbar Amen.\par\newpage
\ueberinitiale{mitt-}{wochs}
\commentary{Für die Regierenden}
\includegabcscore{sext-suffr-mi.gabc}
\vskip0.4em \Vbar \textsc{Herr}, höre uns und sei uns gnädig. \Rbar \textsc{Herr}, sei unser Helfer.\par
 \Vbar Lasset uns beten.\par
 Allmächtiger, ewiger \textsc{Gott}, in dessen Hand alle Gewalt und das Recht aller Völker liegt: siehe gnädig auf alle Regierenden, dass sie den Gehorsam gegen Deinen Willen fördern / und wir in Frieden leben und Dir dienen können. Durch \textsc{Christum} unsern Herrn. \Rbar Amen.\par
\par\vskip1em
\ueberinitiale{donners-}{~tags}
\commentary{Für unsere Feinde}
\includegabcscore{sext-suffr-do.gabc}
\vskip0.4em \Vbar \textsc{Herr}, lenke uns allen unser Herz. \Rbar Wir trauen auf Deinen heiligen Namen.\par
 \Vbar Lasset uns beten.\par
 \textsc{Herr} \textsc{Gott}, dem die Liebe und der Frieden wohlgefällt: gib allen unsern Feinden wahre Liebe zum Frieden / vergib ihnen alles, womit sie uns beleidigen, und schütze uns kräftig vor ihrer Macht und List. Durch \textsc{Christum}, unsern Herrn. \Rbar Amen.\par
\ueberinitiale{frei-}{tags}
\commentary{Für die Gefangenen}
\includegabcscore{sext-suffr-fr.gabc}
\vskip0.4em \Vbar \textsc{Herr}, sende ihnen Hilfe vom Heiligtum. \Rbar Und stärke sie aus Zion.\par
 \Vbar Lasset uns beten.\par
 \textsc{Herr} \textsc{Gott}, der du dem Apostel Petrus aus dem Gefängnis geholfen hast: erbarme Dich Deiner gefangenen Diener und löse ihre Fesseln auf / damit wir uns ihrer Befreiung freuen und Dich allezeit loben. Durch \textsc{Christum}, unsern Herrn. \Rbar Amen.\newpage
\ueberinitiale{sams-}{tags}
\commentary{Um Wort und Glauben}
\includegabcscore{sext-suffr-sa.gabc}
\vskip0.4em \Vbar \textsc{Herr}, unsere Augen sehnen sich nach Deinem Heil. \Rbar Und nach dem Wort Deiner Gerechtigkeit.\par
 \Vbar Lasset uns beten.\par
 \textsc{Herr} \textsc{Gott}, Du hast aus lauter Gnade uns Dein Wort gegeben: wir bitten Dich, wehre dem bösen Feinde, dass er uns nicht überwältige und von Deinem Worte abwende / sondern stärke und erhalte uns fest in Deinem Wort und Glauben bis an unser Ende. Durch \textsc{Christum}, unsern Herrn. \Rbar Amen.
\section*{Benedicamus}
\Vbar Der \textsc{Herr} sei mit euch. \Rbar Und mit deinem Geiste.\par
\ueberinitiale{sonn-}{tags}
\includegabcscore{benedicamus-seco-so.gabc}
\ueberinitiale{werk-}{tags}
\includegabcscore{benedicamus-seco-we.gabc}
\vskip1em
\section*{Benedictio}
\Vbar Der \textsc{Herr} gebe uns Seinen Frieden. \Rbar Und das ewige Leben. Amen.
\begin{center}
{\centering{\scalebox{3}{\grecross}}}
\end{center}

%*********** Vesper ********************************
  \chapter[Vesper]{\textsc{Vesper}}
\def\gebet{\textsc{Vesper}}
\section*{Ingressus}\par
\ueberinitiale{sonn-}{tags}
\includegabcscore{ingressus-so-lave.gabc}\par
\ueberinitiale{werk-}{tags}
\includegabcscore{ingressus-wo-lave.gabc}\newpage
\section*{Psalmodie} \textit{In der Vesper hängt die Psalmodie von einem Leseplan ab. In diesem Buch gibt es eine kleine Auswahl an Antiphonen.}\par
\begin{framed}
\centering{\textsc{Psalmen zur 1. Antiphon}}
\end{framed}
\ueberinitiale{1. \Abar}{VII d}
\commentary{\textbf{Psalm 120}}
\includegabcscore{vesper-1ant-120.gabc}
\begin{multicols}{2}\setlength{\columnseprule}{0.2pt}
\textsc{Herr}, errette mich von den \underline{Lü}gen'mäulern: \grestar{} von
den \underline{fal}schen 'Zungen.\par
 \einzug{Was soll er dir antun, du falsche Zun\emph{ge} \gredagger{} und was
\underline{dir} noch 'geben: \grestar{} scharfe Pfeile eines Starken und
\ulin{feu}rige Kohlen.}
 Wehe mir, dass ich weilen muss \underline{un}ter 'Meschech: \grestar{} ich muss
bei Kedars \underline{Zel}ten 'wohnen.\par
 \einzug{Es wird meiner Seele lang, zu wohnen bei denen, die den Frieden
has\ulin{sen} \gredagger{} ich \underline{hal}te 'Frieden: \grestar{} aber wenn
ich rede, so fangen \underline{sie} mit 'Streiten an.}
 Ehre sei dem \textsc{Vater} \underline{und} dem '\textsc{Sohne}: \grestar{} und
dem \textsc{\underline{Hei}ligen 'Geiste}.\par
 \einzug{Wie im Anfang, so auch \underline{jetzt} und 'allezeit: \grestar{} und
in \underline{E}wigkeit. 'Amen.}
 \textit{$\rightarrow$ Antiphon}
\end{multicols}
\ueberinitiale{1. \Abar}{I g5}
\commentary{\textbf{Psalm 128}}
\includegabcscore{vesper-1ant-128.gabc}
\begin{multicols}{2}\setlength{\columnseprule}{0.2pt}
Du wirst dich nähren von deiner 'Hände Arbeit: \grestar{} wohl dir, 'du hast es
gut.\par
 \einzug{Dein Weib wird sein wie ein 'fruchtbarer Weinstock: \grestar{} um 'dein
Haus 'herum.}
 Deine Kinder wie 'Zweige des Ölbaums: \grestar{} um 'deinen 'Tisch her.\par
 \einzug{Siehe, also 'wird gesegnet: \grestar{} der Mann, der den 'Herren
'fürchtet.}
 Der \textsc{Herr} wird dich 'segnen aus Zion: \grestar{} dass du sehest das
Glück Jerusalems 'dein Le'ben lang.\par
 \einzug{Und sehest deiner 'Kinder Kinder: \grestar{} Friede 'über 'Israel.}
 Ehre sei dem \textsc{Vater} 'und dem \textsc{Sohne}: \grestar{} und dem
\textsc{Hei'ligen 'Geiste}.\par
 \einzug{Wie im Anfang, so auch 'jetzt und allezeit: \grestar{} und in
E'wigkeit. 'Amen.}
 \textit{$\rightarrow$ Antiphon}
\end{multicols}\newpage
\ueberinitiale{1. \Abar}{IV a}
\commentary{\textbf{Psalm 110}}
\includegabcscore{vesper-1ant-reg.gabc}
\ueberinitiale{1. \Abar}{IV a}
\commentary{Michaelis}
\includegabcscore{vesper-1ant-mich.gabc}
\vskip1em
\begin{multicols}{2}\setlength{\columnseprule}{0.2pt}
Der \textsc{Herr} wird das Zepter deiner Macht ausstrec'ken aus Zion: \grestar{} herrsche mitten un'ter deinen Feinden.\par
 \einzug{Wenn du dein Heer aufbietest, wird dir dein Volk willig folgen in hei'ligem Schmucke: \grestar{} deine Söhne werden dir geboren wie der Tau aus 'der Morgenröte.}
 Der \textsc{Herr} hat geschworen, und es wird ihn 'nicht gereuen: \grestar{} \enquote{Du bist ein Priester ewiglich nach der Wei'se Melchisedechs.}\par
\einzug{Der \textsc{Herr} zu deiner Rechten 'wird zerschmettern: \grestar{} die Könige am Ta'ge seines Zornes.}
 Er wird richten unter den Hei\textit{den} \gredagger{} wird vie'le erschlagen: \grestar{} wird Häupter zerschmettern auf 'weitem Gefilde.\par
 \einzug{Er wird trinken vom Bache 'auf dem Wege: \grestar{} darum wird er 'das 'Haupt emporheben.}
 Ehre sei dem \textsc{Vater} 'und dem '\textsc{Sohne}: \grestar{} und dem '\textsc{Heiligen Geiste}.\par
 \einzug{Wie im Anfang, so auch 'jetzt und allezeit: \grestar{} und in 'Ewigkeit. Amen.}
  \textit{$\rightarrow$ Antiphon}
\end{multicols}\newpage
\begin{framed}
\centering{\textsc{Psalmen zur 2. Antiphon}}
\end{framed}
\ueberinitiale{2. \Abar}{VI F}
\commentary{\textbf{Psalm 123}}
\includegabcscore{vesper-2ant-123.gabc}
\vskip0.4em
\begin{multicols}{2}\setlength{\columnseprule}{0.2pt}
Siehe, wie die Augen der Knechte auf die Hände ihrer Herren se\underline{hen}
\gredagger{} wie die Augen der Magd auf die 'Hände 'ihrer Frau: \grestar{} so
sehen unsere Augen auf den Herren, unsern \textsc{Gott}, bis Er uns 'gnädig
'werde.\par
 \einzug{Sei uns gnädig, \textsc{Herr}, 'sei uns 'gnädig: \grestar{} denn allzu
sehr litten 'wir Ver'achtung.}
 Allzusehr litt unsere Seele den 'Spott der 'Stolzen: \grestar{} und die
Verachtung 'der Hof'färtigen.\par
 \einzug{Ehre sei dem \textsc{Vater} 'und dem '\textsc{Sohne}: \grestar{} und
dem \textsc{Hei'ligen 'Geiste}.}
 Wie im Anfang, so auch 'jetzt und 'allezeit: \grestar{} und in E'wigkeit.
'Amen.\par
 \textit{$\rightarrow$ Antiphon}
\end{multicols}
\ueberinitiale{2. \Abar}{VII a}
\commentary{\textbf{Psalm 111}}
\includegabcscore{vesper-2ant-111-reg.gabc}
\ueberinitiale{2. \Abar}{VII a}
\commentary{Michaelis}
\includegabcscore{vesper-2ant-111-mich.gabc}
\begin{multicols}{2}\setlength{\columnseprule}{0.2pt}
Groß sind die 'Werke des 'Herren: \grestar{} wer sie erforscht, 'der hat 'Freude
dran.\par
 \einzug{Was er tut, das ist 'herrlich und 'prächtig: \grestar{} und seine
Gerechtigkeit 'bleibet 'ewiglich.}
 Er hat ein Gedächtnis gestiftet 'seiner 'Wunder: \grestar{} Der \textsc{Herr}
ist barm'herzig und 'gnädig.\par
 \einzug{Er gibet Speise denen. 'die ihn 'fürchten: \grestar{} er gedenket 'ewig
an 'seinen Bund.}
 Er lässt verkündigen seine gewaltigen 'Taten 'seinem Volk: \grestar{} dass er
ihnen gebe das 'Erbe der 'Heiden.\par
 \einzug{Die Werke seiner Hände sind 'Wahrheit 'und Recht: \grestar{} alle seine
Ordnungen 'sind be'ständig.}
 Sie stehen fest für 'immer und 'ewig: \grestar{} sie sind 'recht und
ver'lässlich.\par
 \einzug{Er sendet eine Erlösung sei\underline{nem Volk,} \gredagger{} er
verheißet, dass sein Bund 'ewig 'bleiben soll: \grestar{} heilig und hehr 'ist
sein 'Name.}
 Die Furcht des Herren ist der Weisheit An\underline{fang} \gredagger{} klug
sind 'alle, die 'danach tun: \grestar{} sein Lob 'bleibet 'ewiglich.\par
 \einzug{Ehre sei dem \textsc{Vater} 'und dem '\textsc{Sohne}: \grestar{} und
dem \textsc{'Heiligen 'Geiste}.}
 Wie im Anfang, so auch 'jetzt und 'allezeit: \grestar{} und in 'Ewigkeit
'Amen.\par
 \textit{$\rightarrow$ Antiphon}
\end{multicols}
\begin{framed}
\centering{\textsc{Psalmen zur 3. Antiphon}}
\end{framed}
\ueberinitiale{3. \Abar}{V a}
\commentary{\textbf{Psalm 138}}
\includegabcscore{vesper-3ant-138.gabc}
\vskip0.4em
\begin{multicols}{2}\setlength{\columnseprule}{0.2pt}
Ich will anbeten vor Deinem heiligen 'Tempel: \grestar{} und Deinen Namen
preisen für Deine 'Güte und Treue.\par
 \einzug{Denn Du hast Deinen 'Namen: \grestar{} und Dein Wort herrlich gemacht
'über alles.}
 Wenn ich Dich anrufe, so er'hörst Du mich: \grestar{} und gibst meiner 'Seele
große Kraft.\par
 \einzug{\textsc{Herr}, es danken Dir auf Erden alle 'Könige: \grestar{} dass
sie hören das Wort 'Deines Mundes.}
 Sie singen von den Wegen des 'Herren: \grestar{} dass die Herrlichkeit des
Herren 'so gewaltig ist.\par
 \einzug{Denn der \textsc{Herr} ist hoch und siehet auf die 'Niedrigen:
\grestar{} und kennet den 'Stolzen von Ferne.}
 Wenn ich mitten in der Angst wandle, so erquickst \underline{du mich}
\gredagger{} und reckest Deine Hand gegen den Zorn meiner 'Feinde: \grestar{}
und hilfst mir mit 'Deiner Rechten.\par
 \einzug{Der \textsc{Herr} wird meine Sache hinausfüh\underline{ren}
\gredagger{} \textsc{Herr}, Deine Güte ist 'ewig: \grestar{} das Werk Deiner
Hände wollest 'Du nicht lassen.}
 Ehre sei dem \textsc{Vater} und dem '\textsc{Sohne}: \grestar{} und dem
'\textsc{Heiligen Geiste}.\par
 \einzug{Wie im Anfang, so auch jetzt und 'allezeit: \grestar{} und in
'Ewigkeit. Amen.}
 \textit{$\rightarrow$ Antiphon}
\end{multicols}\newpage
\ueberinitiale{3. \Abar}{Per}
\commentary{\textbf{Psalm 112}}
\includegabcscore{vesper-3ant-112-reg.gabc}
\ueberinitiale{3. \Abar}{Per}
\commentary{Michaelis}
\includegabcscore{vesper-3ant-112-mich.gabc}
\begin{multicols}{2}\setlength{\columnseprule}{0.2pt}
Sein Geschlecht wird gewal'tig sein im 'Lande: \grestar{} die Kinder der Frommen
'werden \ulin{ge}'segnet sein.\par
 \einzug{Reichtum und Fülle wird 'in ihrem 'Hause sein: \grestar{} und ihre
Gerechtigkeit 'blei\ulin{bet} 'ewiglich.}
 Den Frommen gehet das Licht 'auf in der 'Finsternis: \grestar{} von dem
Gnädigen, Barmherzigen 'und \ulin{Ge}'rechten.\par
 \einzug{Wohl dem, der barmherzig ist \ulin{und} 'gerne 'leihet: \grestar{} und
das Seine tut, 'wie \ulin{es} 'recht ist.}
 Denn er wird 'ewiglich 'bleiben: \grestar{} der Gerechte wird nimmer'mehr
\ulin{ver}'gessen.\par
 \einzug{Vor schlimmer Kund\ulin{e} 'fürchtet 'er sich nicht: \grestar{} bis er
auf seine 'Feinde \ulin{her}'absieht.}
 Er streuet aus und gibt den Ar\underline{men} \gredagger{} seine
Gerechtig\ulin{keit }'bleibet 'ewiglich: \grestar{} seine Kraft wird hoch in
'Eh\ulin{ren} 'stehen.\par
 \einzug{Der Gottlose wird's sehen, und es wird ihn verdrie\underline{ßen}
\gredagger{} mit den Zähnen wird er knir\ulin{schen} 'und ver'gehen: \grestar{}
denn was die Gottlosen wollen, das 'wird \ulin{zu}'nichte.}
 Ehre sei dem \textsc{Va'\ulin{ter}} und dem '\textsc{Sohne}: \grestar{} und dem
'\textsc{Heili\ulin{gen} 'Geiste}.\par
 \einzug{Wie im Anfang, so \ulin{auch} 'jetzt und 'allezeit: \grestar{} und in
'Ewig\ulin{keit.} 'Amen.}
 \textit{$\rightarrow$ Antiphon}
\end{multicols}
\section*{Lektion}
\subsection*{Conclusio}
\ueberinitiale{sonn-}{tags}
\includegabcscore{conclusio-la-so.gabc}
\ueberinitiale{werk-}{tags}
\includegabcscore{conclusio-la-we.gabc}\newpage
\section*{Responsorium} \textit{Am Sonntag stehen zusätzlich die beiden
Responsoria prolixa aus dem Anhang zur Verfügung.}\par
\ueberinitiale{So}{V}
\includegabcscore{vesper-rs-so1.gabc}
\ueberinitiale{Mo-Sa}{II}
\includegabcscore{vesper-rs-we.gabc}\par\newpage
\section*{Hymnus}
\ueberinitiale{sonn-}{tags}
\includegabcscore{vesper-h-so.gabc}
\vskip0.4em
\begin{multicols}{2}\setlength{\columnseprule}{0.2pt}
Die Zeit vom Morgen bis zur Nacht / hast selber Du einst Tag genannt. // Es
bricht die dunkle Nacht herein. / erhöre unser heißes Flehn.\par
 \einzug{Lass unsre Seele, schuldbeschwert / verlieren nicht des Lebens Heil. //
wenn sie an Ewiges nicht denkt / und sich mit Sündenschuld befleckt.}
 Sie klopfe an des Himmels Tor, / empfange ewgen Lebens Lohn; // lass meiden uns
doch alle Sünd / und reingen uns von jeder Schuld.\par
 \einzug{\emph{\Abar} Gewähre dies, \textsc{Gott} \textsc{Vater} mild / und Du,
\textsc{Gott} Sohn, Sein Ebenbild, // die ihr mit \textsc{Gott}, dem Heilgen
Geist, / seid mächtig über Welt und Zeit. Amen.}
\end{multicols}
\ueberinitiale{werk-}{tags}
\includegabcscore{vesper-h-we.gabc}
\vskip0.4em
\begin{multicols}{2}\setlength{\columnseprule}{0.2pt}
Damit die Erde grün und blüh, / geziert mit bunter Blumenpracht, // auch dass
sie reich an Früchten sei / und gute Nahrung biete dar.\par
\einzug{Bring Heilung unserm wunden Herz / durch Deine starke Gnadenkraft, // in
Tränen löse sich die Schuld / und böse Neigung falle ab.}\vfill\columnbreak
Das Herz gehorche Deinem Wort / und bleibe jeder Sünde fern; // es werde alles
Guten voll / und kenne nie des Todes Stich.\par
\einzug{\emph{\Abar} Gewähre dies, \textsc{Gott} \textsc{Vater} mild / und Du,
\textsc{Gott} Sohn, Sein Ebenbild, // die ihr mit \textsc{Gott}, dem Heilgen
Geist, / seid mächtig über Welt und Zeit. Amen.}
\end{multicols}\newpage
\section*{Versikel}\par
\ueberinitiale{So-}{Fr}
\includegabcscore{vesper-ver-sofr.gabc}
\ueberinitiale{Sa}{}
\includegabcscore{vesper-ver-sa.gabc}
\ueberinitiale{Micha-}{elis}
\includegabcscore{laudes-ver-mi.gabc}
\ueberinitiale{Himmel-}{fahrt}
\includegabcscore{laudes-ver-hi.gabc}
\newpage
\section*{Canticum Mariæ}
\ueberinitiale{\Abar}{IV E}
\commentary{\textit{Lk 1, 46 -- 55}}
\includegabcscore{vesper-cm-reg.gabc}
\vskip0.4em
\begin{multicols}{2}\setlength{\columnseprule}{0.2pt}
 Denn Er hat die Niedrigkeit Seiner Magd 'ange'sehen: \grestar{} siehe, von nun
an werden mich selig preisen al\ulin{le} 'Kindes'kinder.\par
 \einzug{Denn Er hat große Dinge an mir getan, 'der da 'mächtig ist: \grestar{}
und \ulin{des} 'Name 'heilig ist.}
 Und Seine Barmherzigkeit währet 'immer 'für und für: \grestar{} bei
de\ulin{nen,} 'die ihn 'fürchten.\par
 \einzug{Er übet Ge'walt mit 'Seinem Arm: \grestar{} und zerstreuet, die
hoffärtig sind \ulin{in} 'ihres 'Herzens Sinn.}
 Er stößet die Gewalti'gen vom 'Throne: \grestar{} und er'\ulin{he}bet die
'Niedrigen.\par
 \einzug{Die Hungrigen füllet 'Er mit 'Gütern: \grestar{} \ulin{und} 'läßt die
'Reichen leer.}
 Er denket 'der Barm'herzigkeit: \grestar{} und hilft Seinem Die\ulin{ner}
'Isra'el auf.\par
 \einzug{Wie Er geredet hat 'unsern 'Vätern: \grestar{} Abraham und
sei\ulin{nen} 'Kindern 'ewiglich.}
 Ehre sei dem \textsc{Vater} 'und dem '\textsc{Sohne}: \grestar{} und dem
'\textsc{Heiligen} '\textsc{Geiste}.\par
 \einzug{Wie im Anfang, so auch 'jetzt und 'allezeit: \grestar{} und in
\ulin{'E}wigkeit. 'Amen.}
  \textit{$\rightarrow$ Antiphon}
\end{multicols}\par
\ueberinitiale{\Abar}{VII a}
\commentary{Michaelis}
\includegabcscore{vesper-cm-mich.gabc}
\vskip0.4em
\begin{multicols}{2}\setlength{\columnseprule}{0.2pt}
 Denn Er hat die Niedrigkeit Seiner Magd 'ange'sehen: \grestar{} siehe, von nun
an werden mich selig preisen al\ulin{le} 'Kindes'kinder.\par
 \einzug{Denn Er hat große Dinge an mir getan, 'der da 'mächtig ist: \grestar{}
und \ulin{des} 'Name 'heilig ist.}
 Und Seine Barmherzigkeit währet 'immer 'für und für: \grestar{} bei
de\ulin{nen,} 'die ihn 'fürchten.\par
 \einzug{Er übet Ge'walt mit 'Seinem Arm: \grestar{} und zerstreuet, die
hoffärtig sind \ulin{in} 'ihres 'Herzens Sinn.}
 Er stößet die Gewalti'gen vom 'Throne: \grestar{} und er'\ulin{he}bet die
'Niedrigen.\par
 \einzug{Die Hungrigen füllet 'Er mit 'Gütern: \grestar{} \ulin{und} 'läßt die
'Reichen leer.}
 Er denket 'der Barm'herzigkeit: \grestar{} und hilft Seinem Die\ulin{ner}
'Isra'el auf.\par
 \einzug{Wie Er geredet hat 'unsern 'Vätern: \grestar{} Abraham und
sei\ulin{nen} 'Kindern 'ewiglich.}
 Ehre sei dem \textsc{Vater} 'und dem '\textsc{Sohne}: \grestar{} und dem
'\textsc{Heiligen} '\textsc{Geiste}.\par
 \einzug{Wie im Anfang, so auch 'jetzt und 'allezeit: \grestar{} und in
\ulin{'E}wigkeit. 'Amen.}
  \textit{$\rightarrow$ Antiphon}
\end{multicols}\par
\section*{Orationes}
\ueberinitiale{sonn-}{tags}
\includegabcscore{laudes-o1.gabc}
\ueberinitiale{werk-}{tags}
\includegabcscore{laudes-o2.gabc}
\subsection*{Preces (werktags)}
\includegabcscore{preces.gabc}
{\itshape\small Im Wechsel weiter:}
\begin{multicols}{2}\setlength{\columnseprule}{0.2pt}
\Vbar \textsc{Herr}, erweise uns Deine Gnade.\par
 \Rbar Und schenke uns Dein Heil.\par
 \Vbar \textsc{Herr}, kehre dich doch wieder zu uns,\par
 \Rbar Und sei deinen Knechten gnädig.\par
 \Vbar Deine Güte, \textsc{Herr}, sei über uns.\par
 \Rbar Wie wir auf Dich hoffen.\par
 \vskip0.4em\hrule\vskip0.4em
\Vbar Lasset uns beten für die heilige Kirche Gottes.\par
 \Rbar \textsc{Herr}, tue wohl an Zion nach \underline{Dei}ner Gnade / baue die
Mauern zu Jerusalem.\par
 \Vbar Es möge Friede sein in deinen Mauern,\par
 \Rbar Und Glück in deinen Palästen.\par
 \Vbar Deine Priester lass sich kleiden mit Gerechtigkeit.\par
 \Rbar Und deine Heiligen sich freuen.\par
 \Vbar Lasset uns beten für unsere Hirten und Lehrer.\par
 \Rbar \textsc{Herr}, nimm nicht von ihrem Munde das Wort der Wahrheit.\par
 \Vbar Lass sie auftreten und weiden in deiner Kraft.\par
 \Rbar Und in der Macht Deines Namens, \textsc{Herr}, unser \textsc{Gott}.\par
 \Vbar Lasset uns beten für alle, die im Glauben unterwiesen werden.\par
 \Rbar \textsc{Herr}, lass sie wachsen in der Gnade und Erkenntnis des Herrn
\textsc{Jesus Christus}.\par
 \Vbar Für die Heimführung des Volkes Israel.\par
 \Rbar Nimm weg, \textsc{Herr}, die Decke von \underline{sei}nem Herzen~/ dass
es sich zu Deinem Sohne bekehre.\par
 \Vbar Für die Ausbreitung des Evangeliums unter den Heiden.\par
 \Rbar Sende Arbeiter in \underline{Dei}ne Ernte / dass alle Menschen zur
Erkenntnis der Wahrheit kommen.\par
 \Vbar Lasset uns beten für unser Volk.\par
 \Rbar Hilf Du uns, \textsc{Gott}, unser Helfer, um Deines Namens Ehre
willen.\par
 \Vbar Für alle Regierenden.\par
 \Rbar \textsc{Herr}, gib ihnen Weisheit und Einsicht gerecht \underline{zu}
regieren / dass Dein Wort geehret werde.\par
 \Vbar Für die Fruchtbarkeit der Erde.\par
 \Rbar Suche das Land heim und wässere es und segne sein Gewächs.\par
 \Vbar Für den Frieden der ganzen Welt.\par
 \Rbar \textsc{Herr}, lass Deine Hilfe nahe sein denen, die Dich fürchten.\par
 \Vbar Dass Güte und Treue einander begegnen.\par
 \Rbar Gerechtigkeit und Friede sich küssen.\par
 \Vbar Gedenke, \textsc{Herr}, Deiner Gemeinde.\par
 \Rbar Die Du vor Zeiten erworben hast.\par
  \vskip0.4em\hrule\vskip0.4em
 \Vbar Breite deine Güte über die, die dich kennen.\par
 \Rbar Und Deine Gerechtigkeit über die Frommen.\par
 \Vbar Lasset uns beten für die Elenden und Betrübten.\par
 \Rbar \textsc{Herr}, stehe ihnen bei und tröste sie.\par
 \Vbar Für die Witwen und Waisen.\par
 \Rbar \textsc{Herr}, lass Deine Güte und Treue allewege sie behüten.\par
 \Vbar Für die Kranken.\par
 \Rbar \textsc{Herr}, erquicke sie nach Deiner Gnade.\par
 \Vbar Lasset uns beten für unsere Widersacher und Verfolger.\par
 \Rbar \textsc{Herr}, behalte ihnen \underline{die}se Sünde nicht, / denn sie
wissen nicht, was sie tun.\par
 \Vbar Für die Abtrünnigen und Verirrten.\par
 \Rbar \textsc{Herr}, weise ihnen den Weg und leite sie auf
richti{\tiny$\downarrow$}ger {\tiny$\uparrow$}Bahn.\par
 \Vbar Für die Gefangenen und Angefochtenen.\par
 \Rbar Erlöse sie, \textsc{Gott} Israel, aus aller ihrer Not.\par
 \Vbar Sende ihnen Hilfe vom Heiligtum.\par
 \Rbar Und stärke sie aus Zion.\par
 \Vbar Lasset uns beten für alle unsre Wohltäter.\par
 \Rbar Gewähre, \textsc{Herr}, allen \underline{die} uns Gutes tun / um Deines
Namens willen das ewige Leben.\par
 \Vbar Für alle Reisenden.\par
 \Rbar Erhöre \underline{uns,} \textsc{Gott}, unser Heil / der Du bist
Zuversicht aller auf Erden und fern am Meere.\par
 \Vbar Für die abwesenden Brüder \textit{(und Schwestern)}.\par
 \Rbar Hilf Du, mein \textsc{Gott}, Deinen Knechten, die sich auf dich
verlassen.\par
 \Vbar Für die Sterbenden.\par
 \Rbar In Deine Hände, \textsc{Herr}, befehlen wir ihren Geist.\par
 \Vbar Lehre uns bedenken, dass wir sterben müssen.\par
 \Rbar Damit wir klug werden.\par
 { \itshape \Vbar Für den / die im Glauben entschlafene/n N. N.\par
 \Rbar \textsc{Herr}, gib ihm / ihr die ewige Ruhe / und das ewige Licht leuchte
ihm / ihr.}\par
 \vskip0.4em\hrule\vskip0.4em
 \Vbar Hilf, \textsc{Herr}, deinem Volke und segne Dein Erbe.\par
 \Rbar Weide die Deinen und trage sie ewiglich.\par
 \Vbar \textsc{Herr} \textsc{Gott} Zebaoth, tröste uns.\par
 \Rbar Lass leuchten dein Antlitz, so genesen wir.\par
 \Vbar Mache Dich auf, CHRISTUS, und hilf uns.\par
 \Rbar Erlöse uns um Deiner Güte willen.\par
 \Vbar \textsc{Herr}, höre mein Gebet.\par
 \Rbar Und lass mein Schreien zu Dir kommen.\par
\end{multicols} 
\subsection*{Collecte und Salutatio\label{Salutatio}}
\includegabcscore{laudes-coll.gabc}
\includegabcscore{laudes-colsa.gabc}
 \vskip0.4em
\begin{multicols}{2}\setlength{\columnseprule}{0.2pt}
 \textbf{Sonntag}\par
 \textit{Gebet des Sonntags nach dem Kirchenjahr.} \ding{118} \ding{166} \ding{167}\par
 \textbf{Montag}\par
 \textsc{Herr} \textsc{Gott}, wir bitten Dich, wende Dich zu unserm demütigen Flehen / und schenke uns nach Deiner großen Güte Vergebung und Frieden. \ding{118}\par
 \textbf{Dienstag}\par
 Wir bitten Dich, \textsc{Herr}, komme uns mit Deiner Barmherzigkeit zuvor / und schenke uns den Reichtum Deiner Gnade, noch ehe wir bitten. \ding{118}\par
 \textbf{Mittwoch}\par
 \textsc{Herr}, wir bitten Dich, vertreibe aus unseren Herzen alles Böse / damit wir mit Zuversicht den Weg des Heiles laufen. \ding{118}\par
 \textbf{Donnerstag}\par
 Erhöre uns, \textsc{Herr}, unser \textsc{Gott}, und regiere Deine Kirche mit Deiner Gnade / und leite sie so durch die Stürme der Welt. \ding{118}\par
 \textbf{Freitag}\par
 Wir bitten Dich, \textsc{Herr}, zerreiße die Fesseln der Sünde, die uns gefangen halten / damit wir freien Herzens Deinen Namen bekennen und preisen. \ding{118}\par
 \textbf{Samstag}\par
 Erhöre gnädig, \textsc{Herr}, die zu Dir rufen / reiße sie aus dem Abgrund der Sünde und führe sie zu den ewigen Freuden. \ding{118}\par
\end{multicols}\newpage
\subsection*{Conclusio}
\includegabcscore{laudes-o3.gabc}
\vskip0.4em\enlargethispage{2\baselineskip}
\ding{118} \Vbar \textit{an \textsmcpit{Gott} \textsmcpit{Vater}:}\\ Durch unsern Herrn \textsc{Jesus Christus}, Deinen Sohn: der mit Dir in der Einheit des \textsc{Heiligen Geistes} ein wahrer \textsc{Gott} / lebet und regieret von Ewigkeit zu Ewigkeit. \Rbar Amen.\par
\ding{166} \Vbar \textit{an \textsmcpit{Gott} \textsmcpit{Sohn}:}\\ der Du mit dem \textsc{Vater} in der Einheit des \textsc{Heiligen Geistes} ein wahrer \textsc{Gott} / lebest und regierest von Ewigkeit zu Ewigkeit. \Rbar Amen.\par
\ding{167} \Vbar \textit{\textsmcpit{Jesus} wird genannt:}\\ Durch IHN, unsern Herrn \textsc{Jesus Christus}, Deinen Sohn: der mit Dir in der Einheit des \textsc{Heiligen Geistes} ein wahrer \textsc{Gott} / lebet und regieret von Ewigkeit zu Ewigkeit. \Rbar Amen.\par
\par\vskip0.5em
\section*{Benedicamus}
\includegabcscore{laudes-coll.gabc}
\ueberinitiale{sonn-}{tags}
\includegabcscore{vesper-ben1.gabc}
\ueberinitiale{werk-}{tags}
\includegabcscore{vesper-ben2.gabc}
\vskip1em \Vbar Der \textsc{Herr} gebe uns Seinen Frieden. \Rbar Und das ewige Leben. Amen.\par
\begin{center}
{\centering{\scalebox{3}{\grecross}}}
\end{center}
\input{complet-luh}
\chapter{Reisesegen}\def\gebet{\textsc{Reisesegen}}
\includegabcscore{reisesegen-ant.gabc}\par
\begin{multicols}{2}\setlength{\columnseprule}{0.2pt}
Und hat uns aufgerichtet ein \underline{Horn} des 'Heiles: \grestar{} in dem Hause seines \underline{Die}ners 'David.\par
\einzug{Wie er vor\underline{zei}ten ge'redet hat: \grestar{} durch den Mund seiner \underline{heil}gen Pro'pheten.}
Dass er uns errettete von \underline{un}sern 'Feinden: \grestar{} und von der Hand aller, \underline{die} uns hassen.\par
\einzug{Und Barmherzigkeit erzeigete \underline{un}sern 'Vätern: \grestar{} und gedächte Seines \underline{hei}ligen 'Bundes.}
Des Eides, den Er geschworen hat unserm \underline{Va}ter 'Abraham: \grestar{} \underline{uns} {\tiny$\downarrow$}zu 'geben.\par
\einzug{Dass wir, erlöset aus der Hand \underline{uns}rer 'Feinde: \grestar{} IHM dienten ohne Furcht \ulin{un}ser 'Leben lang.}
In Heiligkeit \ulin{und} Ge'rechtigkeit: \grestar{} die \ulin{IHM} ge'fällig ist.\par
\einzug{Und du, Kindlein, wirst ein Prophet des \underline{Höchs}ten 'heißen: \grestar{} du wirst vor dem Herrn hergehen, dass du Seinen \underline{Weg} be'reitest.}
Und Erkenntnis des Heiles \underline{ge}best 'Seinem Volk: \grestar{} in Vergebung \underline{ih}rer 'Sünden.\par
\einzug{Durch die herzliche Barmherzigkeit \underline{un}sers '\textsc{Gott}es: \grestar{} durch welche uns besucht hat der Aufgang \underline{aus} der 'Höhe.}
Auf dass Er erscheine denen, die da sitzen in Finsternis und \underline{Schat}ten des 'Todes: \grestar{} und richte unsre Füße auf den \underline{Weg} des 'Friedens.\par
\einzug{Ehre sei dem \textsc{Vater} \underline{und} dem '\textsc{Sohne}: \grestar{} und dem \textsc{\underline{Hei}ligen 'Geiste}.}
Wie im Anfang so auch \underline{jetzt} und 'allezeit: \grestar{} und in \underline{E}wigkeit. 'Amen.\par
 \textit{Leitvers wird wiederholt.}
\end{multicols}
\newpage
\section*{Orationes}
\ueberinitiale{sonn-}{tags}
\includegabcscore{laudes-o1.gabc}
\ueberinitiale{werk-}{tags}
\includegabcscore{laudes-o2.gabc}

\subsection*{Preces}
\includegabcscore{reisesegen-preces.gabc}
{\itshape\small Im Wechsel weiter:}
\begin{multicols}{2}\setlength{\columnseprule}{0.2pt} \Vbar \textsc{Herr}, sende uns Hilfe vom Heiligtum.\par
 \Rbar Und stärke uns aus Zion.\par
 \Vbar Gelobet sei der \textsc{Herr} täglich.\par
 \Rbar Eine glückliche Reise verleihe uns der \textsc{Gott} unseres Heiles.\par
 \Vbar \textsc{Herr}, zeige uns Deine Wege.\par
 \Rbar Und lehre uns Deine Steige.\par
 \Vbar O, dass unser Leben Deine Rechte\par
 \Rbar Mit ganzem Ernste hielte.\par
 \Vbar Was uneben ist, soll gerade\par
 \Rbar Und was bergig ist, soll eben werden.\par
 \Vbar Der \textsc{Herr} hat seinen Engeln befohlen über dir\par
 \Rbar Dass sie dich behüten auf allen Deinen Wegen.\par
 \Vbar \textsc{Herr}, höre mein Gebet.\par
 \Rbar Und lass mein Schreien zu Dir kommen.\par
\end{multicols}
\subsection*{Collecte und Salutatio}
\includegabcscore{complet-colsa.gabc}
\vskip0.4em \textsc{Herr} \textsc{Gott}, der Du die Kinder Israels hast trockenen Fußes mitten durchs Meer ziehen lassen, und den drei Weisen durch das Geleit des Sternes den Weg zu Dir gewiesen hast: schenke uns eine glückliche Fahrt und ruhige Zeiten / damit wir unter dem Geleite Deiner heiligen Engel behütet ans Ziel unserer Reise und schließlich zum ewigen Heile gelangen.\par
 \textsc{Herr}  \textsc{Gott}, Du hast Deinen Knecht Abraham aus Ur geführet: und hast ihn auch auf allen Wegen seiner Pilgerschaft unversehrt bewahret / behüte uns, Deine Diener, gleicherweise in Gnaden. Sei uns Helfer beim Aufbruch, Trost auf dem Wege, Schatten in der Hitze, Schutz bei Regen und Kälte, Gefährt bei Müdigkeit, Schirm in Gefahren, Halt auf schlüpfrigen Pfaden, Zuflucht bei Unglücksfällen / damit wir unter Deinem Geleite glücklich das Ziel erreichen und unversehrt in die Heimat zurückkehren.\par
 Lieber Herre \textsc{Gott}, erhöre gnädig unsere Bitten: geleite Deine Diener mit Deinem Segen auf ihren Wegen / damit sie bei allen Wechselfällen der Reise und dieses Lebens niemals ohne Deine Hilfe sind.\par
 Allmächtiger \textsc{Gott}, merke auf unsere Bitten: und verleihe, dass Deine Gemeinde auf dem Wege des Heiles laufe / und, dem Rufe des heiligen Vorläufers Johannes folgend, ungefährdet zu dem gelange, den er verkündet hat. \newpage
\subsection*{Conclusio}
\includegabcscore{coll-con-co.gabc}
\vskip0.3em
\Vbar Durch unsern Herrn \textsc{Jesus Christus}, Deinen Sohn: der mit Dir in der Einheit des \textsc{Heiligen Geistes} ein wahrer \textsc{Gott} / lebet und regieret von Ewigkeit zu Ewigkeit. \Rbar Amen.\par
\section*{Benedictio} \Vbar Lasset uns in Frieden ziehen. \Rbar Im Namen des Herren. Amen.\par
\begin{center}
{\centering{\scalebox{3}{\grecross}}}
\end{center}

 \chapter{Responsoria Prolixa aus der Vesper}\def\gebet{\textsc{Responsoria Prolixa}}
\ueberinitiale{\Rbar}{VII}
\commentary{\textit{Psalm 26, 6-8}}
\includegabcscore{rp1.gabc}\newpage
\ueberinitiale{\Rbar}{I}
\commentary{\textit{Psalm 27, 9-11}}
\includegabcscore{rp2.gabc}\par
\begin{center}
{\centering{\scalebox{3}{\grecross}}}
\end{center}
\chapter{Complet im Advent}
\def\gebet{\textsc{Complet im Advent}}
 \section*{Psalmodie}
\ueberinitiale{\Abar}{II}
\commentary{{\small \emph{Ps. 79}}}
\includegabcscore{oemmanuel-score.gabc}
\begin{multicols}{2}\setlength{\columnseprule}{0.2pt}
Qui sedes super Chérubim, manifes'táre, \grestar{} coram Éphraim, Béniamin, et 'Manásse.\par
\einzug{Éxcita poténtiam tuam, et 'véni, \grestar{} ut salvos, Dómine, 'nos fácias. (Ps 80, 2f.)}\par\vfill\columnbreak
Gloria Pátri et 'fílio \grestar{} et Spirítu'i Sáncto.\par
\einzug{Sicut erat in princípio et nunc et 'sémper \grestar{} et in sæcula sæculo'rum. Amen.}\par
\end{multicols}\enlargethispage{5em}
\textit{Der Leitvers wird wiederholt.}\par
An Capitel und Responsorium schließt sich der \textbf{Hymnus} an:\par
\ueberinitiale{H}{IV}
\includegabcscore{ghsas-score.gabc}
\begin{multicols}{2}\setlength{\columnseprule}{0.2pt}
Denn es ging dir zu Herzen sehr, / da wir gefangen waren schwer // und sollten gar des Todes sein; / drum nahm er auf sich Schuld und Pein.\par
\einzug{Da sich die Welt zum Abend wandt, / der Bräut'gam Christus ward gesandt. // Aus seiner Mutter Kämmerlein / ging er hervor als klarer Schein.}\par
Gezeigt hat er sein groß Gewalt, / dass es in aller Welt erschallt, // sich beugen müssen alle Knie / im Himmel und auf Erden hie.\par
\einzug{Wir bitten Dich, o heilger Christ, / der du zukünftig Richter bist, // lehr uns zuvor dein' Willen tun / und an dem Glauben nehmen zu.}\par
\Abar Lob, Preis sei, \textsc{Vater}, deiner Kraft / und deinem \textsc{Sohn}, der all Ding schafft, // dem heilgen \textsc{Tröster} auch zugleich, / so hier wie dort im Himmelreich.\par
\end{multicols}
\begin{center}
{\centering{\scalebox{3}{\grecross}}}
\end{center}
\chapter{Dominicis ad Vesperam}
\def\gebet{\textsc{Vesper (lateinisch)}}
\section*{Ingressus}
\commentary{\textit{Dominicis per annum}}
\includegabcscore{ve-ingressus-do.gabc}\par
\commentary{\textit{Ad vesperas paschales}}\enlargethispage{5em}
\includegabcscore{ve-ingressus-pa.gabc}\newpage
\section*{Psalmi}
\ueberinitiale{1. \Abar}{VIII a}
\commentary{\textit{Ps. 116}}
\includegabcscore{credidi.gabc}\par
\vskip9mm
\begin{multicols}{2}\setlength{\columnseprule}{0.2px}
Quoniam confirmata est super nos misericordia 'eius \grestar{} et veritas Domini manet 'in æter\textit{num.}\par
\einzug{Gloria Patri et 'Filio \grestar{} et Spiri'tui Sanc\textit{to.}}\par
Sicut erat in principio et nunc et 'semper \grestar{} et in sæcula sæcu'lorum. A\textit{men.}\par
\end{multicols}
\vskip6bp
\ueberinitiale{1. \Abar}{IV E}
\commentary{\textit{Ps. 113}}
\includegabcscore{asolisortu.gabc}\par
\vskip9mm
\begin{multicols}{2}\setlength{\columnseprule}{0.2px}
Sit nomen Domini 'benedictum \grestar{} ex hoc nunc et 'usque \textit{in} \textit{sæ}culum.\par
\einzug{A solis ortu usque 'ad occasum \grestar{} laudabi'le no\textit{men} Do\textit{mi}ni.}\par
Excelsus super omnes 'gentes Dominus \grestar{} et super cælos 'glori\textit{a} \textit{ei}us.\par
\einzug{Quis sicut Dominus Deus noster, qui in 'altis habitat \grestar{} et humilia respicit in cæ'lo et \textit{in} \textit{ter}ra?}\par
Suscitans a 'terra inopem \grestar{} et de stercore 'eri\textit{gens} pau\textit{pe}rem.\par
\einzug{Ut collocet eum 'cum principibus \grestar{} cum principibus 'popu\textit{li} \textit{su}i.}\par
Qui habitare facit steri'lem in domo \grestar{} matrem fili'orum \textit{lætan}tem.\par
\einzug{Gloria Pat'ri et Filio \grestar{} et Spi'ritu\textit{i San}cto.}\par
Sicut erat in principio et 'nunc et semper \grestar{} et in sæcula sæ'culo\textit{rum. A}men.\par
\end{multicols}
\newpage
\ueberinitiale{2. \Abar}{I a}
\commentary{\textit{Ps 132}}
\includegabcscore{eccequam.gabc}\par
\vskip9mm
\begin{multicols}{2}\setlength{\columnseprule}{0.2px}
Quod descendit in oram vesti'menti \underline{e}ius \grestar{} sicut ros Hermon, qui descendit in 'montem Sion.\par
\einzug{Quoniam illic mandavit Dominus bene'dicti\underline{o}nem \grestar{} et vitam us'que in sæculum.}\par\vfill\columnbreak
Gloria 'Patri et \underline{Fi}lio \grestar{} et Spiri'tui Sancto.\par
\einzug{Sicut erat in principio et 'nunc et \underline{sem}per \grestar{} et in sæcula sæcu'lorum. Amen.}
\end{multicols}
\vskip9bp
\ueberinitiale{2. \Abar}{IV*}
\commentary{\textit{Ps 111}}
\includegabcscore{inmandatis.gabc}\par
\vskip9mm
\begin{multicols}{2}\setlength{\columnseprule}{0.2px}
Potens in terra erit 'semen 'eius, \grestar{} generatio rectorum ben'edicetur.\par
\einzug{Gloria et divitiæ in 'domo 'eius, \grestar{} et iustitia eius manet in sæculum sæculi.}\par
Exortum est in tenebris 'lumen 'rectis, \grestar{} misericors et mise'rator et iustus.\par
\einzug{Iucundus homo, qui miseretur et com'modat, \gredagger{} disponet res suas 'in iu'dicio, \grestar{} quia in æternum non 'commovebitur.}\par
In memoria æterna 'erit 'iustus, \grestar{} ab auditione mala 'non timebit.\par
\einzug{Paratum cor eius, sperans in Do'mino, \gredagger{} confirmatum est cor eius, 'non ti'mebit, \grestar{} donec despiciat inim'icos suos.}\par
Distribuit, dedit paupe'ribus; \gredagger{} iustitia eius manet in sæ'culum 'sæculi, \grestar{} cornu eius exaltabi'tur in gloria.\par
\einzug{Peccator videbit et irasce'tur, \gredagger{} dentibus suis fremet 'et ta'bescet. \grestar{} Desiderium pecca'torum peribit.}\par
Gloria Pat'ri et 'Filio, \grestar{} et Spi'ritui Sancto.\par
\einzug{Sicut erat in principio et 'nunc et 'semper, \grestar{} et in sæcula sæcu'lorum. Amen.}\par
\end{multicols}
\newpage
\ueberinitiale{3. \Abar}{VII c2}
\commentary{\textit{Ps 109}}
\includegabcscore{dixitdns.gabc}\par
\vskip9mm
\begin{multicols}{2}\setlength{\columnseprule}{0.2px}
Virgam virtutis tuæ emittet Domi'nus ex Sion \grestar{} dominare in medio inimi'corum tuo\textit{rum.}\par
\einzug{Tecum principium in die virtutis tu\textit{æ} \gredagger{} in splendori'bus sanctorum \grestar{} ex utero ante luciferum 'genui \textit{te.}}\par
Iuravit Dominus, et non pæni'tebit eum \grestar{} Tu es sacerdos in æternum secundum ordi'nem Mel'chise\textit{dech.}\par
\einzug{Dominus a 'dextris tuis \grestar{} confregit in die iræ 'suæ re\textit{ges.}}\par
Iudicabit in nationibus, im'plebit ruinas \grestar{} conquassabit capita in 'terra multo\textit{rum.}\par
\einzug{De torrente in 'via bibet \grestar{} propterea exal'tabit ca\textit{put.}}\par
Gloria 'Patri et Filio \grestar{} et Spi'ritui Sanc\textit{to.}\par
\einzug{Sicut erat in principio et 'nunc et semper \grestar{} et in sæcula sæcu'lorum. A\textit{men.}}
\end{multicols}\par
\section*{Antiphonæ in vesperis paschalibus}
\ueberinitiale{1. \Abar}{VII d}
\commentary{\textit{Ps 109}}
\includegabcscore{alleluia1.gabc}\par
\vskip9bp
\begin{multicols}{2}\setlength{\columnseprule}{0.2px}
Donec ponam ini'micos \underline{tu}os: \grestar{} scabellum 'pedum tuo\textit{rum}.\par
\einzug{Virgam virtutis tuæ emittet Domi'nus ex \underline{Si}on \grestar{} dominare in medio inimi'corum tuo\textit{rum.}}\par
Tecum principium in die virtutis tu\textit{æ} \gredagger{} in splendori'bus san\underline{cto}rum \grestar{} ex utero ante luciferum 'genui \textit{te.}\par
\einzug{Iuravit Dominus, et non pæni'tebit \underline{e}um \grestar{} Tu es sacerdos in æternum secundum ordi'nem Mel'chise\textit{dech.}}\par
Dominus a 'dextris \underline{tu}is \grestar{} confregit in die iræ 'suæ re\textit{ges.}\par
\einzug{Iudicabit in nationibus, im'plebit ru\underline{i}nas \grestar{} conquassabit capita in 'terra multo\textit{rum.}}\par
De torrente in 'via \underline{bi}bet \grestar{} propterea exal'tabit ca\textit{put.}\par
\einzug{Gloria 'Patri et \underline{Fi}lio \grestar{} et Spi'ritui Sanc\textit{to.}}\par
Sicut erat in principio et 'nunc et \underline{sem}per \grestar{} et in sæcula sæcu'lorum. A\textit{men.}\par
\end{multicols}\newpage
\ueberinitiale{2. \Abar}{VIII f}
\commentary{\textit{Ps 116}}
\includegabcscore{alleluia2.gabc}\par
\vskip9bp
\begin{multicols}{2}\setlength{\columnseprule}{0.2px}
Quoniam confirmata est super nos misericordia 'eius \grestar{} et veritas Domini manet 'in æter\textit{num.}\par
\einzug{Gloria Patri et 'Filio \grestar{} et Spiri'tui Sanc\textit{to.}}\par
Sicut erat in principio et nunc et 'semper \grestar{} et in sæcula sæcu'lorum. A\textit{men.}\par
\end{multicols}
\par
\ueberinitiale{3. \Abar}{IV E}
\commentary{\textit{Ps 113}}
\includegabcscore{alleluia3.gabc}\par
\vskip9bp
\begin{multicols}{2}\setlength{\columnseprule}{0.2px}
Sit nomen Domini 'benedictum \grestar{} ex hoc nunc et 'usque \textit{in} \textit{sæ}culum.\par
\einzug{A solis ortu usque 'ad occasum \grestar{} laudabi'le no\textit{men} Do\textit{mi}ni.}\par
Excelsus super omnes 'gentes Dominus \grestar{} et super cælos 'glori\textit{a} \textit{ei}us.\par
\einzug{Quis sicut Dominus Deus noster, qui in 'altis habitat \grestar{} et humilia respicit in cæ'lo et \textit{in} \textit{ter}ra?}\par
Suscitans a 'terra inopem \grestar{} et de stercore 'eri\textit{gens} pau\textit{pe}rem.\par
\einzug{Ut collocet eum 'cum principibus \grestar{} cum principibus 'popu\textit{li su}i.}\par
Qui habitare facit steri'lem in domo \grestar{} matrem fili'orum \textit{lætan}tem.\par
\einzug{Gloria Pa'tri et Filio \grestar{} et Spi'ritu\textit{i San}cto.}\par
Sicut erat in principio et 'nunc et semper \grestar{} et in sæcula sæ'culo\textit{rum. A}men.\par
\end{multicols}
\section*{Lectio}
\includegabcscore{conclusio-ve-lat.gabc}
\newpage
\section*{Responsorium Breve}
\textbf{I}\par
\ueberinitiale{\Rbar}{VI}
\includegabcscore{rs-benedicam.gabc}\par
\textbf{II}\par
\ueberinitiale{\Rbar}{IV}
\includegabcscore{rs-scapulis.gabc}\par
\section*{Responsoria Prolixa}
\textbf{I}\par
\ueberinitiale{\Rbar}{II}
\includegabcscore{rs-tua.gabc}\par
\textbf{II}\par
\ueberinitiale{\Rbar}{}
\includegabcscore{rs-lucerna.gabc}\par\newpage
\textbf{III}\par
\ueberinitiale{\Rbar}{VIII}
\includegabcscore{rs-inmonte.gabc}\par
\textbf{IV}\par
\ueberinitiale{\Rbar}{IV}
\includegabcscore{rs-petre.gabc}\newpage
\section*{Hymnus}
\includegabcscore{h-lucis.gabc}\par
\vskip9mm
\begin{multicols}{2}\setlength{\columnseprule}{0.2px}
Qui mane iunctum ves\textit{pe}ri / di\textit{em} vocari præcipis // tetrum chaos illabi\textit{tur} / au\textit{di} preces cum fletibus.\par
\einzug{Ne mens gravata cri\textit{mi}ne / vit\textit{æ} sit exsul munere // dum nil perenne co\textit{gi}tat / se\textit{se}que culpis illigat.}\par
Cælorum pulset in\textit{ti}mum / vi\textit{ta}le tollat præmium // vitemus omne no\textit{xi}um / pur\textit{ge}mus omne pessimum.\par
\einzug{\Abar Præsta Pater piis\textit{si}me / Pat\textit{ri}que compar unice // Cum Spiritu Parac\textit{li}to / reg\textit{nans} per omne sæculum. Amen.}\end{multicols}
\vskip9mm
\section*{Versiculi}
\includegabcscore{v-angelis.gabc}\par
\includegabcscore{v-dirigatur.gabc}\par
\commentary{\small\textit{ad vesperas paschales}}
\includegabcscore{v-haecdies.gabc}\newpage
\section*{Magnificat}
\ueberinitiale{\Abar}{VIII G}
\includegabcscore{magnificat.gabc}\par
\textit{vel:}\par
\ueberinitiale{\Abar}{VIII G 2}
\includegabcscore{mg-quisunt.gabc}\par\vskip9bp
\begin{multicols}{2}\setlength{\columnseprule}{0.2px}
Qui\textit{a re}spexit humilitatem 'ancillæ 'suæ \grestar{} ecce enim ex hoc beatam me dicent omnes gene'rationes.\par
\einzug{Qui\textit{a fe}cit mihi 'magna qui 'potens est \grestar{} et sanctum 'nomen eius.}\par
\textit{\textbf{Antiphon}}\par
Et \textit{mise}ricordia eius a progeni'e in pro'genies \grestar{} timen'tibus eum.\par
\einzug{Fec\textit{it po}tentiam in 'brachio 'suo \grestar{} dispersit superbos mente 'cordis sui.}\par
De\textit{posu}it po'tentes de 'sede \grestar{} et exal'tavit humiles.\par
\textit{\textbf{Antiphon}}\par
\einzug{E\textit{suri}entes 'implevit 'bonis \grestar{} et divites dimi'sit inanes.}\par
Sus\textit{cepit} Israel 'puerum 'suum \grestar{} recordatus misericor'diæ suæ.\par
\einzug{Si\textit{cut lo}cutus est 'ad patres 'nostros \grestar{} Abraham et semini e'ius in sæcula.}\par
\textit{\textbf{Antiphon}}\par
Glor\textit{ia} Patri et Filio et Spi'ritui 'Sancto \grestar{} sicut erat in principio 'et nunc et 'semper \grestar{} et in sæcula sæcu'lorum. Amen.\par
\textit{\textbf{Antiphon}}\par
\end{multicols}
\vskip3bp
\section*{Kyrie, Preces, Oratio}
\includegabcscore{kyrie.gabc}\par
\includegabcscore{paternoster.gabc}\par
\includegabcscore{dominus.gabc}\par
\includegabcscore{oratio-v.gabc}\par
\section*{Benedicamus}
\includegabcscore{dominus.gabc}\par
\ueberinitiale{I}{}
\includegabcscore{benedicamus1.gabc}\par
\textit{vel:}\par
\ueberinitiale{II}{}
\includegabcscore{benedicamus2.gabc}\par
\subsection*{Benedictio}
\Vbar~Dominus det nobis suam pacem.\par
\Rbar~Et vitam æternam. Amen.
\begin{center}
{\centering{\scalebox{3}{\grecross}}}
\end{center}
\newpage
\chapter{Ad Completorium in Dominicis}
\def\gebet{\textsc{Complet (lateinisch)}}
\textit{Lector:}\par
\includegabcscore{cp-jube.gabc}\par
\textit{Præses Chori:}\par
\includegabcscore{cp-noctem.gabc}\par
\section*{Lectio (1 Pe 5,8-9)}
\includegabcscore{cp-lectio.gabc}
\section*{Versiculum}
\includegabcscore{cp-adiutorium.gabc}
\newpage
\section*{Confiteor}
\Vbar Confiteor Deo omnipotenti et vobis fratres, quia peccavi nimis, cogitatione, verbo et opere: mea culpa, mea culpa, mea maxima culpa. Ideo precor vos fratres, orare pro me ad Dominum Deum nostrum.\par
\Rbar Misereatur tui omnipotens Deus, et dimissis peccatis tuis, perducat te ad vitam æternam.\\\Vbar Amen.\par
\Rbar Confitemur Deo omnipotenti et tibi frater, quia peccavimus nimis, cogitatione, verbo et opere: nostra culpa, nostra culpa, nostra maxima culpa. Ideo precamur te frater, orare pro nobis ad Dominum Deum nostrum.\par
\Vbar Misereatur vestri omnipotens Deus, et dimissis peccatis vestris, perducat vos ad vitam æternam.\\\Rbar Amen.\par
\Vbar Indulgentiam, absolutionem, et remissionem peccatorum nostrorum tribuat nobis omnipotens et misericors Dominus.\\\Rbar Amen.\par
\vskip0.5em
\section*{Versiculum}
\includegabcscore{cp-converte.gabc}
\section*{Ingressus}
\includegabcscore{cp-ingressus.gabc}\newpage
\ueberinitiale{\Abar}{VIII G}
\commentary{\small\textit{Ps. 4}}
\includegabcscore{a-miserere.gabc}\par
\begin{multicols}{2}\setlength{\columnseprule}{0.2px}
Miserere 'mei \grestar{} et exaudi orati'onem meam.\par
\einzug{Filii hominum, usquequo gravi 'corde \grestar{} ut quid diligitis vanitatem et quæri'tis mendacium?}\par
Et scitote quoniam mirificavit Dominus sanctum 'suum \grestar{} Dominus exaudiet me cum clamave'ro ad eum.\par
\einzug{Irascimini, et nolite peccare \gredagger{} quæ dicitis in cordibus 'vestris \grestar{} in cubilibus vestris 'compungimini.}\par
Sacrificate sacrificium iustitiæ \gredagger{} et sperate in 'Domino \grestar{} Multi dicunt: quis ostendit 'nobis bona?\par
\einzug{Signatum est super nos lumen vultûs tui 'Domine \grestar{} dedisti lætitiam in 'corde meo.}\par
A fructu frumenti, vini et olei 'sui \grestar{} mul'tiplicati sunt.\par
\einzug{In pace in id 'ipsum \grestar{} dormiam et 'requiescam.}\par
Quoniam tu Domine singulariter 'in spe \grestar{} con'stituisti me.\par
\einzug{Gloria Patri et 'Filio \grestar{} et Spiri'tui Sancto.}\par
Sicut erat in principio et nunc et 'semper \grestar{} et in sæcula sæcu'lorum. Amen.
\end{multicols}\par
\section*{Capitel}
\commentary{\small\textit{Jer 14,9}}
\includegabcscore{cp-capitel.gabc}\newpage
\section*{Responsorium Breve}
\ueberinitiale{\Rbar}{VI}
\includegabcscore{rs-inmanus.gabc}\par
\section*{Responsorium Prolixum}
\ueberinitiale{\Rbar}{II}
\includegabcscore{rs-tua.gabc}\newpage
\section*{Hymni}
\textit{secundum ordinem}\par
\includegabcscore{h-telucis.gabc}\par
\begin{multicols}{2}\setlength{\columnseprule}{0.2px}
Procul recedant somnia / et noctium phantasmata // Hostemque nostrum comprime / Ne polluantur corpora.\par
\einzug{\Abar~Præsta Pater piissime / Patrique compar unice // Cum Spiritu Paraclito / Regnans per omne sæculum. Amen.}\par
\end{multicols}
\textit{in festis}\par
\ueberinitiale{H}{II}
\includegabcscore{h-utqueant.gabc}\par
\vskip9bp
\begin{multicols}{2}\setlength{\columnseprule}{0.2px}
Nuntius celso veniens Olympo / Te patri magnum fore nasciturum // Nomen, et vitæ seriem gerendæ / Ordine promit.\par
\einzug{Ille promissi dubius superni, / Perdidit promptæ modulos loquelæ: // Sed reformasti genitus peremptæ / Organa vocis}.\par\vfill\columnbreak
Ventris obstruso recubans cubili / Senseras Regem thalamo manentem: // Hinc parens nati meritis uterque / Abdita pandit.\par
\einzug{\Abar~Sit decus Patri, genitæque Proli, / Et tibi compar utriusque virtus, // Spiritus semper, Deus unus, omni / Temporis ævo. Amen.}\par
\end{multicols}\par\vskip9mm
\section*{Versiculum}
\includegabcscore{cp-custodi.gabc}\newpage
\section*{Canticum Simeonis - \enquote{nunc dimittis}}
\includegabcscore{cp-nuncdimittis.gabc}\par
\begin{multicols}{2}\setlength{\columnseprule}{0.2px}
Quia viderunt 'oculi mei: \grestar{} salu'tare tuum.\par
\einzug{'Quod parasti: \grestar{} ante faciem omnium 'populorum.}\par
Lumen ad revelati'onem gentium: \grestar{} et gloriam plebis 'tuæ Israel.\par
\einzug{Gloria 'Patri et Filio \grestar{} et Spi'ritui Sancto.}\par
Sicut erat in principio et 'nunc et semper \grestar{} et in sæcula sæcu'lorum. Amen.\par\vfill
($\rightarrow$ \textit{Salva nos, Domine...})
\end{multicols}
\section*{Orationes}
\includegabcscore{kyrie2.gabc}\par
\includegabcscore{cp-paternoster.gabc}\par
\begin{center}
\begin{tabular}{p{7cm}p{7cm}}
\Vbar Carnis resurrectionem. & \Rbar Vitam æternam. Amen.\\
\Vbar Dignare, Domine, nocte ista & \Rbar Sine peccato nos custodire.\\
\Vbar Miserere nostri, Domine. & \Rbar Miserere nostri.\\
\Vbar Fiat misericordia tua, Domine, super nos. & \Rbar Quemadmodum speravimus in te.\\
\Vbar Domine, exaudi orationem meam. & \Rbar Et clamor meus ad te veniat.\\
\Vbar Dominus vobiscum. & \Rbar Et cum spiritu tuo.\\ \Vbar Oremus. & \\
\end{tabular}
\end{center}
\subsection*{Oratio}
{\small\textit{Schema melodiæ}}\par
\includegabcscore{cp-schema.gabc}\par
Visita, quæsumus Domine, habitationem istam, et omnes insidias inimici ab ea longe repelle \gredagger{} Angeli tui sancti habitent in ea, qui nos in 'pace custodiant / et benedictio tua sit super nos semper.\par
Per Dominum nostrum Jesum Christum Filium tuum \gredagger{} qui tecum vivit et regnat in unitate Spiri'tus Sancti Deus / per omnia sæcula sæculorum. \Rbar Amen.
\vskip0.6em
\section*{Benedicamus}
\includegabcscore{benedicamus3.gabc}\par
\textit{vel:}\par
\ueberinitiale{II}{}
\includegabcscore{benedicamus2.gabc}\par
\subsection*{Benedictio}
\Vbar Benedicat et custodiat nos omnipotens et misericors Dominus, \grecross Pater, et Filius, et Spiritus Sanctus.\par
\Rbar Amen.
\begin{center}
{\centering{\scalebox{3}{\grecross}}}
\end{center}
\chapter{Antiphonæ B. M. V.}
\def\gebet{\textsc{Marienantiphonen}}
\section*{Salve Regina}\par
\ueberinitiale{\Abar}{I}
\commentary{\small\textit{sollemnis}}
\includegabcscore{salveregina.gabc}\newpage
\section*{Regina cæli}\par
\ueberinitiale{\Abar}{VI}
\includegabcscore{reginacaeli.gabc}\par
\begin{center}
{\centering{\scalebox{3}{\grecross}}}
\end{center}
\end{document}
