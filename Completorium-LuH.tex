\pdfprotrudechars2
\pdfadjustspacing2
\pdfminorversion=7
\pdfcompresslevel=9
\pdfobjcompresslevel=3
\documentclass[initial=ZallmanCaps,staff=19,font=greciliae,11pt,a4paper,openany,twoside,choralsign=PfefferMediaeval]{gregorian}
\usepackage[ngerman,pdfstartview={FitH},bookmarksopen,colorlinks=true,menucolor=black,urlcolor=black,linkcolor=black]{hyperref}
\newfontfeature{Microtype}{protrusion=default;expansion=default;}
\setmainfont[Microtype,Ligatures={Common,TeX},Numbers=OldStyle]{Linux Libertine O}
\setsansfont[Microtype,Ligatures={Common,TeX},Numbers=OldStyle,SmallCapsFont={Alcuin URW SC}]{Alcuin URW}
\setmonofont[Microtype,Ligatures={TeX}]{Inconsolata}
\sloppy
\usepackage[a4paper,twoside,inner=3cm,outer=1.5cm,top=2cm,bottom=2cm]{geometry}
\pagestyle{myheadings}
\makefootrule{myheadings}{\textwidth}{0.4pt}{\footruleskip}
\makeheadrule{myheadings}{\textwidth}{0pt}
\makeevenfoot{myheadings}{\thepage}{\gebet}{}
\makeoddfoot{myheadings}{}{\gebet}{\thepage}
\makeevenhead{myheadings}{\rightmark}{}{}
\makeoddhead{myheadings}{\hphantom{\thechapter}}{}{}
\hbadness 99999
\renewcommand{\greabovelinestextstyle}{\scriptsize}
\begin{document}
\frontmatter
\thispagestyle{empty}
\ThisTileWallPaper{21cm}{\paperheight}{lskj.pdf}
\phantom{l}
\vskip12em
\begin{center}
\Huge \scalebox{1.5}{\huge\bfseries\sffamily ~COMPLETORIUM}\par\vskip0.5em
 \scalebox{2}{\scriptsize\itshape\sffamily\scshape Nachtgebet -- Auszug aus dem Stundengebetbuch}\par
 \vskip3em
 \scalebox{1.3}{\large\bfseries\sffamily\scshape Lutherhaus Jena}
\end{center}
\newpage
\thispagestyle{empty}
\textit{gewidmet Dr. Bernhard Gröbler}\par
\vfill
\begin{center}
\textsc{Ich danke allen,}\par
\textsc{die an der Erstellung dieses Buches direkt und indirekt beteiligt waren,
insbesondere:}\par\vskip0.5em
\textsc{Dem Liturgischen Singkreis Jena und seinen Mitgliedern,}\par
\textsc{Élie Roux und dem Gregorio-Team für Gregorio, mit dem dieses Buch gesetzt wurde,}\par
\textsc{Meiner Familie und meinen Freunden für Unterstützung bei der Erstellung,}\par
\textsc{Der Erzabtei St. Ottilien, an der ich mit dem Gregorianischen Choral in Berührung kam.}\par
\end{center}
\vfill
\begin{framed}
3. Aufl. \textcopyright~2010. \textit{Auf der Grundlage des Buches Liturgie III und des Breviarium Lipsiens\ae{} der Evangelisch Lutherischen Gebetsbruderschaft erstellt und bearbeitet von:}\par
David Gippner M.A.\par
Hans-Berger-Str. 20\par
07747 Jena.\par
\textit{\href{http://creativecommons.org/licenses/by-nc-sa/3.0/de/}{\includegraphics{88x31.png}} Dieses Werk steht unter einer \href{http://creativecommons.org/licenses/by-nc-sa/3.0/de/}{Creative Commons Namensnennung-Keine kommerzielle Nutzung-Weitergabe unter gleichen Bedingungen 3.0 Deutschland Lizenz}.}\par\vskip0.5em
\end{framed}
\newpage
\tableofcontents\vfill
\begin{center}
\scalebox{1.6}{\Huge\initfamily SOLI DEO GLORIA}
\end{center}
\vfill
\newpage
\mainmatter
\input{complet-luh}
   %****************************** Anhang *********************************************************
\chapter{Complet im Advent}
\def\gebet{\textsc{Complet im Advent}}
 \section*{Psalmodie}
\ueberinitiale{\Abar}{II}
\commentary{{\small \emph{Ps. 79}}}
\includegabcscore{oemmanuel-score.gabc}
\begin{multicols}{2}\setlength{\columnseprule}{0.2pt}
Qui sedes super Chérubim, manifes'táre, \grestar{} coram Éphraim, Béniamin, et 'Manásse.\par
\einzug{Éxcita poténtiam tuam, et 'véni, \grestar{} ut salvos, Dómine, 'nos fácias. (Ps 80, 2f.)}\par\vfill\columnbreak
Gloria Pátri et 'fílio \grestar{} et Spirítu'i Sáncto.\par
\einzug{Sicut erat in princípio et nunc et 'sémper \grestar{} et in sæcula sæculo'rum. Amen.}\par
\end{multicols}\enlargethispage{5em}
\textit{Der Leitvers wird wiederholt.}\par
An Capitel und Responsorium schließt sich der \textbf{Hymnus} an:\par
\ueberinitiale{H}{IV}
\includegabcscore{ghsas-score.gabc}
\begin{multicols}{2}\setlength{\columnseprule}{0.2pt}
Denn es ging dir zu Herzen sehr, / da wir gefangen waren schwer // und sollten gar des Todes sein; / drum nahm er auf sich Schuld und Pein.\par
\einzug{Da sich die Welt zum Abend wandt, / der Bräut'gam Christus ward gesandt. // Aus seiner Mutter Kämmerlein / ging er hervor als klarer Schein.}\par
Gezeigt hat er sein groß Gewalt, / dass es in aller Welt erschallt, // sich beugen müssen alle Knie / im Himmel und auf Erden hie.\par
\einzug{Wir bitten Dich, o heilger Christ, / der du zukünftig Richter bist, // lehr uns zuvor dein' Willen tun / und an dem Glauben nehmen zu.}\par
\Abar Lob, Preis sei, \textsc{Vater}, deiner Kraft / und deinem \textsc{Sohn}, der all Ding schafft, // dem heilgen \textsc{Tröster} auch zugleich, / so hier wie dort im Himmelreich.\par
\end{multicols}
\begin{center}
{\centering{\scalebox{3}{\grecross}}}
\end{center}
\chapter{Ad Completorium in Dominicis}
\def\gebet{\textsc{Complet (lateinisch)}}
\textit{Lector:}\par
\includegabcscore{cp-jube.gabc}\par
\textit{Præses Chori:}\par
\includegabcscore{cp-noctem.gabc}\par
\section*{Lectio (1 Pe 5,8-9)}
\includegabcscore{cp-lectio.gabc}
\section*{Versiculum}
\includegabcscore{cp-adiutorium.gabc}
\newpage
\section*{Confiteor}
\Vbar Confiteor Deo omnipotenti et vobis fratres, quia peccavi nimis, cogitatione, verbo et opere: mea culpa, mea culpa, mea maxima culpa. Ideo precor vos fratres, orare pro me ad Dominum Deum nostrum.\par
\Rbar Misereatur tui omnipotens Deus, et dimissis peccatis tuis, perducat te ad vitam æternam.\\\Vbar Amen.\par
\Rbar Confitemur Deo omnipotenti et tibi frater, quia peccavimus nimis, cogitatione, verbo et opere: nostra culpa, nostra culpa, nostra maxima culpa. Ideo precamur te frater, orare pro nobis ad Dominum Deum nostrum.\par
\Vbar Misereatur vestri omnipotens Deus, et dimissis peccatis vestris, perducat vos ad vitam æternam.\\\Rbar Amen.\par
\Vbar Indulgentiam, absolutionem, et remissionem peccatorum nostrorum tribuat nobis omnipotens et misericors Dominus.\\\Rbar Amen.\par
\vskip0.5em
\section*{Versiculum}
\includegabcscore{cp-converte.gabc}
\section*{Ingressus}
\includegabcscore{cp-ingressus.gabc}\newpage
\ueberinitiale{\Abar}{VIII G}
\commentary{\small\textit{Ps. 4}}
\includegabcscore{a-miserere.gabc}\par
\begin{multicols}{2}\setlength{\columnseprule}{0.2px}
Miserere 'mei \grestar{} et exaudi orati'onem meam.\par
\einzug{Filii hominum, usquequo gravi 'corde \grestar{} ut quid diligitis vanitatem et quæri'tis mendacium?}\par
Et scitote quoniam mirificavit Dominus sanctum 'suum \grestar{} Dominus exaudiet me cum clamave'ro ad eum.\par
\einzug{Irascimini, et nolite peccare \gredagger{} quæ dicitis in cordibus 'vestris \grestar{} in cubilibus vestris 'compungimini.}\par
Sacrificate sacrificium iustitiæ \gredagger{} et sperate in 'Domino \grestar{} Multi dicunt: quis ostendit 'nobis bona?\par
\einzug{Signatum est super nos lumen vultûs tui 'Domine \grestar{} dedisti lætitiam in 'corde meo.}\par
A fructu frumenti, vini et olei 'sui \grestar{} mul'tiplicati sunt.\par
\einzug{In pace in id 'ipsum \grestar{} dormiam et 'requiescam.}\par
Quoniam tu Domine singulariter 'in spe \grestar{} con'stituisti me.\par
\einzug{Gloria Patri et 'Filio \grestar{} et Spiri'tui Sancto.}\par
Sicut erat in principio et nunc et 'semper \grestar{} et in sæcula sæcu'lorum. Amen.
\end{multicols}\par
\section*{Capitel}
\commentary{\small\textit{Jer 14,9}}
\includegabcscore{cp-capitel.gabc}\newpage
\section*{Responsorium Breve}
\ueberinitiale{\small\Rbar}{VI}
\includegabcscore{rs-inmanus.gabc}\par
\section*{Responsorium Prolixum}
\ueberinitiale{\small\Rbar}{II}
\includegabcscore{rs-tua.gabc}\newpage
\section*{Hymni}
\textit{secundum ordinem}\par
\includegabcscore{h-telucis.gabc}\par
\begin{multicols}{2}\setlength{\columnseprule}{0.2px}
Procul recedant somnia / et noctium phantasmata // Hostemque nostrum comprime / Ne polluantur corpora.\par
\einzug{\Abar~Præsta Pater piissime / Patrique compar unice // Cum Spiritu Paraclito / Regnans per omne sæculum. Amen.}\par
\end{multicols}
\textit{in festis}\par
\ueberinitiale{\Rbar}{II}
\includegabcscore{h-utqueant.gabc}\par
\vskip9bp
\begin{multicols}{2}\setlength{\columnseprule}{0.2px}
Nuntius celso veniens Olympo / Te patri magnum fore nasciturum // Nomen, et vitæ seriem gerend\ae / Ordine promit.\par
\einzug{Ille promissi dubius superni, / Perdidit promptæ modulos loquelæ: // Sed reformasti genitus perempt\ae / Organa vocis}.\par\vfill\columnbreak
Ventris obstruso recubans cubili / Senseras Regem thalamo manentem: // Hinc parens nati meritis uterque / Abdita pandit.\par
\einzug{\Abar~Sit decus Patri, genitæque Proli, / Et tibi compar utriusque virtus, // Spiritus semper, Deus unus, omni / Temporis ævo. Amen.}\par
\end{multicols}\par\vskip9mm
\section*{Versiculum}
\includegabcscore{cp-custodi.gabc}\newpage
\section*{Canticum Simeonis - \enquote{nunc dimittis}}
\includegabcscore{cp-nuncdimittis.gabc}\par
\begin{multicols}{2}\setlength{\columnseprule}{0.2px}
Quia viderunt 'oculi mei: \grestar{} salu'tare tuum.\par
\einzug{'Quod parasti: \grestar{} ante faciem omnium 'populorum.}\par
Lumen ad revelati'onem gentium: \grestar{} et gloriam plebis 'tuæ Israel.\par
\einzug{Gloria 'Patri et Filio \grestar{} et Spi'ritui Sancto.}\par
Sicut erat in principio et 'nunc et semper \grestar{} et in sæcula sæcu'lorum. Amen.\par\vfill
($\rightarrow$ \textit{Salva nos, Domine...})
\end{multicols}
\section*{Orationes}
\includegabcscore{kyrie2.gabc}\par
\includegabcscore{cp-paternoster.gabc}\par
\begin{center}
\begin{tabular}{p{7cm}p{7cm}}
\Vbar Carnis resurrectionem. & \Rbar Vitam æternam. Amen.\\
\Vbar Dignare, Domine, nocte ista & \Rbar Sine peccato nos custodire.\\
\Vbar Miserere nostri, Domine. & \Rbar Miserere nostri.\\
\Vbar Fiat misericordia tua, Domine, super nos. & \Rbar Quemadmodum speravimus in te.\\
\Vbar Domine, exaudi orationem meam. & \Rbar Et clamor meus ad te veniat.\\
\Vbar Dominus vobiscum. & \Rbar Et cum spiritu tuo.\\ \Vbar Oremus. & \\
\end{tabular}
\end{center}
\subsection*{Oratio}
{\small\textit{Schema melodiæ}}\par
\includegabcscore{cp-schema.gabc}\par
Visita, quæsumus Domine, habitationem istam, et omnes insidias inimici ab ea longe repelle \gredagger{} Angeli tui sancti habitent in ea, qui nos in 'pace custodiant / et benedictio tua sit super nos semper.\par
Per Dominum nostrum Jesum Christum Filium tuum \gredagger{} qui tecum vivit et regnat in unitate Spiri'tus Sancti Deus / per omnia sæcula sæculorum. \Rbar Amen.
\vskip0.6em
\section*{Benedicamus}
\includegabcscore{benedicamus3.gabc}\par
\textit{vel:}\par
\ueberinitiale{II}{}
\includegabcscore{benedicamus2.gabc}\par
\subsection*{Benedictio}
\Vbar Benedicat et custodiat nos omnipotens et misericors Dominus, \grecross Pater, et Filius, et Spiritus Sanctus.\par
\Rbar Amen.
\begin{center}
{\centering{\scalebox{3}{\grecross}}}
\end{center}
\chapter{Antiphonæ B. M. V.}
\def\gebet{\textsc{Marienantiphonen}}
\section*{Salve Regina}\par
\ueberinitiale{\Abar}{I}
\commentary{\small\textit{sollemnis}}
\includegabcscore{salveregina.gabc}\newpage
\section*{Regina cæli}\par
\ueberinitiale{\Abar}{VI}
\includegabcscore{reginacaeli.gabc}\par
\begin{center}
{\centering{\scalebox{3}{\grecross}}}
\end{center}
\end{document}