%*********** Vesper ********************************
  \chapter[Vesper]{\textsc{Vesper}}
\def\gebet{\textsc{Vesper}}
\section*{Ingressus}\par
\ueberinitiale{sonn-}{tags}
\includegabcscore{ingressus-so-lave.gabc}\par
\ueberinitiale{werk-}{tags}
\includegabcscore{ingressus-wo-lave.gabc}\newpage
\section*{Psalmodie} \textit{In der Vesper hängt die Psalmodie von einem Leseplan ab. In diesem Buch gibt es eine kleine Auswahl an Antiphonen.}\par
\begin{framed}
\centering{\textsc{Psalmen zur 1. Antiphon}}
\end{framed}
\ueberinitiale{1. \Abar}{VII d}
\commentary{\textbf{Psalm 120}}
\includegabcscore{vesper-1ant-120.gabc}
\begin{multicols}{2}\setlength{\columnseprule}{0.2pt}
\textsc{Herr}, errette mich von den \underline{Lü}gen'mäulern: \grestar{} von
den \underline{fal}schen 'Zungen.\par
 \einzug{Was soll er dir antun, du falsche Zun\emph{ge} \gredagger{} und was
\underline{dir} noch 'geben: \grestar{} scharfe Pfeile eines Starken und
\ulin{feu}rige Kohlen.}
 Wehe mir, dass ich weilen muss \underline{un}ter 'Meschech: \grestar{} ich muss
bei Kedars \underline{Zel}ten 'wohnen.\par
 \einzug{Es wird meiner Seele lang, zu wohnen bei denen, die den Frieden
has\ulin{sen} \gredagger{} ich \underline{hal}te 'Frieden: \grestar{} aber wenn
ich rede, so fangen \underline{sie} mit 'Streiten an.}
 Ehre sei dem \textsc{Vater} \underline{und} dem '\textsc{Sohne}: \grestar{} und
dem \textsc{\underline{Hei}ligen 'Geiste}.\par
 \einzug{Wie im Anfang, so auch \underline{jetzt} und 'allezeit: \grestar{} und
in \underline{E}wigkeit. 'Amen.}
 \textit{$\rightarrow$ Antiphon}
\end{multicols}
\ueberinitiale{1. \Abar}{I g5}
\commentary{\textbf{Psalm 128}}
\includegabcscore{vesper-1ant-128.gabc}
\begin{multicols}{2}\setlength{\columnseprule}{0.2pt}
Du wirst dich nähren von deiner 'Hände Arbeit: \grestar{} wohl dir, 'du hast es
gut.\par
 \einzug{Dein Weib wird sein wie ein 'fruchtbarer Weinstock: \grestar{} um 'dein
Haus 'herum.}
 Deine Kinder wie 'Zweige des Ölbaums: \grestar{} um 'deinen 'Tisch her.\par
 \einzug{Siehe, also 'wird gesegnet: \grestar{} der Mann, der den 'Herren
'fürchtet.}
 Der \textsc{Herr} wird dich 'segnen aus Zion: \grestar{} dass du sehest das
Glück Jerusalems 'dein Le'ben lang.\par
 \einzug{Und sehest deiner 'Kinder Kinder: \grestar{} Friede 'über 'Israel.}
 Ehre sei dem \textsc{Vater} 'und dem \textsc{Sohne}: \grestar{} und dem
\textsc{Hei'ligen 'Geiste}.\par
 \einzug{Wie im Anfang, so auch 'jetzt und allezeit: \grestar{} und in
E'wigkeit. 'Amen.}
 \textit{$\rightarrow$ Antiphon}
\end{multicols}\newpage
\ueberinitiale{1. \Abar}{IV a}
\commentary{\textbf{Psalm 110}}
\includegabcscore{vesper-1ant-reg.gabc}
\ueberinitiale{1. \Abar}{IV a}
\commentary{Michaelis}
\includegabcscore{vesper-1ant-mich.gabc}
\vskip1em
\begin{multicols}{2}\setlength{\columnseprule}{0.2pt}
Der \textsc{Herr} wird das Zepter deiner Macht ausstrec'ken aus Zion: \grestar{} herrsche mitten un'ter deinen Feinden.\par
 \einzug{Wenn du dein Heer aufbietest, wird dir dein Volk willig folgen in hei'ligem Schmucke: \grestar{} deine Söhne werden dir geboren wie der Tau aus 'der Morgenröte.}
 Der \textsc{Herr} hat geschworen, und es wird ihn 'nicht gereuen: \grestar{} \enquote{Du bist ein Priester ewiglich nach der Wei'se Melchisedechs.}\par
\einzug{Der \textsc{Herr} zu deiner Rechten 'wird zerschmettern: \grestar{} die Könige am Ta'ge seines Zornes.}
 Er wird richten unter den Hei\textit{den} \gredagger{} wird vie'le erschlagen: \grestar{} wird Häupter zerschmettern auf 'weitem Gefilde.\par
 \einzug{Er wird trinken vom Bache 'auf dem Wege: \grestar{} darum wird er 'das 'Haupt emporheben.}
 Ehre sei dem \textsc{Vater} 'und dem '\textsc{Sohne}: \grestar{} und dem '\textsc{Heiligen Geiste}.\par
 \einzug{Wie im Anfang, so auch 'jetzt und allezeit: \grestar{} und in 'Ewigkeit. Amen.}
  \textit{$\rightarrow$ Antiphon}
\end{multicols}\newpage
\begin{framed}
\centering{\textsc{Psalmen zur 2. Antiphon}}
\end{framed}
\ueberinitiale{2. \Abar}{VI F}
\commentary{\textbf{Psalm 123}}
\includegabcscore{vesper-2ant-123.gabc}
\vskip0.4em
\begin{multicols}{2}\setlength{\columnseprule}{0.2pt}
Siehe, wie die Augen der Knechte auf die Hände ihrer Herren se\underline{hen}
\gredagger{} wie die Augen der Magd auf die 'Hände 'ihrer Frau: \grestar{} so
sehen unsere Augen auf den Herren, unsern \textsc{Gott}, bis Er uns 'gnädig
'werde.\par
 \einzug{Sei uns gnädig, \textsc{Herr}, 'sei uns 'gnädig: \grestar{} denn allzu
sehr litten 'wir Ver'achtung.}
 Allzusehr litt unsere Seele den 'Spott der 'Stolzen: \grestar{} und die
Verachtung 'der Hof'färtigen.\par
 \einzug{Ehre sei dem \textsc{Vater} 'und dem '\textsc{Sohne}: \grestar{} und
dem \textsc{Hei'ligen 'Geiste}.}
 Wie im Anfang, so auch 'jetzt und 'allezeit: \grestar{} und in E'wigkeit.
'Amen.\par
 \textit{$\rightarrow$ Antiphon}
\end{multicols}
\ueberinitiale{2. \Abar}{VII a}
\commentary{\textbf{Psalm 111}}
\includegabcscore{vesper-2ant-111-reg.gabc}
\ueberinitiale{2. \Abar}{VII a}
\commentary{Michaelis}
\includegabcscore{vesper-2ant-111-mich.gabc}
\begin{multicols}{2}\setlength{\columnseprule}{0.2pt}
Groß sind die 'Werke des 'Herren: \grestar{} wer sie erforscht, 'der hat 'Freude
dran.\par
 \einzug{Was er tut, das ist 'herrlich und 'prächtig: \grestar{} und seine
Gerechtigkeit 'bleibet 'ewiglich.}
 Er hat ein Gedächtnis gestiftet 'seiner 'Wunder: \grestar{} Der \textsc{Herr}
ist barm'herzig und 'gnädig.\par
 \einzug{Er gibet Speise denen. 'die ihn 'fürchten: \grestar{} er gedenket 'ewig
an 'seinen Bund.}
 Er lässt verkündigen seine gewaltigen 'Taten 'seinem Volk: \grestar{} dass er
ihnen gebe das 'Erbe der 'Heiden.\par
 \einzug{Die Werke seiner Hände sind 'Wahrheit 'und Recht: \grestar{} alle seine
Ordnungen 'sind be'ständig.}
 Sie stehen fest für 'immer und 'ewig: \grestar{} sie sind 'recht und
ver'lässlich.\par
 \einzug{Er sendet eine Erlösung sei\underline{nem Volk,} \gredagger{} er
verheißet, dass sein Bund 'ewig 'bleiben soll: \grestar{} heilig und hehr 'ist
sein 'Name.}
 Die Furcht des Herren ist der Weisheit An\underline{fang} \gredagger{} klug
sind 'alle, die 'danach tun: \grestar{} sein Lob 'bleibet 'ewiglich.\par
 \einzug{Ehre sei dem \textsc{Vater} 'und dem '\textsc{Sohne}: \grestar{} und
dem \textsc{'Heiligen 'Geiste}.}
 Wie im Anfang, so auch 'jetzt und 'allezeit: \grestar{} und in 'Ewigkeit
'Amen.\par
 \textit{$\rightarrow$ Antiphon}
\end{multicols}
\begin{framed}
\centering{\textsc{Psalmen zur 3. Antiphon}}
\end{framed}
\ueberinitiale{3. \Abar}{V a}
\commentary{\textbf{Psalm 138}}
\includegabcscore{vesper-3ant-138.gabc}
\vskip0.4em
\begin{multicols}{2}\setlength{\columnseprule}{0.2pt}
Ich will anbeten vor Deinem heiligen 'Tempel: \grestar{} und Deinen Namen
preisen für Deine 'Güte und Treue.\par
 \einzug{Denn Du hast Deinen 'Namen: \grestar{} und Dein Wort herrlich gemacht
'über alles.}
 Wenn ich Dich anrufe, so er'hörst Du mich: \grestar{} und gibst meiner 'Seele
große Kraft.\par
 \einzug{\textsc{Herr}, es danken Dir auf Erden alle 'Könige: \grestar{} dass
sie hören das Wort 'Deines Mundes.}
 Sie singen von den Wegen des 'Herren: \grestar{} dass die Herrlichkeit des
Herren 'so gewaltig ist.\par
 \einzug{Denn der \textsc{Herr} ist hoch und siehet auf die 'Niedrigen:
\grestar{} und kennet den 'Stolzen von Ferne.}
 Wenn ich mitten in der Angst wandle, so erquickst \underline{du mich}
\gredagger{} und reckest Deine Hand gegen den Zorn meiner 'Feinde: \grestar{}
und hilfst mir mit 'Deiner Rechten.\par
 \einzug{Der \textsc{Herr} wird meine Sache hinausfüh\underline{ren}
\gredagger{} \textsc{Herr}, Deine Güte ist 'ewig: \grestar{} das Werk Deiner
Hände wollest 'Du nicht lassen.}
 Ehre sei dem \textsc{Vater} und dem '\textsc{Sohne}: \grestar{} und dem
'\textsc{Heiligen Geiste}.\par
 \einzug{Wie im Anfang, so auch jetzt und 'allezeit: \grestar{} und in
'Ewigkeit. Amen.}
 \textit{$\rightarrow$ Antiphon}
\end{multicols}\newpage
\ueberinitiale{3. \Abar}{Per}
\commentary{\textbf{Psalm 112}}
\includegabcscore{vesper-3ant-112-reg.gabc}
\ueberinitiale{3. \Abar}{Per}
\commentary{Michaelis}
\includegabcscore{vesper-3ant-112-mich.gabc}
\begin{multicols}{2}\setlength{\columnseprule}{0.2pt}
Sein Geschlecht wird gewal'tig sein im 'Lande: \grestar{} die Kinder der Frommen
'werden \ulin{ge}'segnet sein.\par
 \einzug{Reichtum und Fülle wird 'in ihrem 'Hause sein: \grestar{} und ihre
Gerechtigkeit 'blei\ulin{bet} 'ewiglich.}
 Den Frommen gehet das Licht 'auf in der 'Finsternis: \grestar{} von dem
Gnädigen, Barmherzigen 'und \ulin{Ge}'rechten.\par
 \einzug{Wohl dem, der barmherzig ist \ulin{und} 'gerne 'leihet: \grestar{} und
das Seine tut, 'wie \ulin{es} 'recht ist.}
 Denn er wird 'ewiglich 'bleiben: \grestar{} der Gerechte wird nimmer'mehr
\ulin{ver}'gessen.\par
 \einzug{Vor schlimmer Kund\ulin{e} 'fürchtet 'er sich nicht: \grestar{} bis er
auf seine 'Feinde \ulin{her}'absieht.}
 Er streuet aus und gibt den Ar\underline{men} \gredagger{} seine
Gerechtig\ulin{keit }'bleibet 'ewiglich: \grestar{} seine Kraft wird hoch in
'Eh\ulin{ren} 'stehen.\par
 \einzug{Der Gottlose wird's sehen, und es wird ihn verdrie\underline{ßen}
\gredagger{} mit den Zähnen wird er knir\ulin{schen} 'und ver'gehen: \grestar{}
denn was die Gottlosen wollen, das 'wird \ulin{zu}'nichte.}
 Ehre sei dem \textsc{Va'\ulin{ter}} und dem '\textsc{Sohne}: \grestar{} und dem
'\textsc{Heili\ulin{gen} 'Geiste}.\par
 \einzug{Wie im Anfang, so \ulin{auch} 'jetzt und 'allezeit: \grestar{} und in
'Ewig\ulin{keit.} 'Amen.}
 \textit{$\rightarrow$ Antiphon}
\end{multicols}
\section*{Lektion}
\subsection*{Conclusio}
\ueberinitiale{sonn-}{tags}
\includegabcscore{conclusio-la-so.gabc}
\ueberinitiale{werk-}{tags}
\includegabcscore{conclusio-la-we.gabc}\newpage
\section*{Responsorium} \textit{Am Sonntag stehen zusätzlich die beiden
Responsoria prolixa aus dem Anhang zur Verfügung.}\par
\ueberinitiale{So}{V}
\includegabcscore{vesper-rs-so1.gabc}
\ueberinitiale{Mo-Sa}{II}
\includegabcscore{vesper-rs-we.gabc}\par\newpage
\section*{Hymnus}
\ueberinitiale{sonn-}{tags}
\includegabcscore{vesper-h-so.gabc}
\vskip0.4em
\begin{multicols}{2}\setlength{\columnseprule}{0.2pt}
Die Zeit vom Morgen bis zur Nacht / hast selber Du einst Tag genannt. // Es
bricht die dunkle Nacht herein. / erhöre unser heißes Flehn.\par
 \einzug{Lass unsre Seele, schuldbeschwert / verlieren nicht des Lebens Heil. //
wenn sie an Ewiges nicht denkt / und sich mit Sündenschuld befleckt.}
 Sie klopfe an des Himmels Tor, / empfange ewgen Lebens Lohn; // lass meiden uns
doch alle Sünd / und reingen uns von jeder Schuld.\par
 \einzug{\emph{\Abar} Gewähre dies, \textsc{Gott} \textsc{Vater} mild / und Du,
\textsc{Gott} Sohn, Sein Ebenbild, // die ihr mit \textsc{Gott}, dem Heilgen
Geist, / seid mächtig über Welt und Zeit. Amen.}
\end{multicols}
\ueberinitiale{werk-}{tags}
\includegabcscore{vesper-h-we.gabc}
\vskip0.4em
\begin{multicols}{2}\setlength{\columnseprule}{0.2pt}
Damit die Erde grün und blüh, / geziert mit bunter Blumenpracht, // auch dass
sie reich an Früchten sei / und gute Nahrung biete dar.\par
\einzug{Bring Heilung unserm wunden Herz / durch Deine starke Gnadenkraft, // in
Tränen löse sich die Schuld / und böse Neigung falle ab.}\vfill\columnbreak
Das Herz gehorche Deinem Wort / und bleibe jeder Sünde fern; // es werde alles
Guten voll / und kenne nie des Todes Stich.\par
\einzug{\emph{\Abar} Gewähre dies, \textsc{Gott} \textsc{Vater} mild / und Du,
\textsc{Gott} Sohn, Sein Ebenbild, // die ihr mit \textsc{Gott}, dem Heilgen
Geist, / seid mächtig über Welt und Zeit. Amen.}
\end{multicols}\newpage
\section*{Versikel}\par
\ueberinitiale{So-}{Fr}
\includegabcscore{vesper-ver-sofr.gabc}
\ueberinitiale{Sa}{}
\includegabcscore{vesper-ver-sa.gabc}
\ueberinitiale{Micha-}{elis}
\includegabcscore{laudes-ver-mi.gabc}
\ueberinitiale{Himmel-}{fahrt}
\includegabcscore{laudes-ver-hi.gabc}
\newpage
\section*{Canticum Mariæ}
\ueberinitiale{\Abar}{IV E}
\commentary{\textit{Lk 1, 46 -- 55}}
\includegabcscore{vesper-cm-reg.gabc}
\vskip0.4em
\begin{multicols}{2}\setlength{\columnseprule}{0.2pt}
 Denn Er hat die Niedrigkeit Seiner Magd 'ange'sehen: \grestar{} siehe, von nun
an werden mich selig preisen al\ulin{le} 'Kindes'kinder.\par
 \einzug{Denn Er hat große Dinge an mir getan, 'der da 'mächtig ist: \grestar{}
und \ulin{des} 'Name 'heilig ist.}
 Und Seine Barmherzigkeit währet 'immer 'für und für: \grestar{} bei
de\ulin{nen,} 'die ihn 'fürchten.\par
 \einzug{Er übet Ge'walt mit 'Seinem Arm: \grestar{} und zerstreuet, die
hoffärtig sind \ulin{in} 'ihres 'Herzens Sinn.}
 Er stößet die Gewalti'gen vom 'Throne: \grestar{} und er'\ulin{he}bet die
'Niedrigen.\par
 \einzug{Die Hungrigen füllet 'Er mit 'Gütern: \grestar{} \ulin{und} 'läßt die
'Reichen leer.}
 Er denket 'der Barm'herzigkeit: \grestar{} und hilft Seinem Die\ulin{ner}
'Isra'el auf.\par
 \einzug{Wie Er geredet hat 'unsern 'Vätern: \grestar{} Abraham und
sei\ulin{nen} 'Kindern 'ewiglich.}
 Ehre sei dem \textsc{Vater} 'und dem '\textsc{Sohne}: \grestar{} und dem
'\textsc{Heiligen} '\textsc{Geiste}.\par
 \einzug{Wie im Anfang, so auch 'jetzt und 'allezeit: \grestar{} und in
\ulin{'E}wigkeit. 'Amen.}
  \textit{$\rightarrow$ Antiphon}
\end{multicols}\par
\ueberinitiale{\Abar}{VII a}
\commentary{Michaelis}
\includegabcscore{vesper-cm-mich.gabc}
\vskip0.4em
\begin{multicols}{2}\setlength{\columnseprule}{0.2pt}
 Denn Er hat die Niedrigkeit Seiner Magd 'ange'sehen: \grestar{} siehe, von nun
an werden mich selig preisen al\ulin{le} 'Kindes'kinder.\par
 \einzug{Denn Er hat große Dinge an mir getan, 'der da 'mächtig ist: \grestar{}
und \ulin{des} 'Name 'heilig ist.}
 Und Seine Barmherzigkeit währet 'immer 'für und für: \grestar{} bei
de\ulin{nen,} 'die ihn 'fürchten.\par
 \einzug{Er übet Ge'walt mit 'Seinem Arm: \grestar{} und zerstreuet, die
hoffärtig sind \ulin{in} 'ihres 'Herzens Sinn.}
 Er stößet die Gewalti'gen vom 'Throne: \grestar{} und er'\ulin{he}bet die
'Niedrigen.\par
 \einzug{Die Hungrigen füllet 'Er mit 'Gütern: \grestar{} \ulin{und} 'läßt die
'Reichen leer.}
 Er denket 'der Barm'herzigkeit: \grestar{} und hilft Seinem Die\ulin{ner}
'Isra'el auf.\par
 \einzug{Wie Er geredet hat 'unsern 'Vätern: \grestar{} Abraham und
sei\ulin{nen} 'Kindern 'ewiglich.}
 Ehre sei dem \textsc{Vater} 'und dem '\textsc{Sohne}: \grestar{} und dem
'\textsc{Heiligen} '\textsc{Geiste}.\par
 \einzug{Wie im Anfang, so auch 'jetzt und 'allezeit: \grestar{} und in
\ulin{'E}wigkeit. 'Amen.}
  \textit{$\rightarrow$ Antiphon}
\end{multicols}\par
\section*{Orationes}
\ueberinitiale{sonn-}{tags}
\includegabcscore{laudes-o1.gabc}
\ueberinitiale{werk-}{tags}
\includegabcscore{laudes-o2.gabc}
\subsection*{Preces (werktags)}
\includegabcscore{preces.gabc}
{\itshape\small Im Wechsel weiter:}
\begin{multicols}{2}\setlength{\columnseprule}{0.2pt}
\Vbar \textsc{Herr}, erweise uns Deine Gnade.\par
 \Rbar Und schenke uns Dein Heil.\par
 \Vbar \textsc{Herr}, kehre dich doch wieder zu uns,\par
 \Rbar Und sei deinen Knechten gnädig.\par
 \Vbar Deine Güte, \textsc{Herr}, sei über uns.\par
 \Rbar Wie wir auf Dich hoffen.\par
 \vskip0.4em\hrule\vskip0.4em
\Vbar Lasset uns beten für die heilige Kirche Gottes.\par
 \Rbar \textsc{Herr}, tue wohl an Zion nach \underline{Dei}ner Gnade / baue die
Mauern zu Jerusalem.\par
 \Vbar Es möge Friede sein in deinen Mauern,\par
 \Rbar Und Glück in deinen Palästen.\par
 \Vbar Deine Priester lass sich kleiden mit Gerechtigkeit.\par
 \Rbar Und deine Heiligen sich freuen.\par
 \Vbar Lasset uns beten für unsere Hirten und Lehrer.\par
 \Rbar \textsc{Herr}, nimm nicht von ihrem Munde das Wort der Wahrheit.\par
 \Vbar Lass sie auftreten und weiden in deiner Kraft.\par
 \Rbar Und in der Macht Deines Namens, \textsc{Herr}, unser \textsc{Gott}.\par
 \Vbar Lasset uns beten für alle, die im Glauben unterwiesen werden.\par
 \Rbar \textsc{Herr}, lass sie wachsen in der Gnade und Erkenntnis des Herrn
\textsc{Jesus Christus}.\par
 \Vbar Für die Heimführung des Volkes Israel.\par
 \Rbar Nimm weg, \textsc{Herr}, die Decke von \underline{sei}nem Herzen~/ dass
es sich zu Deinem Sohne bekehre.\par
 \Vbar Für die Ausbreitung des Evangeliums unter den Heiden.\par
 \Rbar Sende Arbeiter in \underline{Dei}ne Ernte / dass alle Menschen zur
Erkenntnis der Wahrheit kommen.\par
 \Vbar Lasset uns beten für unser Volk.\par
 \Rbar Hilf Du uns, \textsc{Gott}, unser Helfer, um Deines Namens Ehre
willen.\par
 \Vbar Für alle Regierenden.\par
 \Rbar \textsc{Herr}, gib ihnen Weisheit und Einsicht gerecht \underline{zu}
regieren / dass Dein Wort geehret werde.\par
 \Vbar Für die Fruchtbarkeit der Erde.\par
 \Rbar Suche das Land heim und wässere es und segne sein Gewächs.\par
 \Vbar Für den Frieden der ganzen Welt.\par
 \Rbar \textsc{Herr}, lass Deine Hilfe nahe sein denen, die Dich fürchten.\par
 \Vbar Dass Güte und Treue einander begegnen.\par
 \Rbar Gerechtigkeit und Friede sich küssen.\par
 \Vbar Gedenke, \textsc{Herr}, Deiner Gemeinde.\par
 \Rbar Die Du vor Zeiten erworben hast.\par
  \vskip0.4em\hrule\vskip0.4em
 \Vbar Breite deine Güte über die, die dich kennen.\par
 \Rbar Und Deine Gerechtigkeit über die Frommen.\par
 \Vbar Lasset uns beten für die Elenden und Betrübten.\par
 \Rbar \textsc{Herr}, stehe ihnen bei und tröste sie.\par
 \Vbar Für die Witwen und Waisen.\par
 \Rbar \textsc{Herr}, lass Deine Güte und Treue allewege sie behüten.\par
 \Vbar Für die Kranken.\par
 \Rbar \textsc{Herr}, erquicke sie nach Deiner Gnade.\par
 \Vbar Lasset uns beten für unsere Widersacher und Verfolger.\par
 \Rbar \textsc{Herr}, behalte ihnen \underline{die}se Sünde nicht, / denn sie
wissen nicht, was sie tun.\par
 \Vbar Für die Abtrünnigen und Verirrten.\par
 \Rbar \textsc{Herr}, weise ihnen den Weg und leite sie auf
richti{\tiny$\downarrow$}ger {\tiny$\uparrow$}Bahn.\par
 \Vbar Für die Gefangenen und Angefochtenen.\par
 \Rbar Erlöse sie, \textsc{Gott} Israel, aus aller ihrer Not.\par
 \Vbar Sende ihnen Hilfe vom Heiligtum.\par
 \Rbar Und stärke sie aus Zion.\par
 \Vbar Lasset uns beten für alle unsre Wohltäter.\par
 \Rbar Gewähre, \textsc{Herr}, allen \underline{die} uns Gutes tun / um Deines
Namens willen das ewige Leben.\par
 \Vbar Für alle Reisenden.\par
 \Rbar Erhöre \underline{uns,} \textsc{Gott}, unser Heil / der Du bist
Zuversicht aller auf Erden und fern am Meere.\par
 \Vbar Für die abwesenden Brüder \textit{(und Schwestern)}.\par
 \Rbar Hilf Du, mein \textsc{Gott}, Deinen Knechten, die sich auf dich
verlassen.\par
 \Vbar Für die Sterbenden.\par
 \Rbar In Deine Hände, \textsc{Herr}, befehlen wir ihren Geist.\par
 \Vbar Lehre uns bedenken, dass wir sterben müssen.\par
 \Rbar Damit wir klug werden.\par
 { \itshape \Vbar Für den / die im Glauben entschlafene/n N. N.\par
 \Rbar \textsc{Herr}, gib ihm / ihr die ewige Ruhe / und das ewige Licht leuchte
ihm / ihr.}\par
 \vskip0.4em\hrule\vskip0.4em
 \Vbar Hilf, \textsc{Herr}, deinem Volke und segne Dein Erbe.\par
 \Rbar Weide die Deinen und trage sie ewiglich.\par
 \Vbar \textsc{Herr} \textsc{Gott} Zebaoth, tröste uns.\par
 \Rbar Lass leuchten dein Antlitz, so genesen wir.\par
 \Vbar Mache Dich auf, CHRISTUS, und hilf uns.\par
 \Rbar Erlöse uns um Deiner Güte willen.\par
 \Vbar \textsc{Herr}, höre mein Gebet.\par
 \Rbar Und lass mein Schreien zu Dir kommen.\par
\end{multicols} 
\subsection*{Collecte und Salutatio\label{Salutatio}}
\includegabcscore{laudes-coll.gabc}
\includegabcscore{laudes-colsa.gabc}
 \vskip0.4em
\begin{multicols}{2}\setlength{\columnseprule}{0.2pt}
 \textbf{Sonntag}\par
 \textit{Gebet des Sonntags nach dem Kirchenjahr.} \ding{118} \ding{166} \ding{167}\par
 \textbf{Montag}\par
 \textsc{Herr} \textsc{Gott}, wir bitten Dich, wende Dich zu unserm demütigen Flehen / und schenke uns nach Deiner großen Güte Vergebung und Frieden. \ding{118}\par
 \textbf{Dienstag}\par
 Wir bitten Dich, \textsc{Herr}, komme uns mit Deiner Barmherzigkeit zuvor / und schenke uns den Reichtum Deiner Gnade, noch ehe wir bitten. \ding{118}\par
 \textbf{Mittwoch}\par
 \textsc{Herr}, wir bitten Dich, vertreibe aus unseren Herzen alles Böse / damit wir mit Zuversicht den Weg des Heiles laufen. \ding{118}\par
 \textbf{Donnerstag}\par
 Erhöre uns, \textsc{Herr}, unser \textsc{Gott}, und regiere Deine Kirche mit Deiner Gnade / und leite sie so durch die Stürme der Welt. \ding{118}\par
 \textbf{Freitag}\par
 Wir bitten Dich, \textsc{Herr}, zerreiße die Fesseln der Sünde, die uns gefangen halten / damit wir freien Herzens Deinen Namen bekennen und preisen. \ding{118}\par
 \textbf{Samstag}\par
 Erhöre gnädig, \textsc{Herr}, die zu Dir rufen / reiße sie aus dem Abgrund der Sünde und führe sie zu den ewigen Freuden. \ding{118}\par
\end{multicols}\newpage
\subsection*{Conclusio}
\includegabcscore{laudes-o3.gabc}
\vskip0.4em\enlargethispage{2\baselineskip}
\ding{118} \Vbar \textit{an \textsmcpit{Gott} \textsmcpit{Vater}:}\\ Durch unsern Herrn \textsc{Jesus Christus}, Deinen Sohn: der mit Dir in der Einheit des \textsc{Heiligen Geistes} ein wahrer \textsc{Gott} / lebet und regieret von Ewigkeit zu Ewigkeit. \Rbar Amen.\par
\ding{166} \Vbar \textit{an \textsmcpit{Gott} \textsmcpit{Sohn}:}\\ der Du mit dem \textsc{Vater} in der Einheit des \textsc{Heiligen Geistes} ein wahrer \textsc{Gott} / lebest und regierest von Ewigkeit zu Ewigkeit. \Rbar Amen.\par
\ding{167} \Vbar \textit{\textsmcpit{Jesus} wird genannt:}\\ Durch IHN, unsern Herrn \textsc{Jesus Christus}, Deinen Sohn: der mit Dir in der Einheit des \textsc{Heiligen Geistes} ein wahrer \textsc{Gott} / lebet und regieret von Ewigkeit zu Ewigkeit. \Rbar Amen.\par
\par\vskip0.5em
\section*{Benedicamus}
\includegabcscore{laudes-coll.gabc}
\ueberinitiale{sonn-}{tags}
\includegabcscore{vesper-ben1.gabc}
\ueberinitiale{werk-}{tags}
\includegabcscore{vesper-ben2.gabc}
\vskip1em \Vbar Der \textsc{Herr} gebe uns Seinen Frieden. \Rbar Und das ewige Leben. Amen.\par
\begin{center}
{\centering{\scalebox{3}{\grecross}}}
\end{center}