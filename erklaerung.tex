\chapter[Einführung]{Einführung in das Gregorianische Brevier}
\def\gebet{\textsc{Einführung}}
Im vorliegenden Buch findet man eine Auswahl aus dem \textit{Breviarium
Lipsiensæ}, dem Leipziger Brevier der Evangelisch-Lutherischen
Gebetsbruderschaft. Diese Auswahl ist so getroffen, dass man ganzjährig die
folgenden vier Gebete in einer Grundform beten kann:\par
\begin{asparaenum}
\item Laudes -- das Morgenlob
\item Sext -- das Mittagsgebet
\item Vesper -- das Abendgebet und
\item Complet -- das Nachtgebet.
\end{asparaenum}
Zusätzlich sind im Anhang reichere Responsorien für die Vesper sowie ein
Reisesegen, das Vaterunser, ein lateinisches Credo und ein Psalmvers und Hymnus
für die Complet in der Adventszeit abgedruckt.\par
Laudes, Sext und Vesper sind teilweise an die Kirchenjahreszeit angepasst; der
Charakter eines Auszugs verbietet aber eine vollständige Anpassung, andernfalls
läge mit diesem Band eine Kopie des Breviers vor.
\section*{Bestandteile}
\begin{asparaenum}
\item Psalmen mit Antiphonen und alttestamentliche Cantica
\item Neutestamentliche Cantica:
\begin{asparaitem}[$\triangleright$]
 \item Das \enquote{Benedictus} (Lobgesang des Zacharias, Lk.1,68-79) in den
Laudes
\item Das \enquote{Magnificat} (Lobgesang der Maria, Lk. 1,46-55) in der Vesper
\item Das \enquote{Nunc dimittis} (Lobgesang des Simeon, Lk. 2,29-32) in der
Complet
\end{asparaitem}
\item Lesungen aus der Bibel nach dem Kirchenjahr (Texte sind in dieser Ausgabe
nicht enthalten)
\item Fürbitten
\end{asparaenum}
Die Ordnung dieses Buches reiht die Tagzeitengebete in ihrer Reihenfolge auf, am
Morgen beginnend. Im Anhang des Buches findet sich der Reisesegen, der vor
Antritt einer Reise gebetet werden kann sowie zwei Responsoria Prolixa für die
Vesper am Sonntag und ein gemeinschaftlich gesungenes Vaterunser und
lateinisches Credo für den Gebrauch in der Complet.
\section*{Wie betet man die einzelnen Teile?}
Das Grundprinzip der Tagzeitengebete ist responsorial, d.h. in den Gebeten gibt
es ständigen Wechsel zwischen einem Vorbeter oder Amtsträger (gekennzeichnet mit
\Vbar) und der versammelten Gebetsgemeinde (gekennzeichnet mit \Rbar). An
Stellen, bei denen alle gemeinsam beten, ist dies mit \Abar gekennzeichnet.\par
Daneben gibt es für die Psalmen und Hymnen eine Einteilung in zwei Gruppen, die
mit römischen Zahlen gekennzeichnet sind. Gruppe I ist auf der Seite des
Kantors, Gruppe II die gegenüberliegende.\par
Am Anfang der Gebete steht der Ingressus, der den Beginn des Gebets markiert.
Dabei steht die ganze Gebetsgemeinde.\par
In der Complet geht ihm ein Segen, eine Lesung und ein gemeinsames
Schuldbekenntnis voran.\par
In der Laudes, Vesper und Complet folgt darauf die Psalmodie. Die Antiphon und
der erste Psalmvers werden dabei vom Cantor (markiert mit \Vbar) bis zum Stern
angestimmt, in der Antiphon fallen alle ein, im ersten Psalmvers die Gruppe I.
Am Ende des Psalms wird die Antiphon wiederholt.\par
Responsorien stimmt der Kantor an und wird ab dem Stern dabei von der Schola
(aus anwesenden Amtsträgern) unterstützt. Die gesamte Gebetsgemeinde wiederholt
den Gesang, nach dem von der Schola gesungenen Vers nur die zweite Hälfte, nach
dem Gloria Patri wieder den gesamten Responsorialvers.\par
Beim Hymnus wird die erste Strophe vom Kantor angestimmt und er dann ab dem
Stern von Gruppe I unterstützt, danach die Strophen im Wechsel gesungen. Vor der
letzten Strophe steht die Gebetsgemeinde auf und bleibt bis zum Ende des Gebetes
stehen.\par
Folgende Ämter gibt es:\par
\begin{asparaenum}
\item Lektor -- er/sie liest die Lesungen und das Capitel in der Complet.
\item Kantor -- er/sie stimmt Gesänge an (Psalmodie, Cantica, Responsorium).
\item Hebdomadarius -- \enquote{Wochendiensthaber}; er/sie betet die Gebete vor.
\item Præses Chori -- geistlicher Leiter des Gebetes; er/sie erteilt den
Lesesegen und den Abschlusssegen.
\end{asparaenum}
\section*{Besonderheiten}
Von Aschermittwoch bis Ostersonntag entfällt das Halleluja in allen Gebeten und
wird durch \enquote{Lob sei Dir, \textsc{Herr}, du König der ewigen
Herrlichkeit} ersetzt. Vom Sonntag Judica bis Karsamstag entfällt das Gloria
Patri in den Responsorien. Für Himmelfahrt und St. Michael finden sich für
Laudes, Sext und Vesper gesonderte Versikel.
\section*{Quadratnotation und wie sie gesungen wird\protect\footnote{\textbf{Quelle:} Bernhard K. Gröbler, Einführung in den Gregorianischen Choral, 2. Aufl. Jena 2005, S. 144.}}
\subsection*{Schlüssel}
\includegabcscore{schluessel.gabc}
\includegabcscore{schluesself.gabc}
Die vom Schlüssel umschlossene Linie markiert das c bzw. f.
\subsection*{Einzeltöne}
\includegabcscore{einzeltoene.gabc}\par
\subsection*{Schnelle Tonverbindungen (Ligaturen)}
\includegabcscore{ligaturen.gabc}
Umschrift:\par
{%
\parindent 0pt
\noindent
\ifx\preLilyPondExample \undefined
\else
  \expandafter\preLilyPondExample
\fi
\def\lilypondbook{}%
\input 57/lily-42def9a8-systems.tex
\ifx\postLilyPondExample \undefined
\else
  \expandafter\postLilyPondExample
\fi
}\par
Der letzte Ton von Ligaturen ist stets leicht gedehnt bzw. artikuliert.
\subsection*{Gedehnte Noten}
\includegabcscore{episem.gabc}
Bei Clivis und Pes gilt das Episem für beide Töne, wenn es bei der ersten Note
steht, sonst gilt es nur für den zweiten Ton.
\subsection*{\enquote{Liqueszierende} Noten (Liqueszenzneumen)}
\includegabcscore{liqueszenz.gabc}
Die kleine Note wird auf einen klingenden Konsonant (Semivokal) am Silbenende
gesungen. Auch am Ende einer Ligatur sind entsprechende Liqueszenznoten möglich.
\subsection*{Neographien}
\includegabcscore{quilisma.gabc}
Die gezackte Note ist ein kurzer, schwacher Ton, der zum Zielton leitet und kann
ähnlich einem Glissando gesungen werden.\par
\includegabcscore{neographien.gabc}
\subsection*{Reperkussion}
\includegabcscore{reperkussion.gabc}
Der doppelte Ton wird auf gleicher Höhe neu angesetzt.
\subsection*{Custos}
Der Custos ist eine halbe Note am Ende einer Notenzeile, der den nachfolgenden
Ton auf der nächsten Zeile anzeigt.\par
\section*{Lektionstöne\protect\footnote{Quelle: Breviarium Lipsiensae 1988, S. 78f.}}
Für die nicht vorgegebenen Lesungen in Laudes, Sext und Vesper gelten die untenstehenden Melodiemodelle. Der jeweilige Abschluss (Conclusio) findet sich in den Gebeten jeweils an der Stelle der Lesung.\par
\subsection*{Laudes und Vesper sonntags}
\includegabcscore{lt-lave-so.gabc}
\subsection*{Laudes und Vesper werktags, Sext sonntags}
\includegabcscore{lt-lave-we.gabc}
\subsection*{Sext werktags}
\includegabcscore{lt-se-we.gabc}