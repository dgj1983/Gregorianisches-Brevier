\chapter[Complet]{\textsc{Complet}}
\def\gebet{\textsc{Complet}}
\textit{Der Lektor bittet um den Lesesegen:}\par
\includegabcscore{complet-lektor.gabc}
 \vskip0.5em
 \vskip0.5em\noindent \textit{Der Præses Chori erteilt ihn:}\par
\includegabcscore{complet-praeses.gabc}
 \vskip0.5em
\section*{Lektion}
\commentary{\textit{1. Pe 5,8-9}}
\includegabcscore{complet-lektion.gabc}
\section*{Versikel (werktags)}
\includegabcscore{complet-versikel1.gabc}
\newpage
\section*{Confiteor (werktags)}
\begin{multicols}{2}\setlength{\columnseprule}{0.2pt}
\subsection*{Bis zu zwei Beter}
\Abar Ich bekenne Dir, heiliger \textsc{Gott}, dass ich oft und auf mancherlei Weise gegen Dich gesündigt habe mit Gedanken, Worten und Werken -- durch meine Schuld, durch meine Schuld, durch meine übergroße Schuld. Gedenke mein nicht nach meinen Sünden, sondern nach Deiner großen Barmherzigkeit und vergib mir meine Schuld um Jesu Christi willen. Amen.\par
\Vbar Wenn wir unsre Sünden bekennen, dann ist \textsc{Gott} treu und gerecht, dass Er uns die Sünde vergibt und reinigt uns von aller Untugend.\par
\Abar Amen.\par
\vfill\columnbreak
\subsection*{Drei und mehr Beter} \Vbar Ich bekenne \textsc{Gott}, dem Allmächtigen, und euch, (Brüder), dass ich oft und viel gesündigt habe mit Gedanken, Worten und Werken -- durch meine Schuld, durch meine Schuld, durch meine übergroße Schuld. Darum bitte ich euch, (Brüder), dass ihr für mich betet zu \textsc{Gott}, unserm Herren.\par
\Rbar Der allmächtige \textsc{Gott} erbarme sich deiner, Er vergebe dir deine Sünden und führe dich zum ewigen Leben.\par
\Vbar Amen.\par
\Rbar Wir bekennen \textsc{Gott}, dem Allmächtigen, und dir, (Bruder), dass wir oft und viel gesündigt haben mit Gedanken, Worten und Werken -- durch unsre Schuld, durch unsre Schuld, durch unsre übergroße Schuld. Darum bitten wir dich, (Bruder), dass du für uns betest zu \textsc{Gott}, unserm Herren.\par
\Vbar Der allmächtige \textsc{Gott} erbarme sich euer, Er vergebe euch eure Sünden und führe euch zum ewigen Leben.\par
\Rbar Amen.\par
\Vbar Nachlass, Lossprechung und Vergebung unsrer Sünden schenke uns der allmächtige und barmherzige \textsc{Herr}.\par
\Rbar Amen.
\end{multicols}
\section*{Versikel}
\ueberinitiale{sonn-}{tags}
\includegabcscore{complet-versikel2-so.gabc}
\ueberinitiale{werk-}{tags}
\includegabcscore{complet-versikel2-we.gabc}
\newpage
\section*{Ingressus}
\ueberinitiale{sonn-}{tags}
\includegabcscore{ingressus-so-seco.gabc}\par
\ueberinitiale{werk-}{tags}
\includegabcscore{ingressus-wo-seco.gabc}
\newpage
\section*{Psalmodie}
\ueberinitiale{\Abar}{VIII G}
\commentary{\textbf{Psalm 4/91/134}}
\includegabcscore{complet-psalmodie.gabc}
\includegabcscore{complet-ps-4.gabc}
\vskip1em
\begin{multicols}{2}\setlength{\columnseprule}{0.2pt}
Der Du mich tröstest in meinen \underline{Ängs}ten: \grestar{} sei mir gnädig und er\underline{hö}re mein Gebet.\par
 \einzug{Ihr Herren, wie lange soll meine Ehre geschändet \underline{wer}den? \grestar{} Wie habt ihr das Eitle so lieb und die Lü\underline{ge} so gerne?}
 Erkennet doch, dass der \textsc{Herr} seine Heiligen wunderbar \underline{fü}hret: \grestar{} der \textsc{Herr} höret, wenn ich \underline{ihn} anrufe.\par
 \einzug{Zürnet ihr, so \underline{sün}digt nicht: \grestar{} redet mit eurem Herzen auf eurem Lager \underline{und} seid stille.}
 Opfert, was \underline{recht} ist: \grestar{} und hoffet \underline{auf} den Herren.\par
 \einzug{Viele sagen - wer wird uns Gutes sehen \underline{las}sen? \grestar{} \textsc{Herr}, lass über uns das Licht Deines Ant\underline{lit}zes leuchten.}
 Du er\underline{freust} mein Herz: \grestar{} ob jene auch viel Wein \underline{und} Korn haben.\par
 \einzug{Ich liege und schlafe ganz mit \underline{Frie}den: \grestar{} denn alleine Du, \textsc{Herr}, hilfst mir, dass ich \underline{si}cher wohne.}
 \textbf{Psalm 91}\par
 Wer unter dem Schirm des Höchsten \underline{sit}zet: \grestar{} und unter dem Schatten des Allmäch\underline{ti}gen bleibet.\par
 \einzug{Der spricht zu dem Herren: meine Zuversicht und \underline{mei}ne Burg: \grestar{} mein \textsc{Gott}, auf \underline{den} ich traue.}
 Denn Er errettet dich vom Stricke des \underline{Jä}gers: \grestar{} und von
der verderb\underline{li}chen Seuche.\par
 \einzug{Er wird dich mit seinen Fittichen \underline{dec}ken: \grestar{} und Zuflucht wirst du unter seinen \underline{Flü}geln haben.}
 Seine Wahrheit ist Schirm \underline{und Schild}, \gredagger{} dass du nicht erschrecken musst vor dem nächtlichen \underline{Grau}en: \grestar{} vor den Pfeilen, die des \underline{Ta}ges fliegen.\par
 \einzug{Vor der Pest, die im \underline{Fins}tern schleicht: \grestar{} vor der Seuche, die am Mit\underline{tag} Verderben bringt.}
 Wenn auch tausend fallen zu deiner Sei\underline{te} \gredagger{} und zehntausend zu deiner \underline{Rech}ten: \grestar{} so wird es doch \underline{dich} nicht treffen.\par
 \einzug{Ja, du wirst es mit eigenen Augen \underline{se}hen: \grestar{} und schauen, wie den Gottlo\underline{sen} vergolten wird.}
 Denn der \textsc{Herr} ist deine \underline{Zu}versicht: \grestar{} der Höchste ist \underline{dei}ne Zuflucht.\par
 \einzug{Es wird dir kein Übel be\underline{geg}nen: \grestar{} und keine Plage wird sich deinem \underline{Hau}se nahen.}
 Denn er hat seinen Engeln be\underline{foh}len: \grestar{} dass sie dich behüten auf allen \underline{dei}nen Wegen.\par
 \einzug{Dass sie dich auf den Händen \underline{tra}gen: \grestar{} und du deinen Fuß nicht an ei\underline{nen} Stein stößest.}
 Über Löwen und Ottern wirst du \underline{ge}hen: \grestar{} und junge Löwen und Drachen \underline{nie}dertreten.\par
 \einzug{Er liebet Mich, darum will ICH ihn erret\underline{ten} \gredagger{} er kennet Meinen Namen, darum will ICH ihn \underline{schüt}zen: \grestar{} er rufet Mich an, darum will ICH \underline{ihn} erhören.}
 ICH bin bei ihm \underline{in} der Not: \grestar{} ICH will ihn herausreißen und zu \underline{Eh}ren bringen.\par
 \einzug{ICH will ihn sättigen mit langem \underline{Le}ben: \grestar{} ICH will ihm \underline{zei}gen mein Heil.}
 \textbf{Psalm 134}\par
 Wohlan, lobet den Herren, alle Knechte des \textsc{\underline{Her}ren:}
\grestar{} die ihr stehet des Nachts im Hau\underline{se} des
\textsc{Herren.}\par
 \einzug{Hebet eure Hände auf zum \underline{Hei}ligtum: \grestar{} und
lo\underline{bet} den \textsc{Herren}.}
 Der \textsc{Herr} segne dich aus \underline{Zi}on: \grestar{} der Himmel und Er\underline{de} gemacht hat.\par
 \einzug{Ehre sei dem \textsc{Vater} und dem \textsc{\underline{Soh}ne}: \grestar{} und dem \textsc{Hei\underline{li}gen Geiste}.}
  Wie im Anfang, so auch jetzt und \underline{al}lezeit: \grestar{} und in E\underline{wig}keit. Amen.\par
 \textit{$\rightarrow$ Antiphon}
\end{multicols}
 \newpage
\section*{Capitel}
\commentary{\textit{Jer 14,9}}
\includegabcscore{complet-capitel.gabc}
\section*{Responsorium Breve}
\ueberinitiale{sonn-}{tags}
\includegabcscore{complet-rs-so.gabc}
\ueberinitiale{werk-}{tags}
\includegabcscore{complet-rs-we.gabc}
\newpage
\section*{Hymnen}
\textit{\small An die letzte Strophe des Hymnus schließt sich das Amen an.}\par
\ueberinitiale{Mo-Fr}{II}
\commentary{Sommer}
\includegabcscore{complet-h-we1.gabc}\par
\ueberinitiale{Mo-Fr}{VIII}
\commentary{Winter}
\includegabcscore{complet-h-we2.gabc}
\vskip1em
\begin{multicols}{2}\setlength{\columnseprule}{0.2pt}
 Der allem Form und Wesen gibt, / den Unterschied der Zeiten setzt: // erquicke
Du durch Ruh der Nacht / die Leiber von der Arbeit müd.\par
\einzug{Dich flehen wir in Demut an: / mach uns vom Widersacher frei; // dass er
nicht Macht hab, zu verführn, / die Du erkauft mit Deinem Blut.}
Solang im schlafestrunknen Leib / wir bleiben eine kurze Zeit: // lass unser
Fleisch dann also ruhn, / dass unser Herz vom Schlaf nichts weiß.\par
\einzug{\textit{\Abar} O milder König \textsc{Jesu Christ}, / Dir und dem
\textsc{Vater} sei die Ehr // zusamt dem Tröster, \textsc{Heilgen Geist}, /
jetzt und in alle Ewigkeit. Amen.}
\end{multicols}
\ueberinitiale{Mo-Fr}{VIII}
\includegabcscore{complet-h-we3.gabc}
\begin{multicols}{2}\setlength{\columnseprule}{0.2pt}
Hüllt Schlaf die müden Glieder ein, / lass uns in Dir geborgen sein // und mach
am Morgen uns bereit / zum Lobe Deiner Herrlichkeit.\par\vfill\columnbreak
\textit{\Abar} Dank Dir, o \textsc{Vater} reich an Macht, / der über uns voll
Güte wacht // und mit dem \textsc{Sohn} und \textsc{Heilgen Geist} / des Lebens
Fülle uns verheißt. Amen.\par
\end{multicols}\par \newpage
\ueberinitiale{Sa}{VII}
\commentary{Sommer}
\includegabcscore{complet-h-sa1.gabc}\par
\ueberinitiale{Sa}{IV}
\commentary{Winter}
\includegabcscore{complet-h-sa2.gabc}
\vskip1em
\begin{multicols}{2}\setlength{\columnseprule}{0.2pt}
Damit die Ruh den matten Leib / dem Brauch der Arbeit wieder geb, // den müden
Sinn erheitere, / ihn lös von Angst und Traurigkeit.\par
\einzug{Da nun der Tag vergangen ist, / die Nacht beginnt, so bitten wir: //
steh uns Gebundnen immer bei, / die jetzt Dir singen Dankes Lied.}
Dich preise unsres Wesens Grund, / Dich lob der Wohllaut unsrer Stimm, // Dich
liebe keusche Liebe recht, / Dich bete nüchtern an das
Herz.\par\vfill\columnbreak
\einzug{Auf dass, wenn tiefe Dunkelheit / der Nacht den lichten Tag beschließt,
// der Glaub nichts weiß von Finsternis, / die Nacht ihm leuchte wie der Tag.}
Lass unser Herz nicht müde sein, / mit Gnad bedecke alle Schuld: // des keuschen
Glaubens Nüchternheit, / sie kühle unsrer Träume Glut.\par
\einzug{Vom bösen Trachten freigemacht, / des Herzens Tiefe träum von Dir: //
dass nicht durch bösen Feindes List / die Angst aufjag die Ruhenden.}
\textit{\Abar} So bitten wir den Einen \textsc{Gott}: / den \textsc{Vater},
\textsc{Sohn} und Beider \textsc{Geist}: // den Flehenden allmächtig helf /
durch alles die Dreieinigkeit. Amen.\par
\end{multicols}\newpage
\ueberinitiale{So}{IV}
\commentary{Sommer}
\includegabcscore{complet-h-so1.gabc}\par
\ueberinitiale{So}{I}
\commentary{Winter}
\includegabcscore{complet-h-so2.gabc}
\vskip1em
\begin{multicols}{2}\setlength{\columnseprule}{0.2pt}
Lass Träume fern von hinnen fliehn / samt allem Wahngebild der Nacht / dämpf
unsres Widersachers List / und halt die Leiber unbefleckt.\par\vfill\columnbreak
\textit{\Abar} Allmächtger \textsc{Vater}, das verleih / durch \textsc{Jesum
Christum}, unsern Herrn: / der mit Dir selbst in Ewigkeit / regiert zusamt dem
\textsc{Heilgen Geist}. Amen.\par
\end{multicols}
\section*{Versikel}
\includegabcscore{complet-versikel3.gabc}
\newpage
\section*{Canticum Simeonis - \enquote{nunc dimittis}}
\ueberinitiale{\Abar}{III}
\commentary{\textit{Lk 2, 29-32}}
\includegabcscore{complet-nd.gabc}
\vskip0.6em
\begin{multicols}{2}\setlength{\columnseprule}{0.2pt}
Denn meine Augen haben Deinen 'Heiland gesehen: \grestar{} welchen du bereitet
hast vor 'allen Völkern.\par
\einzug{Ein Licht zu er'leuchten die Heiden: \grestar{} und zum Preise deines
'Volkes Israel.}\vfill\columnbreak
Ehre sei dem \textsc{Vater} 'und dem \textsc{Sohne}: \grestar{} und dem
'\textsc{Heiligen Geiste}.\par
\einzug{Wie im Anfang, so auch 'jetzt und allezeit: \grestar{} und in 'Ewigkeit.
Amen.}
\textit{$\rightarrow$ Antiphon}
\end{multicols} \vskip0.5em \vskip0.4em
\section*{Orationes}
\textit{\small Hier kann alternativ auf S. \pageref{Vaterunser} fortgefahren werden.}\par
\includegabcscore{complet-orationes.gabc}
\includegabcscore{complet-vaterunser.gabc}\newpage
\begin{leftbar}
\textsc{\sffamily werktags:}\par
\noindent \Vbar Ich glaube an Gott, den Vater \textit{(still bis)} \par
\begin{tabular}{ll}
\Vbar Auferstehung der Toten & \Rbar Und das Ewige Leben. Amen.\\\label{OratComplet}
\Vbar O \textsc{Herr}, bewahre uns in dieser Nacht. &  \Rbar nach Deiner Gnade ohne Sünde.\\
\Vbar Sei uns gnädig, \textsc{Herr}. & \Rbar Sei uns gnädig.\\
\Vbar Deine Güte, \textsc{Herr}, sei über uns. & \Rbar Wie wir auf Dich hoffen.\\
\Vbar \textsc{Herr}, höre mein Gebet &  \Rbar und lass mein Schreien zu dir kommen.\\
\end{tabular}
\end{leftbar}
\section*{Tagesgebete}
\includegabcscore{complet-colsa.gabc}
\begin{multicols}{2}\setlength{\columnseprule}{0.2pt} \textbf{Sonntag}\par
 Wir bitten Dich, \textsc{Herr}, suche gnädig heim dieses Haus und vertreibe alle List des Feindes: lass Deine heiligen Engel bei uns wohnen und uns in Frieden bewahren / und dein Segen sei allezeit über uns.\par
 \textbf{Montag}\par
 \textsc{Herr} \textsc{Gott}, dem der Tag und die Nacht gehört: lass, wenn die Finsternis kommt, die Sonne der Gerechtigkeit uns aufgehen / und das Dunkel unheiliger Gedanken vertreiben. \ding{118}\par
 \textbf{Dienstag}\par
 Allmächtiger, ewiger \textsc{Gott}, wir denken des Nachts an deinen Namen: und bitten Dich -- treibe alle Finsternis der Sünde aus unseren Herzen / und führe uns zum wahren Lichte Jesus Christus. \ding{167}\par
 \textbf{Mittwoch}\par
 Wir bitten dich, \textsc{Herr}, schenke uns eine ruhige Nacht und bewahre uns vor der Gewalt des Teufels, damit wir in Deinem Frieden schlafen / und wenn der Tag anbricht, Deinen Namen preisen. \ding{118}\par
 \textbf{Donnerstag}\par
 \textsc{Herr} \textsc{Gott}, Du wachest über uns, damit uns die Schrecken der Nacht nicht bedrohen: bewahre uns durch himmlischen Schutz / und sei Du in unseren Herzen, wenn wir schlafen, \ding{118}\par
\textbf{Freitag}\par
 \textsc{Herr} \textsc{Jesus Christus}, Du Erlöser aller Menschen: Du hast uns mit deinem teuren Blute erkauft / schenke uns, so mit dem Leibe zu ruhen, dass wir im Glauben allezeit mit dir wachen. \ding{166}\par
\textbf{Samstag}\par
 Wache über uns, \textsc{Herr}, und bewahre uns vor allem Übel an Leib und Seele: verleihe gnädig, dass wir in dieser Nacht sicher unter Deinem Schutze ruhn / und wenn dann unser letzter Abend kommt, lass uns einschlafen in Frieden, dass wir erwachen zu Deiner Herrlichkeit. \ding{118}\par
\end{multicols} \normalsize
\subsection*{Conclusio}
\includegabcscore{coll-con-co.gabc}
\vskip0.4em
\ding{118} \Vbar \textit{an \textsmcpit{Gott Vater}:}\\ Durch unsern Herrn \textsc{Jesus Christus}, Deinen Sohn: \ding{167} der mit Dir in der Einheit des \textsc{Heiligen Geistes} ein wahrer \textsc{Gott} / lebet und regieret von Ewigkeit zu Ewigkeit. \Rbar Amen.\par
\ding{166} \Vbar \textit{an \textsmcpit{Gott Sohn}:}\\ der Du mit dem \textsc{Vater} in der Einheit des \textsc{Heiligen Geistes} ein wahrer \textsc{Gott} / lebest und regierest von Ewigkeit zu Ewigkeit. \Rbar Amen.\par\newpage
\section*{Benedicamus}
\Vbar Der \textsc{Herr} sei mit euch. \Rbar Und mit deinem Geiste.\par
\ueberinitiale{sonn-}{tags}
\includegabcscore{benedicamus-seco-so.gabc}
\ueberinitiale{werk-}{tags}
\includegabcscore{benedicamus-seco-we.gabc}
\vskip1em
\section*{Benedictio}
\Abar \textit{kurze Gebetsstille}\par
\textit{(auf einem beliebigen, tiefen Ton)}\par
 \Vbar Es segne und behüte uns der allmächtige und barmherzige \textsc{\textsc{Herr}}, \grecross\par
 der \textsc{Vater}, der \textsc{Sohn} und der \textsc{Heilige Geist}.\par
 \Rbar Amen. \par
\begin{center}
{\centering{\scalebox{3}{\grecross}}}
\end{center}

