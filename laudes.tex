\chapter[Laudes]{\textsc{Laudes}}
\def\gebet{\textsc{Laudes}}
\section*{Ingressus}
\ueberinitiale{sonn-}{tags}
\includegabcscore{ingressus-so-lave.gabc}\par
\ueberinitiale{werk-}{tags}
\includegabcscore{ingressus-wo-lave.gabc}\newpage
\section*{Alttestamentliches Canticum (fakultativ)}
\commentary{\textbf{Lied der Hannah}}
\ueberinitiale{\Abar}{VII a}
\includegabcscore{laudes-lh.gabc}
\begin{multicols}{2}\setlength{\columnseprule}{0.2pt}
 Mein Mund hat sich aufgetan wider 'meine Feinde, \grestar{} denn ich freue mich
'deines Heiles.\par
 \einzug{Es ist niemand heilig wie der \textsc{Herr}, \gredagger{} außer
'Dir ist keiner \grestar{} und ist kein Fels wie 'unser Gott ist.}
 Lasst euer großes 'Rühmen und Trotzen; \grestar{} freches Reden gehe nicht aus
'eurem Munde.\par
 \einzug{Denn der \textsc{Herr} ist ein Gott, 'der es merket \grestar{} und von
ihm werden 'Taten gewogen.}
 Der Bogen der Starken 'ist zerbrochen \grestar{} und die Schwachen sind
um'gürtet mit Stärke.\par
 \einzug{Die da satt waren, müssen 'um Brot dienen \grestar{} und die Hunger
litten, 'hungert nicht mehr.}
 Die Unfruchtbare hat 'sieben geboren \grestar{} und die viele Kinder hatte,
'welket dahin.\par
 \einzug{Der \textsc{Herr} tötet und 'macht lebendig \grestar{} führt hinab zu
den Toten und 'wieder herauf.}
 Der \textsc{Herr} macht 'arm und macht reich \grestar{} er er'niedrigt und
erhöht.\par
 \einzug{Er hebt auf den Dürftigen 'aus dem Staube \grestar{} und erhöht den
Armen 'aus der Asche.}
 Dass er ihn setze 'unter die Fürsten \grestar{} und den Thron der Ehre 'erben
lasse.\par
 \einzug{Denn der Welt Grundfesten 'sind des Herren \grestar{} und er hat die
'Erde daraufgesetzt.}
 Er wird behüten die Füße seiner Heiligen \gredagger{} aber die Gottlosen
sollen zunichte 'werden in Finsternis, \grestar{} denn viel Macht 'hilft doch
niemand.\par
 \einzug{Die mit dem Herrn hadern, sollen zugrunde gehen. \gredagger{} Der
Höchste im Himmel wird 'sie zerschmettern; \grestar{} der \textsc{Herr} wird
richten 'der Welt Enden.}
Ehre sei dem \textsc{Vater} 'und dem \textsc{Sohne} \grestar{} und dem
'\textsc{Heiligen Geiste}.\par
\einzug{Wie im Anfang, so auch 'jetzt und allezeit \grestar{} und in 'Ewigkeit.
Amen.}
 \textit{$\rightarrow$ Antiphon}
\end{multicols} \vskip0.4em
\newpage
\section*{Psalmodie}
\ueberinitiale{1. \Abar}{IIIa}
\commentary{\textbf{Psalm 51}}
\includegabcscore{laudes-1ant.gabc}
\begin{multicols}{2}\setlength{\columnseprule}{0.2pt}
 Wasche mich rein von \underline{mei}ner 'Missetat: \grestar{} und reinige mich von \underline{mei}ner 'Sünde.\par
 \einzug{Denn ich erkenne \underline{mei}ne 'Missetat: \grestar{} und meine \underline{Sün}de ist 'stets vor mir.}
 An Dir allein habe \ulin{ich} ge'sündigt: \grestar{} und \ulin{ü}bel vor 'Dir getan.\par
 \einzug{Damit Du Recht behaltest in \underline{Dei}nen 'Worten: \grestar{} und rein dastehst, \underline{wenn} Du 'richtest.}
 Siehe, ich bin als \underline{Sün}der ge'boren: \grestar{} und meine Mutter hat mich in \underline{Sün}den emp'fangen.\par
 \einzug{Siehe, Dir gefällt Wahrheit, die \underline{im} Ver'borgnen liegt: \grestar{} und im Geheimen tust \underline{Du} mir 'Weisheit kund.}
 Entsündige mich mit Ysop, dass \underline{ich} rein 'werde: \grestar{} wasche mich, dass ich \underline{schnee}weiß 'werde.\par
 \einzug{Lass mich hören \underline{Freu}de und 'Wonne: \grestar{} dass die Gebeine fröhlich werden, die \underline{Du} zer'schlagen hast.}
 Verbirg dein Antlitz vor \underline{mei}nen 'Sünden: \grestar{} und tilge alle \underline{mei}ne 'Missetat.\par
 \einzug{Schaffe in mir, \underline{\textsc{Gott}}, ein 'reines Herz: \grestar{} und gib mir einen be\underline{stän}digen, 'neuen Geist.}
 Verwirf mich nicht von \underline{Dei}nem 'Angesicht: \grestar{} und nimm Deinen \underline{heil}gen Geist 'nicht von mir.\par
 \einzug{Erfreue mich wieder mit \underline{Dei}ner 'Hilfe: \grestar{} und mit einem willigen \underline{Gei}ste 'stärke mich.}
 Ich will die Übertreter Deine \underline{We}ge 'lehren: \grestar{} dass sich die Sünder zu \underline{Dir} be'kehren.\par
 \einzug{Er\-ret\-te mich von Blut\-schuld, \textsc{Gott}, der Du mein \underline{\textsc{Gott}} und 'Hei\-land bist: \grestar{} \hfill dass meine Zunge deine Ge\underline{rech}\-tigkeit 'rühme.}
 \textsc{Herr}, tue \underline{mei}ne 'Lippen auf: \grestar{} dass mein Mund deinen \underline{Ruhm} ver'kündige.\par
 \einzug{Denn du willst keine Schlachtop\-\ulin{fer} \gredagger{} ich wollte sie \underline{Dir} sonst 'ge\-ben: \grestar{} und dir gefallen \underline{kei}ne Brand'op\-fer.}
 Die Opfer, die \textsc{Gott} gefallen, sind eine ge\underline{beug}te 'Seele: \grestar{} ein geängstetes, zerschlagenes Herz wirst Du, \textsc{Gott}, \underline{nicht} ver'achten.\par
 \einzug{Tue wohl an Zion nach \underline{Dei}ner 'Gnade: \grestar{} baue die Mauern \underline{zu} Je'rusalem.}
 Dann werden Dir rechte Op\-fer gefal\ulin{len} \gredagger{} Brandop\-fer \underline{und} Ganz'op\-fer: \grestar{} dann wird man Stiere auf Deinem \underline{Al}\-tar 'op\-fern.\par
 \einzug{Ehre sei dem \textsc{Vater} \underline{und} dem '\textsc{Sohne}: \grestar{} und dem \underline{\textsc{Hei}}\textsc{ligen 'Geiste}.}
 Wie im Anfang, so auch \ulin{jetzt} und alle'zeit: \grestar{} und in \ulin{E}wigkeit. 'Amen.\par
  \textit{$\rightarrow$ Antiphon}
\end{multicols} \vskip0.4em
\newpage
\commentary{\textbf{Psalm 108}}
\ueberinitiale{2. \Abar}{IVb}
\includegabcscore{laudes-2ant.gabc}
\begin{multicols}{2}\setlength{\columnseprule}{0.2pt}
Wach auf, Psal\underline{ter} und 'Harfen: \grestar{} ich will das
\underline{Mor}genrot 'wecken.\par
 \einzug{Ich will Dir danken, \textsc{Herr}, un\underline{ter} den 'Völkern:
\grestar{} ich will dir lobsingen \underline{un}ter den 'Leuten.}
 Denn Deine Gnade reicht, so \underline{weit} der 'Himmel ist: \grestar{} und
deine Treue, so weit \underline{die} Wolken 'gehen.\par
 \einzug{Erhebe Dich, \textsc{Gott}, ü\underline{ber} den 'Himmel: \grestar{}
und Deine Herrlichkeit ü\underline{ber} alle 'Lande.}
 Lass Deine Freunde er\underline{ret}tet 'werden: \grestar{} dazu hilf mit
Deiner Rech\underline{ten} und er'höre uns.\par
 \einzug{\textsc{Gott} hat in Seinem Heiligtum geredet - ICH will
frohloc\underline{ken} \gredagger{} ICH will Si\underline{chem} ver'teilen:
\grestar{} und das Tal \underline{Suk}kot aus'messen.}
 Gilead ist mein, Manasse ist auch \underline{mein} \gredagger{} und Ephraim ist
der Schutz \underline{mei}nes 'Hauptes: \grestar{} Ju\underline{da} ist mein
'Zepter.\par
 \einzug{Moab ist mein Waschbec\underline{ken} \gredagger{} ich will meinen
Schuh auf \underline{E}dom 'werfen: \grestar{} über die Philis\underline{ter}
will ich 'jauchzen.}
 Wer wird mich führen \underline{in} die 'feste Stadt: \grestar{} wer wird mich
\underline{nach} Edom 'leiten?\par
 \einzug{Wirst Du es nicht tun, \textsc{Gott}, der Du \underline{uns}
ver'stoßen hast: \grestar{} und ziehst nicht aus, \textsc{Gott},
\underline{mit} unserm 'Heere?}
 Schaffe uns \underline{Bei}stand 'vor dem Feind: \grestar{} denn
Menschenhil\underline{fe} ist nichts 'nütze.\par
 \einzug{Mit \textsc{Gott} wol\underline{len} wir 'Taten tun: \grestar{} Er wird
unsere Fein\underline{de} nieder'treten.}
 Ehre sei dem \textsc{Vater} \underline{und} dem '\textsc{Sohne}: \grestar{} und
dem \textsc{\underline{Hei}ligen 'Geiste}.\par
 \einzug{Wie im Anfang, so auch \underline{jetzt} und 'allezeit: \grestar{} und
in \underline{E}wig'keit. Amen.}
  \textit{$\rightarrow$ Antiphon}
\end{multicols} \vskip0.4em
\commentary{\textbf{Psalm 147j}}
\ueberinitiale{3. \Abar}{Ia}
\includegabcscore{laudes-3ant.gabc}
\begin{multicols}{2}\setlength{\columnseprule}{0.2pt}
Der \textsc{Herr} baut \underline{auf} Je'rusalem: \grestar{} und bringt
zusammen die Ver\underline{streu}ten 'Israels.\par
 \einzug{Er heilet, die zer\underline{broch}enen 'Herzens sind: \grestar{} und
verbindet \underline{ih}re 'Wunden.}
 Er \underline{zäh}let die 'Sterne: \grestar{} und nennt sie al\underline{le}
mit 'Namen.\par
 \einzug{Unser \textsc{Herr} ist \underline{groß} und von 'großer Kraft:
\grestar{} und unbegreiflich ist, wie \underline{Er} re'gieret.}
 Der \textsc{Herr} richtet \underline{auf} die 'Elenden: \grestar{} und stößt
die Gottlo\underline{sen} zu 'Boden.\par
 \einzug{Singet umeinander dem \underline{Herrn} ein 'Danklied: \grestar{} und
lobet unsern \underline{\textsc{Gott}} mit 'Harfen.}
 Der den Himmel mit Wolken bedec\underline{ket} \gredagger{} und gibt
\underline{Re}gen auf 'Erden: \grestar{} der Gras auf den \underline{Ber}gen
'wachsen lässt.\par
 \einzug{Der dem \underline{Vieh} sein 'Futter gibt: \grestar{} den jungen
Raben, die \underline{zu} ihm 'rufen.}
 Er hat keine Freude an der \underline{Stär}ke des 'Rosses: \grestar{} und kein
Gefallen an den Schen\underline{keln} des 'Mannes.\par
 \einzug{Der \textsc{Herr} hat Gefallen an denen, \underline{die} IHN 'fürchten:
\grestar{} die auf Seine \underline{Gü}te hoffen.}
 Ehre sei dem \textsc{Vater} \ulin{und} dem '\textsc{Sohne}: \grestar{} und dem
\textsc{Hei\underline{li}gen 'Geiste.}\par
 \einzug{Wie im Anfang, so auch \underline{jetzt} und 'allezeit: \grestar{} und
in E\underline{wig}keit. 'Amen.}
  \textit{$\rightarrow$ Antiphon}
\end{multicols} \par
\ueberinitiale{4. \Abar}{Irr.}
\includegabcscore{laudes-4ant-ant.gabc}
\textit{Die Antiphon wird im Vers nicht wiederholt.}\par
\commentary{\textbf{Psalm 147ij}}
\includegabcscore{laudes-4ant-ps.gabc}
\begin{multicols}{2}\setlength{\columnseprule}{0.2pt}
Denn er macht die Riegel Deiner 'Tore fest: \grestar{} und segnet deine Kinder
\underline{in} der 'Mitte.\par
 \einzug{Er schafft deinen Grenzen 'Frieden: \grestar{} und sättiget dich mit
dem \underline{bes}ten 'Weizen.}
 Er sendet Sein Gebot auf die 'Erde: \grestar{} \underline{Sein} Wort 'läuft
schnell.\par
 \einzug{Er gibt Schnee wie 'Wolle: \grestar{} Er streuet \underline{Reif} wie
'Asche.}
 Er wirft seine Schloßen herab wie 'Brocken: \grestar{} wer kann bleiben
\underline{vor} 'Seinem Frost?\par
 \einzug{Er sendet Sein Wort, da 'schmilzt der Schnee: \grestar{} Er lässt
Seinen Wind wehen, \underline{da} 'tauet es.}\vfill\columnbreak
 Sein Wort hat Er Jakob ver'kündet: \grestar{} Israel Seine Gebo\underline{te}
'und Sein Recht.\par
 \einzug{So hat Er an keinem Volk 'getan: \grestar{} sie kennen
Sei\underline{ne} 'Rechte nicht.}
 Ehre sei dem \textsc{Vater} und dem '\textsc{Sohne}: \grestar{} und dem
\textsc{Hei\underline{li}gen 'Geiste}.\par
 \einzug{Wie im Anfang, so auch jetzt und 'allezeit: \grestar{} und in
E\ulin{wig}keit. 'Amen.}
  \textit{$\rightarrow$ Antiphon}
\end{multicols} \vskip0.4em
\section*{Lektion} \subsection*{Conclusio}
\ueberinitiale{sonn-}{tags}
\includegabcscore{conclusio-la-so.gabc}
\ueberinitiale{werk-}{tags}
\includegabcscore{conclusio-la-we.gabc}
\newpage
\section*{Responsorium}
\ueberinitiale{So}{IV}
\includegabcscore{laudes-rs-so.gabc}\par
\ueberinitiale{Mo-Sa}{IV}
\includegabcscore{laudes-rs-we.gabc}\newpage
\section*{Hymnus}
\ueberinitiale{So}{VIII}
\includegabcscore{laudes-h-so.gabc}\par
\begin{multicols}{2}\setlength{\columnseprule}{0.2pt}
Du bist des Wandrers nächtlich Licht, / führst auch zum Ende alle Nacht; //
schon tönt des Hahnes Morgenruf / und lockt hervor der Sonne Strahl.\par
 \einzug{Sein Ruf erweckt den Morgenstern, / macht frei die Welt von Finsternis.
// Der bösen Geister Heer entflieht, / verlässt den Weg der Trug und List.}
 Der Seemann schöpfet neuen Mut, / die Meereswogen glätten sich. // Der Kirche
Fels vernahm den Schrei, / bereut nun alle seine Schuld.\par
 \einzug{Erheben wir uns deshalb schnell! / Die Schlummernden erweckt der Hahn,
// er klagt die trägen Schläfer an, / und, die vergessen ihre Pflicht.}
 Beim Hahnenschrei zieht Hoffnung ein, / Genesung strömt dem Kranken zu. // Der
Räuber nimmt die Waffe weg / und neu erglänzt des Glaubens Licht.\par
\einzug{\textsc{Jesu}, uns Wankende sieh an, / durch Deinen Blick verleih uns
Kraft; // so sinke unsre Sündenlast, / in Tränen schwinde alle Schuld.}
 Du Licht, erleuchte unser Herz, / vertreibe allen Geistesschlaf: // Dir sei das
erste Lied geweiht / als Dank, den wir dir schuldig sind.\par
 \einzug{\emph{\Abar} Lob sei dem \textsc{Vater} auf dem Thron / und Seinem
eingebornen \textsc{Sohn}, // dem \textsc{Heilgen Geist} auch allezeit / von nun
an bis in Ewigkeit. Amen.}
\end{multicols}
\ueberinitiale{Mo-Sa}{I}
\includegabcscore{laudes-h-we.gabc}\par
\begin{multicols}{2}\setlength{\columnseprule}{0.2pt}
Beim Aufstehn reich uns Deine Hand, / es stehe nüchtern auf der Geist, // er
bring, entflammt zum \textsc{Gott}eslob, / den Dank, den wir Dir schuldig
sind.\par
 \einzug{Schon leuchtet auf der Morgenstern / und schreitet vor der Sonn einher;
// der Nächte Nebel fallen tief, / in uns erstrahle heilges Licht.}
 Es bleib in unsern Herzen stets / und treib hinweg die Nacht der Welt, // und
bis zum Ende aller Zeit / bewahre es die Seele rein.\par\vfill\columnbreak
 \einzug{In unsern Herzen wurzle ein / der Glaube, der zuerst gepflanzt, // dann
soll die Hoffnung uns erfreun / und größer noch die Liebe sein.}
 \emph{\Abar} Lob sei dem \textsc{Vater} auf dem Thron / und Seinem eingebornen
\textsc{Sohn}, // dem \textsc{Heilgen Geist} auch allezeit / von nun an bis in
Ewigkeit. Amen.\par
\end{multicols} \vskip0.4em
\section*{Versikel}
\ueberinitiale{sonn-}{tags}
\includegabcscore{laudes-ver-so.gabc}
\ueberinitiale{werk-}{tags}
\includegabcscore{laudes-ver-we.gabc}
\ueberinitiale{Micha-}{elis}
\includegabcscore{laudes-ver-mi.gabc}
\ueberinitiale{Himmel-}{fahrt}
\includegabcscore{laudes-ver-hi.gabc}\newpage
\section*{Canticum Zachariae}
\ueberinitiale{\Abar}{VII a}
\commentary{\small\textit{Lk 1, 68-79}}
\includegabcscore{laudes-cz.gabc}
\begin{multicols}{2}\setlength{\columnseprule}{0.2pt}
Und hat uns aufgerichtet ein \underline{Horn} des 'Heiles: \grestar{} in dem
Hause seines \underline{Die}ners 'David.\par
\einzug{Wie er vor\underline{zei}ten ge'redet hat: \grestar{} durch den Mund
seiner \underline{heil}gen Pro'pheten.}
Dass er uns errettete von \underline{un}sern 'Feinden: \grestar{} und von der
Hand aller, \underline{die} uns hassen.\par
\einzug{Und Barmherzigkeit erzeigete \underline{un}sern 'Vätern: \grestar{} und
gedächte Seines \underline{hei}ligen 'Bundes.}
Des Eides, den Er geschworen hat unserm \underline{Va}ter 'Abraham: \grestar{}
\underline{uns} {\tiny$\downarrow$}zu 'geben.\par
\einzug{Dass wir, erlöset aus der Hand \underline{uns}rer 'Feinde: \grestar{}
IHM dienten ohne Furcht \underline{un}ser 'Leben lang.}
In Heiligkeit \underline{und} Ge'rechtigkeit: \grestar{} die \underline{IHM}
ge'fällig ist.\par
\einzug{Und du, Kindlein, wirst ein Prophet des \underline{Höchs}ten 'heißen:
\grestar{} du wirst vor dem Herrn hergehen, dass du Seinen \underline{Weg}
be'reitest.}
Und Erkenntnis des Heiles \underline{ge}best 'Seinem Volk: \grestar{} in
Vergebung \underline{ih}rer 'Sünden.\par
\einzug{Durch die herzliche Barmherzigkeit \underline{un}sers '\textsc{Gott}es:
\grestar{} durch welche uns besucht hat der Aufgang \underline{aus} der 'Höhe.}
Auf dass Er erscheine denen, die da sitzen in Finsternis und
\underline{Schat}ten des 'Todes: \grestar{} und richte unsre Füße auf den
\underline{Weg} des 'Friedens.\par
\einzug{Ehre sei dem \textsc{Vater} \underline{und} dem '\textsc{Sohne}:
\grestar{} und dem \textsc{\underline{Hei}ligen 'Geiste.}%
}
Wie im Anfang so auch \underline{jetzt} und 'allezeit: \grestar{} und in
\underline{E}wigkeit. 'Amen.\par
 \textit{$\rightarrow$ Antiphon}
\end{multicols}
\section*{Orationes}
\ueberinitiale{sonn-}{tags}
\includegabcscore{laudes-o1.gabc}
\ueberinitiale{werk-}{tags}
\includegabcscore{laudes-o2.gabc}
\subsection*{Preces (werktags)}
\includegabcscore{preces.gabc}
{\itshape\small Im Wechsel weiter:}
\begin{multicols}{2}\setlength{\columnseprule}{0.2pt}
\Vbar \textsc{Herr}, erweise uns Deine Gnade.\par
 \Rbar Und schenke uns Dein Heil.\par
 \Vbar \textsc{Herr}, kehre dich doch wieder zu uns,\par
 \Rbar Und sei deinen Knechten gnädig.\par
 \Vbar Deine Güte, \textsc{Herr}, sei über uns.\par
 \Rbar Wie wir auf Dich hoffen.\par
 \vskip0.4em\hrule\vskip0.4em
\Vbar Lasset uns beten für die heilige Kirche Gottes.\par
 \Rbar \textsc{Herr}, tue wohl an Zion nach \underline{Dei}ner Gnade / baue die
Mauern zu Jerusalem.\par
 \Vbar Es möge Friede sein in deinen Mauern,\par
 \Rbar Und Glück in deinen Palästen.\par
 \Vbar Deine Priester lass sich kleiden mit Gerechtigkeit.\par
 \Rbar Und deine Heiligen sich freuen.\par
 \Vbar Lasset uns beten für unsere Hirten und Lehrer.\par
 \Rbar \textsc{Herr}, nimm nicht von ihrem Munde das Wort der Wahrheit.\par
 \Vbar Lass sie auftreten und weiden in deiner Kraft.\par
 \Rbar Und in der Macht Deines Namens, \textsc{Herr}, unser \textsc{Gott}.\par
 \Vbar Lasset uns beten für alle, die im Glauben unterwiesen werden.\par
 \Rbar \textsc{Herr}, lass sie wachsen in der Gnade und Erkenntnis des Herrn
\textsc{Jesus Christus}.\par
 \Vbar Für die Heimführung des Volkes Israel.\par
 \Rbar Nimm weg, \textsc{Herr}, die Decke von \underline{sei}nem Herzen~/ dass
es sich zu Deinem Sohne bekehre.\par
 \Vbar Für die Ausbreitung des Evangeliums unter den Heiden.\par
 \Rbar Sende Arbeiter in \underline{Dei}ne Ernte / dass alle Menschen zur
Erkenntnis der Wahrheit kommen.\par
 \Vbar Lasset uns beten für unser Volk.\par
 \Rbar Hilf Du uns, \textsc{Gott}, unser Helfer, um Deines Namens Ehre
willen.\par
 \Vbar Für alle Regierenden.\par
 \Rbar \textsc{Herr}, gib ihnen Weisheit und Einsicht gerecht \underline{zu}
regieren / dass Dein Wort geehret werde.\par
 \Vbar Für die Fruchtbarkeit der Erde.\par
 \Rbar Suche das Land heim und wässere es und segne sein Gewächs.\par
 \Vbar Für den Frieden der ganzen Welt.\par
 \Rbar \textsc{Herr}, lass Deine Hilfe nahe sein denen, die Dich fürchten.\par
 \Vbar Dass Güte und Treue einander begegnen.\par
 \Rbar Gerechtigkeit und Friede sich küssen.\par
 \Vbar Gedenke, \textsc{Herr}, Deiner Gemeinde.\par
 \Rbar Die Du vor Zeiten erworben hast.\par
  \vskip0.4em\hrule\vskip0.4em
 \Vbar Breite deine Güte über die, die dich kennen.\par
 \Rbar Und Deine Gerechtigkeit über die Frommen.\par
 \Vbar Lasset uns beten für die Elenden und Betrübten.\par
 \Rbar \textsc{Herr}, stehe ihnen bei und tröste sie.\par
 \Vbar Für die Witwen und Waisen.\par
 \Rbar \textsc{Herr}, lass Deine Güte und Treue allewege sie behüten.\par
 \Vbar Für die Kranken.\par
 \Rbar \textsc{Herr}, erquicke sie nach Deiner Gnade.\par
 \Vbar Lasset uns beten für unsere Widersacher und Verfolger.\par
 \Rbar \textsc{Herr}, behalte ihnen \underline{die}se Sünde nicht, / denn sie
wissen nicht, was sie tun.\par
 \Vbar Für die Abtrünnigen und Verirrten.\par
 \Rbar \textsc{Herr}, weise ihnen den Weg und leite sie auf
richti{\tiny$\downarrow$}ger {\tiny$\uparrow$}Bahn.\par
 \Vbar Für die Gefangenen und Angefochtenen.\par
 \Rbar Erlöse sie, \textsc{Gott} Israel, aus aller ihrer Not.\par
 \Vbar Sende ihnen Hilfe vom Heiligtum.\par
 \Rbar Und stärke sie aus Zion.\par
 \Vbar Lasset uns beten für alle unsre Wohltäter.\par
 \Rbar Gewähre, \textsc{Herr}, allen \underline{die} uns Gutes tun / um Deines
Namens willen das ewige Leben.\par
 \Vbar Für alle Reisenden.\par
 \Rbar Erhöre \underline{uns,} \textsc{Gott}, unser Heil / der Du bist
Zuversicht aller auf Erden und fern am Meere.\par
 \Vbar Für die abwesenden Brüder \textit{(und Schwestern)}.\par
 \Rbar Hilf Du, mein \textsc{Gott}, Deinen Knechten, die sich auf dich
verlassen.\par
 \Vbar Für die Sterbenden.\par
 \Rbar In Deine Hände, \textsc{Herr}, befehlen wir ihren Geist.\par
 \Vbar Lehre uns bedenken, dass wir sterben müssen.\par
 \Rbar Damit wir klug werden.\par
 { \itshape \Vbar Für den / die im Glauben entschlafene/n N. N.\par
 \Rbar \textsc{Herr}, gib ihm / ihr die ewige Ruhe / und das ewige Licht leuchte
ihm / ihr.}\par
 \vskip0.4em\hrule\vskip0.4em
 \Vbar Hilf, \textsc{Herr}, deinem Volke und segne Dein Erbe.\par
 \Rbar Weide die Deinen und trage sie ewiglich.\par
 \Vbar \textsc{Herr} \textsc{Gott} Zebaoth, tröste uns.\par
 \Rbar Lass leuchten dein Antlitz, so genesen wir.\par
 \Vbar Mache Dich auf, CHRISTUS, und hilf uns.\par
 \Rbar Erlöse uns um Deiner Güte willen.\par
 \Vbar \textsc{Herr}, höre mein Gebet.\par
 \Rbar Und lass mein Schreien zu Dir kommen.\par
\end{multicols} 
\subsection*{Collecte und Salutatio\label{Salutatio}}
\includegabcscore{laudes-coll.gabc}
\includegabcscore{laudes-colsa.gabc}
\begin{multicols}{2}\setlength{\columnseprule}{0.2pt} \textbf{Sonntag}\par
 \textit{Gebet des Sonntags nach dem Kirchenjahr.} \ding{118} \ding{166} \ding{167}\par
 \textbf{Montag}\par
 \textsc{Herr}, auf Deiner himmlischen Gnade steht allein unsre Hoffnung / darum bitten wir Dich, erhöre freundlich das Flehen Deines Volkes und bewahre uns mit himmlischen Schutze. \ding{118}\par
 \textbf{Dienstag}\par
 \textsc{Herr}, \textsc{Gott}, wir bitten Dich, bewahre die Herzen Deiner Gläubigen und stärke sie mit der Kraft Deiner Gnade / damit sie beständig vor Dir beten und einander wahrhaftig lieben. \ding{118}\par
 \textbf{Mittwoch}\par
 \textsc{Herr}, höre gnädig unser Flehen und hilf Du selbst unsrer Schwachheit auf / vergib uns unsre Schuld, damit wir uns Deiner Barmherzigkeit unser Leben lang freuen. \ding{118}\par
 \textbf{Donnerstag}\par
 Wir bitten Dich, \textsc{Herr}, erhöre das Flehen Deiner Kirche und schenke ihr Vergebung der Sünden / damit sie fromm werde durch Dein Wirken und unter Deinem Schutze sicher sei. \ding{118}\par
 \textbf{Freitag}\par
 \textsc{Herr}, stehe denen bei, die zu Dir beten und schütze gnädig, die allein auf Deine Barmherzigkeit hoffen / damit sie, von Sünden gereinigt, ein heiliges Leben führen. \ding{118}\par
\textbf{Samstag}\par
 \textsc{Herr} \textsc{Gott}, wir bitten Dich, Deine Rechte schütze das Volk, das zu Dir betet / damit es dieses Leben im Gehorsam führe und so das ewige Leben erlange. \ding{118}\par
\end{multicols} \subsection*{Conclusio}
\includegabcscore{laudes-o3.gabc}
\vskip0.4em
\ding{118} \Vbar \textit{an \textsmcpit{Gott Vater}:}\\
 Durch unsern Herrn \textsc{Jesus Christus}, Deinen Sohn: der mit Dir in der Einheit des \textsc{Heiligen Geistes} ein wahrer \textsc{Gott} / lebet und regieret von Ewigkeit zu Ewigkeit. \Rbar Amen.\par
\ding{166} \Vbar \textit{an \textsmcpit{Gott Sohn}:}\\ der Du mit dem \textsc{Vater} in der Einheit des \textsc{Heiligen Geistes} ein wahrer \textsc{Gott} / lebest und regierest von Ewigkeit zu Ewigkeit. \Rbar Amen.\par
\ding{167} \Vbar \textit{\textsmcpit{Jesus} wird genannt:}\\ Durch IHN, unsern Herrn \textsc{Jesus Christus}, Deinen Sohn: der mit Dir in der Einheit des \textsc{Heiligen Geistes} ein wahrer \textsc{Gott} / lebet und regieret von Ewigkeit zu Ewigkeit. \Rbar Amen.\par\newpage
\section*{Benedicamus}
\includegabcscore{laudes-coll.gabc}
\ueberinitiale{sonn-}{tags}
\includegabcscore{laudes-b1.gabc}
\ueberinitiale{werk-}{tags}
\includegabcscore{laudes-b2.gabc}
\vskip1em \Vbar Der \textsc{Herr} gebe uns Seinen Frieden. \Rbar Und das ewige Leben. Amen.\par
\begin{center}
{\centering{\scalebox{3}{\grecross}}}
\end{center}
