\documentclass[medstaff,decorativeinitials,fontsize=14pt,paper=a4]{gregorian}
\renewcommand{\rmdefault}{ugm}
\renewcommand{\sfdefault}{uop}
\usepackage{gregoriotex}
\begin{document}
\chapter*{Einführung in das Gregorianische Brevier}
Im vorliegenden Buch findet man eine Auswahl aus dem \textit{Breviarium Lipsiens\ae{}}, dem Leipziger Brevier der Evangelisch-Lutherischen Gebetsbruderschaft. Diese Auswahl ist so getroffen, dass man ganzjährig die folgenden vier Gebete in einer Grundform beten kann:\par
\begin{enumerate}
 \item Laudes -- das Morgenlob
\item Sext -- das Mittagsgebet
\item Vesper -- das Abendgebet und
\item Complet -- das Nachtgebet.
\end{enumerate}
Dabei sind die ersten drei Gebete teilweise an die Kirchenjahreszeit angepasst; der Charakter eines Auszugs verbietet aber eine vollständige Anpassung, andernfalls läge mit diesem Band eine Kopie des Breviers vor.
\section*{Bestandteile}

\end{document}
